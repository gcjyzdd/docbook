
\chapter{Processes and Threads}





\section{Introducing Process}

Q: \textit{What is a process exactly?} \\
Q: \textit{Is it possible to have more than one process? How?}  \\
Q: \textit{How does the operating system' code get executed?}

Right now, the operating system sees a program \textit{during execution} 
an ongoing sequence of machine code instructions. Before code is executed 
a program can be seen as high-level source code but during 
execution the operating system doesn't have anything to describes its execution. 
There is nothing that captures its execution state, memory usage or register contents. 

The problem is that we \textit{need} this higher level view of a program running on a processor.
The execution of a program machine code instructions will lead to it requiring io devices.
Recall that use of io devices is complex and expensive. 
The operating system as is resonsible for managing io devices as resources that 
are to be allocated to programs. It acts as a resource manager

So, as a resource manager, an operating system should know how programs in 
execution affect resources. This is done with the idea of a \textit{process}. 
A \textit{process} is an invention by an operating system. 
It is used as a placeholder for executing programs so that we can keep a precise 
account of how program execution is affecting the use of hardware and software resources.

\highlightdef{A \textbf{Process} is the os's view of a program in execution}

It's an abstraction of a running program that's useful for the operating system.
As a virtual machine, the os needs to provide the user with easy access to the 
resources. As a processor-helper it acts as software that controls the
medium between the processor and devices.

\highlightdef{A \textbf{Process Control Block} or \textit{PCB} is a data structure 
the os uses to store information about a process }

We will see nthat we can have processes running in a somewhat simulatenous manner.
Here, the operating system will allocates a process control block for each process running. 


The PCB contains a \textit{huge amount of information} in
one \textit{large} structure. It is a heavyweight abstraction.


\begin{example}
In Unix, a process is represented by a struct called \lstinline{task}. 
Below shows a definition of a typical Unix \lstinline{task} struct.  
\end{example}

Within the unix communicity this struct is known as a \textit{process descriptor}
or a \textit{task descriptor}.

\highlightdef{The operating system executes its own code using \textit{kernel processes}} 



\highlightdef{A \textbf{Context Switch} is }



\section{Unix Fork-Exec Model}


\highlightdef{The \lstinline{fork()} system call creates a copy of the current process}

The contents of all the registers will be preserved. The exact execution context 
will exist for both the parent process and the newly created process.
\begin{example}
\begin{lstlisting}
void example()
{
	printf("E");
	fork();
	printf("F");
}
\end{lstlisting}
\end{example}

There is
\textit{one difference}. The return result of \lstinline{fork} is different.
Below shows an example.

\begin{lstlisting}
void example()
{
	pid_t pid = fork();
	if (pid != 0) printf("P");
	else          printf("C");
}
\end{lstlisting}

\frmrule

\begin{example}
Explain what the following 
means: \textit{A single call to function \lstinline{fork()} returns twice}.
In particular explain why we can think of this function as returning 
twice. 
\end{example}

\frmrule

\begin{example}
For the following code, state what is printed, and precisely in what order.
Explain your answer. State how many processes are running 
after \lstinline{example()} returns.
\begin{lstlisting}
void example()
{
	pid_t pid = fork();
	if (pid != 0) printf("P");
	else          printf("C");
	printf("X");
}
\end{lstlisting}
\end{example}

\frmrule


\begin{example}
For the following code, state what is printed, and precisely in what order.
Explain your answer. State how many processes are running 
after \lstinline{example()} returns.
\begin{lstlisting}
void example()
{
	pid_t pid1 = fork();
	if (pid1 != 0) {
		printf("A")
	} else {
		pid_t pid2 = fork();
		if (pid2 != 0) {
			printf("B");
		} else {
			printf("C");
		}
	} 
}
\end{lstlisting}
\end{example}


\frmrule

\begin{example}
For the following code, state what is printed, and precisely in what order.
Explain your answer. State how many processes are running 
after \lstinline{example()} returns.
\begin{lstlisting}
void example()
{
	pid_t pid1 = fork();
	if (pid1 != 0) {
		printf("A")
	} else {
		pid_t pid2 = fork();
		if (pid2 != 0) {
			printf("B");
		} else {
			printf("C");
		}
		printf("Y");
	} 
	printf("X");
}
\end{lstlisting}
\end{example}

\frmrule

\begin{example}
For the following code, state what is printed, and precisely in what order.
Explain your answer. State how many processes are running 
after \lstinline{example()} returns.
\begin{lstlisting}
void example()
{
	pid_t pid1 = fork();
	if (pid1 != 0) {
		printf("X")
	}
	pid_t pid2 = fork();
	if (pid2 != 0) {
		printf("Y")
	}
}
\end{lstlisting}
\end{example}

\frmrule

\begin{example}
For the following code, state what is printed, and precisely in what order.
Explain your answer. State how many processes are running 
after \lstinline{example()} returns.
\begin{lstlisting}
void example()
{
	pid_t pid1 = fork();
	if (pid1 != 0) {
		printf("X")
	}
	printf("S")
	pid_t pid2 = fork();
	if (pid2 != 0) {
		printf("Y")
	}
	printf("T")
}
\end{lstlisting}
\end{example}



\highlightdef{The \lstinline{execve()} system call
 replaces the current process with a new process}
By this we mean that the memory allocated for a process is replaced with new 
instructions and new data. There is a large family of \lstinline{exec} calls.

\frmrule

\begin{example}
For the following code, state what is printed, and precisely in what order.
Explain your answer. State how many processes are running 
after \lstinline{example()} returns.

\begin{lstlisting}
void example()
{
	pid_t pid = fork();
	if (pid != 0) {
		printf("P");
	} else {
		printf("C");
		execv('./prog', 5, 'hello');
	}       
	printf("X");
}
\end{lstlisting}
\end{example}



\section{Execution-Schedule Management}

The job of a \textit{scheduler} is to arbitrate access to the 
current CPU between multiple processes.
The header file include/linux/sched.h is included (either explicitly or indirectly) by
virtually every kernel source file.

The Linux kernel maintains a global variable called \lstinline{jiffies}. 
This represents the number of timer ticks since the machine started. 
This variable is initialized to zero and increments each timer interrupt.

Below shows an example that uses \lstinline{get_jiffies_64}. This reads the jiffies 
variable. This value can be converted to a millisecond value using 
\lstinline{jiffies_to_msecs}.





\section{Interprocess Communication I}


\highlightdef{\textbf{Interprocess Communication} (IPC)
describes methods for communication by sending data and sharing memory} 

Recall that Unix is written in C. From a source-code perspective, 
a program has a standard output stream and a standard input stream. 

\highlightdef{A \textbf{Unix Pipe} connects the \textit{standard output} of $P_1$ 
to the \textit{standard output} of $P_2$ }

Unix pipes therefore provide a mechanism
for \textit{one-way communication} betweeen $P_1$ and $P_2$ where 
$P_1$ is sending data to $P_2$. 

We can create a pipe using the \lstinline{pipe()} system call.




\begin{example}
Below shows an example of a Unix pipe used in shell command \lstinline{ls | wc -l}.

\begin{figure}[h]
\begin{tikzpicture}[
  process/.style={rectangle, draw, minimum height=2em},
  pipe/.style={cylinder, draw, minimum height=3cm, minimum width=0.5cm}
]
  \node [process] (ls) {\lstinline{ls}};
  \node [pipe, right=0.2cm of ls] (pipe1) {};
  \node [process, right=0.2cm of pipe1] (wc) {\lstinline{wc}};
  \draw [draw, dashed, ->] (ls) -- (wc);
\end{tikzpicture}
\end{figure} 
\end{example}


(Anonymous) pipes can only be used by related 
processes
FIFOs == pipe with name in file system
Any process can open and use FIFO
I/O is same as for pipes

A named pipe works much like a regular pipe, but does have
some noticeable differences.
* Named pipes exist as a device special file in the file
system.
* Processes of different ancestry can share data through a
named pipe.
* When all I/O is done by sharing processes, the named 
pipe remains in the file system for later use.


\begin{example}
Here is another shell command \lstinline{ls -l | tee file.txt | less}.
Technically tee is a process but semantically it is a nice to
visualise this as a T. 
\end{example}



\section{Interprocess Communication II}

\section{Introducing Threads}