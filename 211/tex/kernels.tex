
\chapter{Operating Systems}


\section{Introducing Operating Systems}

An operating system is a \textit{unified software solution for many problems} 
that come about in managing the \textit{execution of machine code instructions}.

In particular, we are concerned with how \textit{a user can write programs} for the machine 
by telling \textit{the processor where the program is} and, how we \textit{manage the
physical resources} of the machine during this execution.  
An operating system is software that ties together three entities. 

\begin{itemize}   
\renewcommand{\labelitemi}{$\Box$}
\item \textbf{Processor} 
The cpu needs help with knowing what instructions to be executed.
\item \textbf{IO Devices} 
The machine's physical resources.
\item \textbf{User} 
The user needs help providing his program to the cpu to be executed.
\end{itemize}

We have three views of what an operating system is.

\begin{itemize}   
\renewcommand{\labelitemi}{$\Box$}
\item \textbf{Processor Controller} 
The operating system can be seen as software sitting on top of the cpu.
\item \textbf{IO Manager}
The operating system as is resonsible for managing \textsc{io} 
devices as \textit{resources}. \textsc{io} devices operate at a
much slower speed than the processor. As a result, 
any communication with an \textsc{io} device isn't cheap which is why we think of 
\textsc{io} devices as limited resources.
The operating system acts as a \textit{resource manager}. 
The machine's physical resources are to be allocated to programs. 
software protects against simultaneous accesses
and usage of resources (one user of the floppy disk at a time)
fair share of the resources (scheduling) account for using resources.
\item \textbf{Virtual Machine} 
The operating system can be seen as an abstraction sitting on top of hardware. 
It can be seen as a central controller for the processor, memory and \textsc{io} devices.
The user need only be presented with simple software interface for 
allowing his/her program to execute. A simple interface for controlling the machine.
The user doesn't need to know how the internals of the cpu works. 
Nor does the user need to worry about how the hardware is connected together. 
Everything about the hardware is abstract away. 
This abstraction that hides all hardware from the user to provide a 
purely software-driven view of machine known as a \textit{virtual machine}.
\end{itemize}

Responsibilities of a modern operating system.
\begin{itemize}   
\renewcommand{\labelitemi}{$\Box$}
\item \textbf{Memory Management} %link to above three views%
\item \textbf{Execution-Schedule Management} 
\item \textbf{File System Management} 
\end{itemize}



\section{Introducing Kernels}

Operating system designers have the large task of 
organising a huge amount of code. 
A part of structuring operating systems,
operating system designers find it helpful to follow the idea of having a \textit{kernel}.

\highlightdef{The \textbf{Kernel} of an operating a low-level 
software layer of the \textsc{os}}

An operating system has only one kernel.
It's considered the \textit{lowest layer} 
when we view the operating system source code as a whole. 
The kernel can be a thick layer or a relatively very thin layer. 






By defining the term \textit{kernel}, we are also highlighting
the os's role as a virtual machine. 
By having a kernel, we hide certain details of the hardware from the user to present an 
abstract programming interface. Often code written outside the kernel is 
highly compatible with other \textsc{io} devices because of this kernel interface. 

A kernel interface provides special functions known as \textit{system calls}.
The interface also provides general library functions for higher os layers. 



\highlightdef{An \textsc{os} appears to users/applications as 
a library of kernel functions}





Unix systems have \textit{several hundred} system calls, but, each one 
sits between the user and the hardware to provide a virtual machine interface.  

\begin{example}
One Unix system call is \lstinline{time()} that returns the number of milliseconds 
since January 1st, 1970 (the \textit{\textsc{UNIX} Epoch}). The user doesn't need to know 
how the operating system worked this out. It relies on \textit{hardware interrupts}
and a notion of a \textit{timer device}, but, this is all hidden from this user. 
\end{example}

\begin{example}
Linux has two system calls: \lstinline{create_module()} and \lstinline{query_module()}.
The \lstinline{create_module()} system call lets the user package code into
what is called a \textit{linux module} to be loaded into memory.
The \lstinline{query_module()} allows the user to request information about 
the linux modules that have already been packaged and loaded into memory.
\end{example}


\highlightdef{A \textbf{System call} is a high-level operating system function. }

When the user calls a system call, we can think of this as the user making 
requests to the operating system. In fact system calls are sometimes called
\textit{user functions} because of how well they provide an interface for the user. 
This allows a user to \textit{build programs of his/her own} on top of this system call layer. 
In this view, system calls provide an \textit{application programming interface} 
(\textsc{api}) 
to the user.

\highlightdef{System calls provide an \textit{application programming interface} 
for the user.}

However system calls are not just for the user. The kernel itself 
uses system calls. 

\frmrule

On an x86, there are six registers that are used for the arguments that
the system call takes. The first argument goes in \textsc{ebx}, the
second in \textsc{ecx}, then \textsc{edx}, \textsc{esi}, \textsc{edi}, and 
finally \textsc{ebp}. If
more then 6 arguments needed (not likely), the \textsc{ebx}
register must contain the memory location where the list of arguments is stored.

The interrupt call ‘0x80’ call to a \textit{system call handler} (sometimes
called the \textit{call gate}).
• The system call handler in turns calls the system call interrupt
service routine (\textsc{isr}).
• To perform Linux system calls we have to do following:
– Put the system call numberin \textsc{eax} register.
– Set up the argumentsto the system call in \textsc{ebx}, \textsc{ecx}, etc.
– call the relevant interrupt (for \textsc{dos}, 21h; for Linux,
80h).
– The result is usually returned in \textsc{eax}.


\begin{example}

\begin{lstlisting}
printf()
\end{lstlisting}
\end{example}





\begin{itemize}   
\renewcommand{\labelitemi}{$\Box$}
\item \textbf{Solaris} 
Completely modular kernel
\item \textbf{BSD} 
\item \textbf{Linux} 
\item \textbf{Minix} 
\item \textbf{Windows} 
\end{itemize}

Unix has the notion of \textit{kernel space} and \textit{user space}. 
\highlightdef{\textbf{User Space} is source code outside the kernel. \\
\textbf{Kernel Space} is source code inside the kernel}
We similarly think of code above the kernel as \textit{user level} code and 
code inside the kernel as \textit{kernel level} code.
For instructions in exection, the context is either in 
\textit{user mode} and \textit{kernel mode}.

This distinction of user versus kernel
 was done to help organise a large codebase but also provide security/safety.
When we look at memory later, we will see that there protection of 
memory between user process and kernel processes. Any process that attempts to access 
kernel memory will be terminated. We say that the kernel memory is protected from the user. 

\begin{example}
Explain the difference between \textit{user mode} and \textit{kernel mode}

In user mode, execution is restricted in that not all instructions can be executed, nor can all registers or memory locations be accessed. In contrast, in kernel mode, execution can make use of all facilities offered by the hardware.
\end{example}


\begin{example}
\textsc{minix} is organized as a client-server operating system. What does this mean?

It means that \textsc{minix} consists of a (relatively small) kernel whose main task is to establish communication between client applications and specific service programs. These service programs run
as normal processes (i.e., in user mode), but there is no way that a client application can directly
communicate with them: the kernel always sits in between.
\end{example}



\section{\textsc{unix} Special Files}

\highlightdef{In \textsc{unix}, \textit{everything} is file}

Most operating system have \textit{normal files}.
Any structure is determined by positions of 
binary digits in the file. It is up to the reader to 
parse the file to determine the structure. These files 
can are independent of the operating system. 

\textsc{unix} however, has the notion of \textit{special files}.
These can be viewed as operating system constructs that 
have a much richer structure and are tied to the operating system 
source code. Special files are not normal files with additions. 
Special files are data strctures with a rich structure.

\begin{figure}[h]
\begin{tikzpicture} [
   every node/.style={node_sty, align=center},
   edge from parent/.style={edge_sty},
   edge from parent path={(\tikzparentnode.south) 
   |- ($(\tikzparentnode.south)!0.5!(\tikzchildnode.north)$) -| (\tikzchildnode.north)},
   level 1/.style={sibling distance=6cm, level distance = 0.9cm},
   level 2/.style={sibling distance=1cm},
   ch/.style={text width=0.5cm}
]
   \node 
   (root) {\textsc{unix} Files}
      child {
        node (pio) {Normal Files}
        child {
            node(n)[ch] {\lstinline{-}}
        }
      }
      child {
        node(intio) {Special Files}
        child {node(d)[ch] {\lstinline{d}}}
        child {node(l)[ch] {\lstinline{l}}}
        child {node(p)[ch] {\lstinline{p}}}
        child {node(c)[ch] {\lstinline{c}}}
        child {node(b)[ch] {\lstinline{b}}}
        child {node(s)[ch] {\lstinline{s}}}
      };
   
\end{tikzpicture}
\end{figure}


A \textit{file descriptor} an integer used for accessing to uniquely 
identifer \textsc{unix} files whether they be normal files or special files.