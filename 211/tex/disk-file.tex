
\chapter{Disk and File}


\section{Introducing Disk Systems}




\highlightdef{A \textbf{Disk Cylinder} is a collection of}

In other words, a \textit{cylinder} is a collection of tracks 
with the same radius but on different platters.


\section{Unix Fast File System}

Original UNIX File System (UFS).
Simple, elegant, but \textit{slow}.


Each file is described by an \textit{index node}.
Index nodes are data structures used to store information about each file
An \textit{inode} may contain references to single, double 
and triple indirect blocks.

Every file has one index nodes that is assigned unique number.




\highlightdef{An \textbf{index node} (inode) is the root of a tree of references to 
the blocks of a file}.

\begin{example}
For a particular filesystem, an inode contains 8 direct pointers, 
1 pointer to an indirection block and 1 pointer to a 
two-level indirection block. 
Each pointer is $p$ bytes. Each disk block is $b$ bytes.
Each indirection block has $k$ pointers.

For this file system, \\
(a) determine the maximum file size \\
(b) determine the disk space required to store this file
\end{example}

\frameans{}{
	(a) $(8+k+k^2)b$ \\
	(b) $(8+k+k^2)b + (10+2k+k^2)p$
}

Drawing a diagram is useful for this question. 

\begin{tikzpicture}[
      start chain=1 going right,
      start chain=2 going right,
      node distance=0mm,
      inode/.style={draw, minimum height=1.4em,
      text width=2em, text centered, inner sep=1.7pt}
  ]
  % Index node
  \node[on chain=1, inode, text width=6em] (i1) {$8p$};
  \node[on chain=1, inode] (i2) {$p$};
  \node[on chain=1, inode] (i3) {$p$};
  % Data nodes
  \node[inode, below=2.1cm of i2] (d1) {$kp$};
  \node[inode, below=0.8cm of i3] (d2) {$kp$};
  \node[inode, below=0.8cm of d2] (d3) {$kp$};
  % File nodes
  \node[on chain=2, below=1cm of d3, inode, text width=2em, xshift=-3cm] (f1) {$b$};
  \node[on chain=2, inode, text width=5em] (f2) {$b$};
  \node[on chain=2, inode, text width=8em] (f3) {$b$};

  % Draw edges
  \coordinate (f1n) at (f1.north);
  \draw [->] (i1) |- node[pos=0.25, name=s1, rotate=70] {\tiny$/$} ($(f1n)+(0.0cm, 0.5cm)$) -- (f1n);
  \draw [->] (i2) -- node[pos=0.25, name=s2, rotate=70] {\tiny$/$} (d1);
  \draw [->] (i3) -- node[pos=0.5, name=s3, rotate=70] {\tiny$/$} (d2);
  \draw [->] (d2) -- node[pos=0.5,name=s4, rotate=70] {\tiny$/$} (d3);
  \coordinate (f2n) at (f2.north);
  \draw [->] (d1) |- node[pos=0.3, name=s5, rotate=70] {\tiny$/$} ($(f2n)+(0.0cm, 0.5cm)$) -- (f2n);
  \coordinate (f3n) at (f3.north);
  \draw [->] (d3) |- node[pos=0.7, name=s6, rotate=70] {\tiny$/$} ($(f3n)+(0.0cm, 0.5cm)$) -- (f3n);

  \node[left=0.05cm of s1, yshift=0.2cm] {8};
  \node[left=0.05cm of s2, yshift=0.2cm] {1};
  \node[left=0.05cm of s3, yshift=0.2cm] {1};
  \node[left=0.05cm of s4, yshift=0.2cm] {$k$};
  \node[left=0.05cm of s5, yshift=0.2cm] {$k$};
  \node[right=0.05cm of s6, yshift=0.1cm] {$k$};

  % Manually draw the strike thru
  %\coordinate (m1) at ($(i1)-(0, 1.5cm)$);
  %\draw [-] ($(m1)-(0.15,0.15)$) -- ($(m1)+(0.15,0.15)$);
  %\coordinate (m2) at ($(i2)!0.5!(d1)$);
  %\draw [-] ($(m2)-(0.15,0.15)$) -- ($(m2)+(0.15,0.15)$);


  %
  %\coordinate (MidWay) at ($(2.east)!0.5!(1.west)$);
  %\draw [thick, red,-] ($(MidWay)-(0.15,0.15)$) -- ($(MidWay)+(0.15,0.15)$);

\end{tikzpicture}

From the diagram, we can work out the required values:\\
(a) The total file size is $8b + kb + k^2b = (8+k+k^2)b$ \\
(b) The total file size plus the size of the pointers which is:
$[(8+k+k^2)b] + [(8p + p + p) + (kp) + (kp) + (k^2p)] = (8+k+k^2)b + (10+2k+k^2)p$










\begin{example}
In the \textit{Linux ext2fs filesystem}, an inode has 15 pointers.
The first 12 pointers directly locate 12 data blocks, 
the 13th is an indirect pointer, the 14th is a doubly-indirect pointer and the 
15th is triply-indirect pointer. 

Each of these pointers is 8 bytes long.

\end{example}




\section{Unix Block Buffer Cache}

We can cache files in main memory for a speedup (compared to accessing the files 
from disk). The cache works in the usual manner, we have a cache hit case and a 
cache miss case.



\section{Unix inodes}


\begin{lstlisting}
typedef struct{
unsigned inode_number : 16; /* 2 bytes */
char file_name[14] : 112; /* 14 bytes */
} DIRECTORY_ENTRY
\end{lstlisting}

An inode number uniquely identifies an inode.
There is precisely one inode per file.

\section{Virtual File Systems}


\section{Unix Permissions}


\highlightdef{Unix has access levels: 
\textbf{r} for \textit{read},
\textbf{w} for \textit{write},
\textbf{x} for \textit{execution}}



\section{File System Recovery}

storage devices still mess up – they have
so-called bad blocks that make it hard to keep a file
system reliable.

simply backup the system regularly so that
parts of it can be restored when a bad block occurs.
The problem is how to do backups efficiently:

incremental dumps, by which changes are added
to the backup, say, every day
use doubling technique, such as doing writes to
two drives, but reading only from one.




