
\chapter{Unix Drivers}


\section{Introducing Unix Drivers}

Unix puts io devices into three categories.


\begin{figure}[h]
\begin{tikzpicture} [
   every node/.style={node_sty, align=center},
   edge from parent/.style={edge_sty},
   edge from parent path={(\tikzparentnode.south) 
   |- ($(\tikzparentnode.south)!0.5!(\tikzchildnode.north)$) -| (\tikzchildnode.north)},
   level 1/.style={sibling distance=4cm, level distance = 0.9cm}
]
   \node 
   (root) {IO Device}
      child {node (pio) {Character}}
      child {node(intio) {Block}}
      child {node(intio) [text width=9em] {Network}};
   
\end{tikzpicture}
\end{figure}

We a familiar with io devices being hardware.
However unix allows an io device to be purely software. 


\begin{itemize}   
\renewcommand{\labelitemi}{$\Box$}
\item \textbf{Character} 
A \textit{character io device} is one that works with a sequential streams of bytes. 
The driver for character io devices are written to be able to work with streams.
They need to implement at least the \textit{open}, \textit{close}, \textit{read}, and 
\textit{write} unix system calls. 
\item \textbf{Block} 
A \textit{block io device} is the like a character io device. 
It works with streams of bytes. It also must implement at least \textit{open}, \textit{close}, \textit{read}, and \textit{write}. However, \textit{more operations} are provided to handle streams of bytes as a \textit{flow of blocks}. Block io devices are 
able to process blocks whose data arrives in a non-sequential order.
\item \textbf{Network} 
A network device is in charge of sending and receiving network bytes.
Instead of \textit{read} and \textit{write}, functions
related to packet transmission are provided.
\end{itemize}

Note that these three categories are not absolute. 
The distinction is often blurred. For example, is a tape a 
block device or a character device? Others devices don't fit 
into any of these characters.
For example clocks.


Block device transfers are buffered through a \textit{block buffer cache}.

Each device has a file in Unix. This is called the \textit{device file}.
Character and Block devices are located in the \lstinline{/dev} directory.
Network files however, are located 

When the user opens /dev/docdrv, the kernel calls a function in docdrv called the
\textit{driver open function}. Similarly, when the user closes /dev/docdrv, 
the kernel calls a function in docdrv called the \textit{driver release function}.



\section{Writing a Unix Driver}

Io modules can be loaded and removed using \textit{insmod} and \textit{rmmod}.

\begin{lstlisting}
doc > gcc -c hello.c
doc > insmod hello.o
Hello world
doc > rmmod hello.o
Goodbye
doc >
\end{lstlisting}