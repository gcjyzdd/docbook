
\chapter{Unix Drivers}


\section{Introducing Unix Drivers}


\begin{itemize}   
\renewcommand{\labelitemi}{$\Box$}
\item \textbf{Immediate Handler} (Top Half)
Since interrupt handler must be minimal, all other 
processing related to the event that caused the 
interrupt must be \textit{deferred}.
\item \textbf{Kernel Thread} (Bottom Half)
A thread performing the remainder of the work that 
was too heavy to do in the immediate handler. 
\end{itemize}

\begin{example}
Network interrupt causes packet to be copied from 
network card to kernel space (kernel interrupt handler)
Other processing on the packet should be deferred until 
its time comes (kernel deferrable function)
\end{example}

\frmrule

The bottom half can’t block. 
Interrupts are enabled during bottom half, 
there are not disabled like they are during top half.

If an interrupt occurs during the bottom half, 
a new top half is introduced. 
The bottom half preemptable by top half of a different driver.
This introduces the idea of \textit{interrupt nesting}.




Unix puts io devices into three categories.


\begin{figure}[h]
\begin{tikzpicture} [
   every node/.style={node_sty, align=center},
   edge from parent/.style={edge_sty},
   edge from parent path={(\tikzparentnode.south) 
   |- ($(\tikzparentnode.south)!0.5!(\tikzchildnode.north)$) -| (\tikzchildnode.north)},
   level 1/.style={sibling distance=4cm, level distance = 0.9cm}
]
   \node 
   (root) {IO Device}
      child {node (pio) {Character}}
      child {node(intio) {Block}}
      child {node(intio) [text width=9em] {Network}};
   
\end{tikzpicture}
\end{figure}

We a familiar with io devices being hardware.
However unix allows an io device to be purely software. 


\begin{itemize}   
\renewcommand{\labelitemi}{$\Box$}
\item \textbf{Character} 
A \textit{character io device} is one that works with a sequential streams of bytes. 
The driver for character io devices are written to be able to work with streams.
They need to implement at least the \textit{open}, \textit{close}, \textit{read}, and 
\textit{write} unix system calls. 
\item \textbf{Block} 
A \textit{block io device} is the like a character io device. 
It works with streams of bytes. It also must implement at least \textit{open}, \textit{close}, \textit{read}, and \textit{write}. However, \textit{more operations} are provided to handle streams of bytes as a \textit{flow of blocks}. Block io devices are 
able to process blocks whose data arrives in a non-sequential order.
\item \textbf{Network} 
A network device is in charge of sending and receiving network bytes.
Instead of \textit{read} and \textit{write}, functions
related to packet transmission are provided.
\end{itemize}

Note that these three categories are not absolute. 
The distinction is often blurred. For example, is a tape a 
block device or a character device? Others devices don't fit 
into any of these characters.
For example clocks.


Block device transfers are buffered through a \textit{block buffer cache}.

Each device has a \textsc{unix} special file. 
Recall from the first chapter the difference between normal file system files 
and \textsc{unix} special files.
This is called the \textit{device file}.
Character and Block devices are located in the \lstinline{/dev} directory.
Network devices also have \textsc{unix} special files, but these are located 

Suppose we create our own driver called \textsc{doc} Driver or \lstinline{docdrv}. 
This will be stored in \lstinline{/dev/docdrv}.
When the user opens \lstinline{/dev/docdrv}, 
the kernel calls a function in \lstinline{docdrv} called the
\textit{driver open function}. Similarly, when the user closes \lstinline{/dev/docdrv}, 
the kernel calls a function in \lstinline{docdrv} called the 
\textit{driver release function}. These are event functions that give us a chance 
to write code that executes when the driver is loaded and also write code 
for when it is stopped. 



\section{Writing a Unix Driver}

Io modules can be loaded and removed using \textit{insmod} and \textit{rmmod}.

\begin{lstlisting}
doc > gcc -c hello.c
doc > insmod hello.o
Hello world
doc > rmmod hello.o
Goodbye
doc >
\end{lstlisting}