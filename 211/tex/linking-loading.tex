
\chapter{Linking and Loading}


\section{Relocatable Files}


The compiler creates \textit{symbol references} and \textit{symbol definitions}.
The linker must form a one-to-many relationship between symbol definitions 
and symbol references. We are \textit{resolving} the symbol reference.

\highlightdef{\textbf{Symbol Resolution}:
the linker must associate each symbol reference with \textit{exactly one}
symbol definition. A definition may have many \textit{zero or more} references.}

A common mistake is to think that there is a one-to-one relationship between 
symbol definitions and symbol references. 

\frmrule

We introduce \textit{relocatable files} 
by introducing the internals of the \textsc{elf} \textit{format}.
\textsc{elf} is popular in \textsc{unix} operating systems and 
found in systems such as Solaris, \textsc{bsd} and Linux.

\highlightdef{The \textbf{ELF Format}:
specifies file format for linking and loading file types in \textsc{unix}}

The executable and linkable format or \textsc{elf} format is used for 
different file types in the field of linking and loading. 
It has one flexible format that is has been reused for \textit{relocatable 
object files}, \textit{executable object files} and \textit{shared object files}. 
The elf format is \textit{extremely flexible}. Many of the available sections can be shrunk, 
expanded and removed. There are no fixed sizes and no fixed offsets. 




\begin{figure}[h]
\begin{tikzpicture} [
   every node/.style={node_sty, align=center},
   edge from parent/.style={edge_sty},
   edge from parent path={(\tikzparentnode.south) 
   |- ($(\tikzparentnode.south)!0.5!(\tikzchildnode.north)$) -| (\tikzchildnode.north)},
   level 1/.style={sibling distance=2.9cm, level distance = 0.9cm},
   r/.style={text width=5.8cm},
   ch/.style={text width=2.4cm},
]
   \node 
   (root) [r] {\textsc{elf} File Format Specification}
      child {node[ch] {Relocatable Object Files (.o)}}
      child {node[ch] {Executable Object Files}}
      child {node[ch] {Shared Object Files (.so)}};
   
\end{tikzpicture}
\end{figure}

\begin{itemize}   
\renewcommand{\labelitemi}{$\Box$}
\item \textbf{Relocatable Object Files (.o)} 
Each .o file is produced from exactly one .c file. They contain program code and data. 
However the program code and data are augmented with data about the location of symbol 
definitions and the location of symbol references. 
This allows several relocatable object files to be easily compared and combined to form an 
executable object file (see below).
\item \textbf{Executable Object Files} 
An executable object file is the result of combining one or more relocatable object files. 
It contains program code and data but is augmented with 
system code; making it ready to be copied directly into memory and then be executed. 
Executable object files typically do not have a file extension. 
\item \textbf{Shared Object Files (.so)} 
A shared object file is a special type of relocatable object 
file that is used for \textit{dynamic linking}. We will see dynamic linking later. 
In Windows, the equivalent of a shared object file is a \textit{dynamic link library file} (.dll).  
\end{itemize}


In this section, we will look at relocatable object files. 

As previously mentioned, 
a \textit{relocatable object file} is a file that contain machine instructions but is
\textit{augmented symbol location information}. Relocatable object files are written
so the linker can perform symbol resolution with good knowledge of the \textit{location} 
of symbol definitions and symbol refereneces.


\highlightdef{A \textbf{Relocatable Object File}:
stores extra data to help the linker 
\textit{relocate} symbol references and definitions}

The compiler was the tool that first located these references and definitions, 
but the linker has to know where they are as well. To this effect, the linker 
has to \textit{relocate} where references and definitions are. When we 
call a file \textit{relocatable}, we are referring to the fact that the file 
has been augmented with symbol location information that so it's much easier 
to locate symbol definitions/references \textit{again} 
after the compiler did the first locating.


Below shows a slightly simplified version of what is inside 
a reloctable object file formatted in \textsc{elf}. 


\begin{itemize}   
\renewcommand{\labelitemi}{$\Box$}
\item \textbf{Program Code} The sections \lstinline{.init}, \lstinline{.text},
and \lstinline{.rodata} contain program code. They contain the binary of the 
machine code instructions for the program. They simply form a sequence 
of instructions in binary format with labels dotted around for instructions 
that jump or branch (these instructions are used for function calls, conditions and loops). 
\item \textbf{Program Data} 
The sections \lstinline{.data}, \lstinline{.bss} contain program data. 
The data section (\lstinline{.data} section) typically stores data that intended to be read-only 
whereas the \textit{block-started-by-symbol-store} section (\lstinline{.bss} section) 
is intended for data that is read from \textit{and written to}. 
Data in the data section has an initial value where as 
data in the bss store is commonly uninitialized.

\item \textbf{Symbol Table Section} 
The symbol table section contains the symbol table that the 
compiler generated. This is a list of all the symbol definitions for easy access. 
With the symbol table inside a relocatable object file, 
we can easily \textit{relocate} symbol definitions.
\item \textbf{Reference Information Section} 
The relocation information section is list of all the symbol references. 
Every time a variable is used, a note of this reference is put into this table. 
It stores where the reference occurs in program code. With the 
reference information table inside a relocatable object file, 
we can easily \textit{relocate} where the symbol references are that need to be resolved.
\end{itemize}

\frmrule

The \lstinline{readelf} tool helps us learn 

The column named \lstinline{Ndx} corresponds to which \textsc{elf} section
the entry belongs to. 3 for \lstinline{.data}, 1 is for \lstinline{.text}


\frmrule

\begin{figure}[h]
\begin{tikzpicture} [
   every node/.style={node_sty, align=center},
   edge from parent/.style={edge_sty},
   edge from parent path={(\tikzparentnode.south) 
   |- ($(\tikzparentnode.south)!0.5!(\tikzchildnode.north)$) -| (\tikzchildnode.north)},
   level 1/.style={sibling distance=2.5cm, level distance = 0.9cm},
   level 2/.style={sibling distance=2.2cm},
   r/.style={text width=4.5cm},
   ch/.style={text width=1.8cm},
   g/.style={text width=1.6cm}
]
   \node 
   (root) [r] {Symbol Definitions}
   	  child {node[ch] {Local Symbols}}
      child {node[ch] {Local Variables}}
      child {node[ch] {External Symbols}}
      child {
        node[ch] {Global Symbols}
        child {node[g] {Strong}}
        child {node[g] {Weak}}
      };
   
\end{tikzpicture}
\end{figure}


\begin{itemize}   
\renewcommand{\labelitemi}{$\Box$}
\item \textbf{Local Variables}  Local variables only have scope \textit{within a particular function}. 
They are not visible outside the current function and so no other part of the program can 
possibly have a symbol reference to this symbol. 
\item \textbf{Local Symbols}  Local symbols have scope \textit{within a particular file}. 
They are not visible outside the file. The linker may not map symbol references in 
one file to local symbol definitions that belong to a different file. 
\item \textbf{External Symbols} These symbols have scope \textit{within a particular module}. 
They are not visible outside the module. The linker may not map symbol references in 
one module to external symbol definitions that belong to a different module. 
In \textsc{c}, the \lstinline{extern} keyword is used to refer to external symbols. 
\item \textbf{Global Symbols} These symbols have a \textit{truely global scope}.
Global symbols defined in a given file of given module 
can be referenced by any file of any module. 
Global symbols can be either \textit{strong} or \textit{weak}. 
Functions and initialized global variables are strong. 
Overridable functions and uninitialized variables are weak. 
\end{itemize}

\frmrule

\begin{example}

\begin{lstlisting}
int age = 18;
int main() {
    printf(age);
}
\end{lstlisting}

\textbf{(a)} The symbol \lstinline{age} is a ..................
(global symbol/external symbol/local symbol/local variable). \\
\textbf{(b)} The symbol \lstinline{printf} is a ..................
(global symbol/external symbol/local symbol/local variable).

\end{example}



\frmrule

\begin{example}

\begin{lstlisting}
static void f() { 
    int a, b, c; 
}
\end{lstlisting}

\textbf{(a)} The symbol \lstinline{f} is a ..................
(global symbol/external symbol/local symbol/local variable). \\
\textbf{(b)} The symbol \lstinline{a} is a ..................
(global symbol/external symbol/local symbol/local variable).

\end{example}

\frmrule

\begin{example}

First file:
\begin{lstlisting}
int x = 7;
int main() {
    int t = f();
    exit(0);
}
\end{lstlisting}

second file:
\begin{lstlisting}
extern int x;
int f() {
   return x + 2;
}
\end{lstlisting}


\end{example}

\frmrule


\highlightdef{Local symbols are \textit{not} the same as local variables.}

Multiple strong symbols with the same name are not allowed. 
Suppose a strong symbol and a weak symbol share the same name, 
in this case, the strong symbol is given preference over the weak one.
If there are multiple weak symbols with the same name, 
\textit{any} could be chosen to map to the given reference.

\frmrule

During linking, the linker goes through each inputted relocatable file and 
determines if unknown symbols are defined in other relocatable files.

The linker is supplied with a list of relocatable object files. 
Linker scans these object files \textit{sequentially left to right} 
to resolve unknown symbols. It examines a file \textit{exactly one}. 

For each input file on command line, linker
updates a list of defined symbols with object’s defined symbols
tries to resolve the undefined symbols (from object and from list of 
previously undefined symbols) with the list of previously defined 
symbolscarries over the list of defined and undefined symbols to next input 
object file so linker looks for undefined symbols only after they’re 
undefined!it doesn’t go back over the entire set of input files to resolve the 
unknown symbolif an unknown symbol becomes referenced after it was defined, then 
linker won’t be able to resolve the symbol!

Thus, order on the command line is important - put libraries last!
% http://stackoverflow.com/questions/45135/linker-order-gcc



\section{Loading Executable Object Files}



We may even choose to insert redundant sections. 
We can take advantage of this to increase security. 
If we deliberately make abnormal, yet valid elf headers and files, 
it makes reverse engineering executable files difficult. vmware 
does this, for example. Similarly, when programs are loaded, a process called 
\textit{ASLR (Address Space Layout Randomization)} may be used. 
The exact address of where each section starts is randomized. 
This makes it harder for hackers to compromise the security of the program’s code. 


Although the format was made for segmented memory, 
we actually do not require segmentation to be used. 


\section{Dynamic Loading}

\begin{itemize}   
\renewcommand{\labelitemi}{$\Box$}
\item \textbf{Static Loading} The linker creates an executable file with 
\textit{all symbols} resolved. Results in \textit{larger} executable files. 
\item \textbf{Dynamic Loading} The linker creates an executable file, but some
symbol links \textit{are missing} and must be resolved at loadtime. 
For example, a dynamic linker might postpone resolving the definition of printf until loadtime.
This, in general, gives \textit{smaller} executable files. 
\end{itemize}

\section{Booting}

%http://www.moses.uklinux.net/patches/lki-1.html#ss1.8

