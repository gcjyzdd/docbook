
\chapter{Space Complexity}



\section{Introducing Space Complexity}

Clearly space is costly as well as time.
But have seen that

\highlightdef{Bound on times implies bound on space}

Putting a bound on the number of steps taken 
by a Turing machine also puts a bound on the 
number of tape squares we can write symbols to. 

So why bother measuring space?

\begin{itemize}   
\renewcommand{\labelitemi}{$\Box$}
\item \textbf{New Classification} 
Space complexity classes have nice closure properties
\item \textbf{Parallel computation} 
Space complexity classes have connection with parallel computation.
sequential logspace $\sim$ parallel logtime
\end{itemize} 

\frmrule



A multi-tape Turing machine has several heads that 
can move independently. This should not be confused with 
a multi-\textit{track} Turing machine that only has one 
head. 

\begin{itemize}   
\renewcommand{\labelitemi}{$\Box$}
\item \textbf{Work Tapes} 
There may be \textit{zero or more} work tapes.

\item \textbf{Input Tape} 
An IO-TM has exactly one \textit{input tape}. 
The head for this tape can move freely, but we 
cannot modify the symbols on the squares of this tape. 
This tape is \textit{read-only}. 
\item \textbf{Output Tape} 
An IO-TM has exactly one \textit{output tape}. 
The head for this tape can \textit{only move right}, and we 
will not use the tape for reading.
The squares of this tape are strictly write only.  
The tape is \textit{write-only}. 
\end{itemize} 


\highlightdef{\textbf{IO k-Tape TM}: $k-2$ work tapes, 1 input tape, 1 output tape}



An IO-TM decides a language $L$ in space $f(n)$ 
iff every input, $w$, of length $|w| = n$, 
the TM uses $f(n)$ squares on each work tape.
The contents of the output tape IO-TM 

Now using this definition, we define $\text{SPACE}(f(x))$.
$L$ is in SPACE($f(n)$) if $L$ is decided by an IOTM operating
within space $f(n)$. 

The previous definition is the same as 
the definitions of $\text{TIME}(f(x))$ except that now we are 
using a IO-TM instead of the TM definition we gave in the 
computational models chapter. 

One question is whether it is reasonble to include the 
input space and the output space as part of the space needed 
to do the work. After all, surely all the work is done on the 
work tape? It doesn't matter whether or not we include 
the input tape space and the output tape space. 
The difference changes by a factor of $k-2$, 
and as with time, we can ignore constant factors. 
The speedup is linear which makes no difference to it 
being a polynomial. 


\highlightdef{\textbf{LOGSPACE}: $\text{LOGSPACE} = \text{SPACE}(\log(n))$ }

We abbreviate \textsc{logspace} by L.

\frmrule

\begin{example}

Languages are often decided by maintaining counters with values within
poly bound in input length n.
Suppose we have $m$ counters whose digits have a 
p-bound $c_i \leqslant p(n)$.
Then the overall space usage $\sum c_i \leqslant m\log(p(n))$. 
Hence the language is $O(\text{log}(n))$ and so it in \textsc{logspace}. 
The problem associated to this language has a 
logarithmic space complexity.

\end{example}




\section{Time-Space Relationships}


We want to compare space and time usage:
For instance, we want to argue that a bound on time usage implies a
bound on space usage. 

\highlightdef{\textbf{Proper Function}: can be computed 
by an IO-TM within time and space\\ given by the value of the function}
Also called a \textit{proper complexity function}. 

$f:\mathbb{N} \rightarrow \mathbb{N}$ is a \textit{proper function} iff

\begin{itemize}   
\renewcommand{\labelitemi}{$\Box$}
\item \textbf{Nondecreasing} 
$f$ is a nondecreasing function. That is, $f(n+1) \geqslant f(n)$ for all n
\item \textbf{IO-TM} 
There is a $k$-tape IO-TM which for every input, $w$, 
outputs $1^{f(|w|)}$s on the output tape, decides operates within time $O(|w| + f(|w|))$ 
and space $O(f(|w|))$. 
\end{itemize} 


\frmrule

\begin{example}
Show that the addition of two proper functions, $f + g$, is a proper function.

\textbf{(a)} $f$ is nondecreasing so ................... for all n \\
\textbf{(b)} $g$ is nondecreasing so ................... for all n \\
\textbf{(c)} Hence $(f + g)(n+1) = f(n+1) + g(n+1) \geqslant$ .......... $+$ ......... $ = $ ....... for all n. \\
So $f + g$ is nondecreasing. 

Next we need to find a $k$ tape IO-TM which for 
every input, $w$, of length $|w| = n$, outputs \textbf{(d)} ............ 
on the output tape and operates within time \textbf{(e)} ............. 
and space \textbf{(f)} ............. 


\end{example}



\frameans{Complete the proof}
{
\textbf{(d)} $1^{(f+g)(\abs{w})}$. \\
\textbf{(e)} $O(\abs{w} + (f+g)(\abs{w}))$. \\
\textbf{(f)} $O((f+g)(\abs{w}))$ 
}

We already have IO-TMs that compute 

\begin{tikzpicture}
[n/.style={draw, fill=white,inner sep=11pt}]
    \node[n] (m) {$M_f$};
    \node[n, right=1cm of m] (m2) {$M_g$};
    \begin{pgfonlayer}{background} 
        \node[draw, fill=black!10, fit={(m2) (m)}, drop shadow, inner sep=5pt] (m1) {};
    \end{pgfonlayer} 
    \coordinate (in) at ($(m)+(-1.5cm,0.0cm)$);
    \coordinate (out) at ($(m2)+(2.5cm,0.0cm)$);
    \draw[->] (in) -- node[above]{} (m);
    \draw[->] (m) -- node[above]{} (m2);
    \draw[->] (m2) -- node[above]{} (out);

    \node[above of=m1] {$M_{f+g}$};
\end{tikzpicture}

Idea: \\
$M_{f+g}$ is a IO-TM with $(k_f - 2) + (k_g - 2)$ work tapes.

$M_{f+g}$ first runs $M_f$ on input $w$. 
The result is $1^{f(\abs{w})}$ on $M_f$'s output tape.
$M_{f+g}$ first runs $M_g$ on input $w$. 
The result is $1^{g(\abs{w})}$ on $M_g$'s output tape.
$M_{f+g}$ concatenates $M_f$ and $M_g$ uses copying
to copy from $M_f$'s output tape then from $M_g$'s output tape
to produce a concatenation of $1^{f(\abs{w} + g(\abs{w})}$ 
on $M_{f+g}$'s output tape.

This runs in total time $= [O(\abs{w}+f(\abs{w}))] + [O(\abs{w}+g(\abs{w}))]  
+ [f(\abs{w}) + g(\abs{w})]$ which is $O(\abs{w}+f(\abs{w})+g(\abs{w}))$. 
This uses space $= \max{f(\abs{w}),g(\abs{w})} = O(f(\abs{w}) + g(\abs{w}))$. 
Hence $f + g$, is a proper function.

\frmrule



\highlightdef{$\text{TIME}(f(n)) \subseteq \text{SPACE}(f(n))$ }
\highlightdef{$\text{NTIME}(f(n)) \subseteq \text{SPACE}(f(n))$ }

\frmrule

\highlightdef{$\text{NSPACE}(f(n)) \subseteq \text{TIME}(k^{\log(n)+f(n)})$ }



We can finally say that if $f$ is proper then,
\highlightdef{
$
\text{TIME}(f(n)) \subseteq 
\text{SPACE}(f(n)) \subseteq 
\text{NSPACE}(f(n)) \subseteq 
\text{TIME}(k^{\log(n)+f(n)})
$
}
