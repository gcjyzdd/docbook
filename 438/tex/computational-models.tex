
\chapter{Models of Computation}



\section{Turing Machines I}

Formally a TM is $M_k = (Q,\Sigma,q_0,\delta)$ 
where 

\begin{itemize}   
\renewcommand{\labelitemi}{$\Box$}
\item \textbf{State Set} Formally, set $Q$ is the set of \textit{states} of a TM. 
$Q$ will always include three special states, $h$, $\textsc{yes}$.
and $\textsc{no}$. Intuitively, these are the nodes of our TM transition diagram.
%
\item \textbf{Symbol Set} Formally, the set $\Sigma$ is 
the set of \textit{symbols} of a TM. 
$\Sigma$ will always include two special symbols, $>$ and $\sqcup$.
Intuitively, the symbol set are the symbols that can appear on the squares of the TM.
%
\item \textbf{Start State} Formally, the start state is one state chosen 
from the state set. 
Intuitively, this is the start state on the TM. 
%
\item \textbf{Transition Function} The transition function is a partial function 
that performs the map $(q,s) \mapsto (q',s',\{L,R,-\})$.
Intuitively, this function determines the arrows on our TM transition diagram.
\end{itemize} 

\frmrule


A \textit{Turing machine configuration} has the form $c = (q,u,v)$. 

\begin{itemize}   
\renewcommand{\labelitemi}{$\Box$}
\item \textbf{Left+Head Squares} $u$ are the symbols that are to the left of, 
\textit{and including}, the symbol under the head.
%
\item \textbf{Right Squares} $v$ are the symbols that are to the right of, 
\textit{not including}, the symbol under the head.
\end{itemize}

Notice that $u$ cannot be empty. The leftmost square the head can be on is 
the first square which always non-empty and usually 
contains the start symbol. In this case, 
the configuration is $(q,>,\epsilon)$.

\frmrule

\begin{itemize}   
\renewcommand{\labelitemi}{$\Box$}
\item \textbf{1. Read Symbol} 
\item \textbf{2. Check Instruction Table} 
\item \textbf{3. Write Symbol} 
\item \textbf{4. Update State} 
\item \textbf{5. Move Head} 
\end{itemize}

\frmrule

\begin{figure}[h]
\begin{tikzpicture}[ 
      node distance=0mm,
      txt/.style={font=\small}
  ]
  %
  % RULE
  %
  \node [fill=black!10] (rule) { \parbox{5.5cm}
  {
    \begin{prooftree}
      \def\defaultHypSeparation{\hskip .01in}
      \AxiomC{$u=u's$}\AxiomC{$(q,s) \mapsto (q',t,L)$}
      \LeftLabel{\textsc{l}}\RightLabel{$q \in Q$}
      \BinaryInfC{$(q,u,v) \rightarrow_M (q',u',tv)$}
    \end{prooftree}
  }};
  %
  \node[state] (q1) [below left= 0.5cm and -2cm of rule] {$q$};
  \node[state] (q2) [right = 1cm of q1] {$q'$};
  \path[->,>=stealth',shorten >=1pt] (q1) edge[above] node {$s,t,L$} (q2);
  %
  % TM1
  %
  \begin{scope}[start chain=1,xshift=4cm,
      every node/.style={minimum height=0.65cm,  
        minimum width=0.65cm, inner sep=4pt}]
  {
    \foreach \xx in {0,1,2,3,4,5,6} 
    {
      \ifthenelse{3 = \xx} 
        {\node[on chain=1,fill=black!10] (\xx) {$s$};} 
        {\node[on chain=1] (\xx) {};}
      \ifthenelse{\NOT 0 = \xx}
        {\draw[dashed, draw=black!50] (\xx.north west) -- (\xx.south west);}
        {}
    }
  }
  \end{scope}
  \begin{pgfonlayer}{background} 
      \node [draw=black, fill=white, drop shadow,
      inner sep=1pt, outer sep=5pt, fit={(0) (6)}] {};
  \end{pgfonlayer} 
  \draw[->,>=stealth',shorten >=1pt,thick] ($(3)+(0.0cm,1.0cm)$) -- (3);
  \draw[decorate,decoration={brace,mirror}]
    ($(0.south west)+(0.0cm,-0.2cm)$) 
    to node[midway,below,yshift=-0.1cm] {$u$}
    ($(3.south east)+(-0.0cm,-0.2cm)$);
  \draw[decorate,decoration={brace,mirror}]
    ($(4.south west)+(0.0cm,-0.2cm)$) 
    to node[midway,below,yshift=-0.1cm] {$v$}
    ($(6.south east)+(-0.0cm,-0.2cm)$);
  \draw[decorate,decoration={brace}]
    ($(0.north west)+(0.0cm,0.2cm)$) 
    to node[midway,above,yshift=0.1cm] {$u'$}
    ($(2.north east)+(-0.0cm,0.2cm)$);
  %
  % TM2
  %
  \begin{scope}[start chain=2,xshift=4cm, yshift=-2cm,
      every node/.style={minimum height=0.65cm,  
        minimum width=0.65cm, inner sep=4pt}]
  {
    \foreach \xx in {0,1,2,3,4,5,6} 
    {
      \ifthenelse{2 = \xx} 
      {
        \node[on chain=2,fill=black!10] (t2\xx) {};
      } 
      {
        \ifthenelse{3 = \xx} 
        {
          \node[on chain=2] (t2\xx) {$t$};
        } 
        {
          \node[on chain=2] (t2\xx) {};
        }
      }
      \ifthenelse{\NOT 0 = \xx} 
        {\draw[dashed,draw=black!50] (t2\xx.north west) -- (t2\xx.south west);}
        {}
    }
  }
  \end{scope}
  \begin{pgfonlayer}{background} 
      \node [draw=black, drop shadow, fill=white,
      inner sep=1pt, outer sep=5pt, fit={(t20) (t26)}] {};
  \end{pgfonlayer} 
  \draw[->,>=stealth',shorten >=1pt,thick] ($(t22)+(0.0cm,1.0cm)$) -- (t22);
  \draw[decorate,decoration={brace,mirror}]
    ($(t20.south west)+(0.0cm,-0.2cm)$) 
    to node[midway,below,yshift=-0.1cm] {$u'$}
    ($(t22.south east)+(-0.0cm,-0.2cm)$);
  \draw[decorate,decoration={brace,mirror}]
    ($(t23.south west)+(0.0cm,-0.2cm)$) 
    to node[midway,below,yshift=-0.1cm] {$tv$}
    ($(t26.south east)+(-0.0cm,-0.2cm)$);

\end{tikzpicture}
\end{figure} 





\begin{prooftree}
\def\defaultHypSeparation{\hskip .01in}
\AxiomC{$u=u's$}
\AxiomC{$(q,s) \mapsto (q',t,-)$}
\LeftLabel{\textsc{S}}
\RightLabel{$q \in Q$}
\BinaryInfC{$(q,u,v) \rightarrow_M (q',u't,v)$}
\end{prooftree}

\begin{prooftree}
\def\defaultHypSeparation{\hskip .01in}
\AxiomC{$u=u's$}
\AxiomC{$v=mv'$}
\AxiomC{$(q,s) \mapsto (q',t,R)$}
\LeftLabel{\textsc{R}}
\RightLabel{$q \in Q$}
\TrinaryInfC{$(q,u,v) \rightarrow_M (q',u'tm,v')$}
\end{prooftree}

\begin{prooftree}
\def\defaultHypSeparation{\hskip .01in}
\AxiomC{$u=u's$}
\AxiomC{$v=\epsilon$}
\AxiomC{$(q,s) \mapsto (q',t,R)$}
\LeftLabel{\textsc{R+}}
\RightLabel{$q \in Q$}
\TrinaryInfC{$(q,u,v) \rightarrow_M (q',u't\sqcup,\epsilon)$}
\end{prooftree}





\frmrule


\begin{itemize}   
\renewcommand{\labelitemi}{$\Box$}
\item \textbf{Initial Config} $c_0 = (q_0,>,\epsilon)$
%
\item \textbf{Computation} 
A \textit{computation} of a TM, $M$ is a \textit{sequence of
configurations}, $c_0, c_1, c_2,...$ (finite or infinite) 
where $c_0$ is the initial configuration and 
$c_{i} \rightarrow_M c_{i+1}$ for $i=0,1,...$.
\end{itemize}


\frmrule


\begin{itemize}   
\renewcommand{\labelitemi}{$\Box$}
\item \textbf{Input word} A word $w$ is the \textit{input} 
is the initial data that we have
provided for $M$ for computation. 
\item \textbf{Output word} A word $w$ is the \textit{input} 
is the output data $M$ has provided from its computation 
based on the input given. This \textit{assumes $M$ halts}.
To halt, the state must be any of $\textsc{yes}$, $\textsc{no}$
or $h$. 
\end{itemize} 



\section{Turing Machines II}

\begin{example}
The empty function. 

We can use the halting states $\textsc{yes}$ and
$\textsc{no}$ to output.
\end{example}

\frmrule

\begin{example}
The shift right function. 

\begin{tikzpicture}
[->,>=stealth',shorten >=1pt,auto,node distance=2.8cm,
                    semithick]
  \node[initial,state] (A)                    {$q_a$};
  \node[state]         (B) [above right of=A] {$q_b$};
  \node[state]         (D) [below right of=A] {$q_d$};
  \node[state]         (C) [below right of=B] {$q_c$};

  \path (A) edge              node {0,1,L} (B)
            edge              node {1,1,R} (C)
        (B) edge [loop above] node {1,1,L} (B)
            edge              node {0,1,L} (C)
        (C) edge              node {0,1,L} (D)
        (D) edge [loop below] node {1,1,R} (D)
            edge              node {0,1,R} (A);
\end{tikzpicture}


\end{example}

\frmrule

\begin{example}
The head function. 


\end{example}

\frmrule

\begin{example}
The tail function. 
\end{example}

\begin{example}
Append $<$ function. 
\end{example}

\frmrule

\begin{example}
The shift left function. 
\end{example}




\section{Turing Machines III}

\section{Turing Machines IV}

\section{Multitape Turing Machines}

\highlightdef{\textbf{Multitape TM}: several heads that move independently}

Formally a $k$-tape TM is $M_k = (Q,\Sigma,q_0,\delta)$ 
where 

Don't confuse a multitrack TM with a multitape TM. 

\begin{tikzpicture}[->,>=stealth',shorten >=1pt,auto,node distance=2.8cm,
                    semithick]
  %\tikzstyle{every state}=[draw=none,text=white]

  \node[initial,state] (A)                    {$q_a$};
  \node[state]         (B) [above right of=A] {$q_b$};
  \node[state]         (D) [below right of=A] {$q_d$};
  \node[state]         (C) [below right of=B] {$q_c$};
  \node[state]         (E) [below of=D]       {$q_e$};

  \path (A) edge              node {0,1,L} (B)
            edge              node {1,1,R} (C)
        (B) edge [loop above] node {1,1,L} (B)
            edge              node {0,1,L} (C)
        (C) edge              node {0,1,L} (D)
            edge [bend left]  node {1,0,R} (E)
        (D) edge [loop below] node {1,1,R} (D)
            edge              node {0,1,R} (A)
        (E) edge [bend left]  node {1,0,R} (A);
\end{tikzpicture}


\section{Random Access Machines}

These are often called \textsc{ram}s. 
This is not to be confused with random access \textit{memory}.
We are talking about random access \textit{machines}