
\chapter{Nondeterminsm}



\section{Introducing Nondeterminsm}





$P\overset{?}{=}NP$




\highlightdef{
A \textbf{Nondeterministic Turing Machine} $N$ can: 
\begin{itemize}   
\renewcommand{\labelitemi}{$\Box$}
\item \textbf{Accept an Input}:
$N$ accepts an input $x$, ($x$ is accepted by $N$) iff there is some sequence of
nondeterministic choices that results in state \textsc{yes}.
\item \textbf{Reject an Input}:
$N$ rejects an input $x$, ($x$ is rejected by $N$) iff for all 
sequence of nondeterministic choices, none lead to the state \textsc{yes}.
\item \textbf{Accept a Language}:
$N$ accepts a langauge $L$, ($L$ is accepted by $N$) iff for every $x \in L$, 
and for only every $x \in L$, 
there is some sequence of nondeterministic choices 
that results in state \textsc{yes}. That is, 
$x \in L$ iff $(s,\triangleright \cdot x) \rightarrow (\textsc{yes}, w\cdot u)$ 
\item \textbf{Reject a Language}:
$N$ rejects a langauge $L$, ($L$ is rejected by $N$) iff there is some $x \in L$, 
such that for all sequence of nondeterministic choices,
none lead to the state \textsc{yes}.
\item \textbf{Operate in Time $f(n)$ for a Langauge}:
$N$ operates in time $f(n)$ iff for all inputs $x \in L$, $N$ always results in 
either state \textsc{yes} or state \textsc{no} within $f(|x|)$ nondeterministic choices.
\item \textbf{Decide a Langauge in Time $f(n)$}:
$N$ operates in time $f(n)$ iff for all $x \in L$, and for only every $x \in L$, 
if $N$ reaches state \textsc{yes} or state \textsc{no} 
then it is always within $f(|x|)$ nondeterministic choices.
\end{itemize}
}

\highlightdef{$N$ decides $L$ implies all computation paths have length at most $f(|x|$)}


\frmrule


\highlightdef{\textbf{Random Generator Machine} $N_G$ generates an arbitrary binary sequence}


\frmrule


\highlightdef{$\text{NP} = \cup_{k>0} \text{NTIME}(n^k)$}







\frmrule

\begin{example}
Show that \textsc{tsp(d)} is in NP. \\

Consider instances of the language \textsc{tsp(d)} consisting of pairs $(M, k)$ such that M
is an $n \times n$ distance matrix and $B$ is an integer and there is a tour of 
$n$ cities whose length is at most $k$. 

The following nondeterministic Turing machine $N$ decides \textsc{tsp(d)}:
\begin{itemize}   
\renewcommand{\labelitemi}{$\Box$}
\item $N$ generates after its input $(M, k)$ an arbitrary sequence of 0,1,/ 
no longer than a polynomial in $n$ using the idea from $N_G$.
\item After this, $N$ is completely deterministic. 
\item First that the random generated sequence is well-formed path (of 
the form $/c_1/c_2/\cdots/c_n$) and 
that the original input of $(M, k)$ is still well-formed (wasn't modified).
This amounts to examining the entire tape. If these checks fail, 
$N$ halts in the rejecting state. 
\item Otherwise $N$ continues to determine whether the generated sequence
$/c_1/c_2/\cdots/c_n$ presents a tour of the cities of cost at most $k$.
If this is the case, N halts in the accepting state otherwise in the rejecting state.
\end{itemize}

We will justify that $N$ does indeed decide \textsc{tsp(d)} in polynomial time and hence 
justify that \textsc{tsp(d)} is indeed in NP. 

\end{example}


\frmrule

\begin{example}
Does it matter whether we define NP as $\text{NP} = \cup_{k>0}$ or $\text{NP} = \cup_{k \geqslant 0}$? Explain.
Does it matter whether $k$ takes real values or natural number values? 
\end{example}

\frmrule



\section{Succinct Certificates}



Let $L$ be any language. 
\highlightdef{ \textbf{Succinct Certificate Theorem}: $L$ is in NP iff there is some 
\textit{succinct certificate relation} satisfying 
$L = \{x \in \Sigma^{*} | (x,y) \in R \text{ for some y} \in \Sigma^{*}\}$  }

Let's prove this.
\begin{itemize}   
\renewcommand{\labelitemi}{$\Box$}
\item ($\Leftarrow$) Assume there is some succinct certificate relation satisfying
$L = \{x \in \Sigma^{*} | (x,y) \in R \text{ for some y} \in \Sigma^{*}\}$.\\
By definition of a succinct certification relation, 
there is some deterministic turing machine $M$ that 
can decide the language $\{x;y | (x,y) \in R\}$

We form a non-deterministic turing machine $N$ that on input $x$, 
guesses a $y$ of length at most $|x|^k$ and uses the machine
$M$ to check whether $x;y$ is in $R$. If it is, then $N$ 
goes to the \textsc{yes} state. 
By definition of non-deterministic turing machine 
deciding a language in polynomial time, $L$ 
is decided by $N$ in a polynomial time. Hence $L$ is in NP.

\item ($\Rightarrow$) Assume $L \in \text{NP}$. \\
By definition of NP, there is a non-deterministic turing machine 
$N$ that decides $L$ in time polynomially bounded by $x$. 

Define a relation $R$ as follows: $(x, y) \in R$ iff $y$ is the 
encoding of an accepting computation of $N$ on input $x$. 
This is a \textit{succinct certificate relation} because:
\begin{itemize} 
\item it is balanced (each computation of $N$ is polynomially bounded so $|y| \leqslant |x|^k$ 
for some $k$), 
\item a deterministic turing machine can decide the language $\{x;y | (x,y) \in R\}$
in polynomial time (this is simple check of whether $y$ 
encodes an accepting computation of $N$ on $x$). We check that 
the initial state is correct, each step on the computation makes is possible 
according to $N$'s transition relation and that the final state is \textsc{yes}. 
\end{itemize}
Hence $L$ has some succinct certificate relation satisfying
$L = \{x \in \Sigma^{*} | (x,y) \in R \text{ for some y} \in \Sigma^{*}\}$.
\end{itemize}


We can think of languages in NP as those languages  nondeterministic turing 