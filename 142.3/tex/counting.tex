
\chapter{Counting}




\section{Counting Arguments}

Next, we use overcounting to count the number of edges in a binary tree. 

\begin{example}
We notice that each node corresponds to two edges. 
So we write: $\text{edges} = 2 \times \text{nodes} \dotsm$ but then 
we notice that the leave nodes have no corresponding edges. 
We have \textit{overcounted}. We fix our count by subtracting 
$2 \times \text{leaves}$. 
Hence the final equation is $\text{edges} = 2 \times \text{nodes} - 2 \times \text{leaves}$ 
or  $\text{edges} = 2 \times (\text{nodes} - \text{leaves})$ 
\end{example}

For a tree of depth $d$, 
we know $\text{leaves} = 2^{d}$ 
and $\text{nodes} = 2^{d+1}-1$

If we substitute these into the above edge relationship, we get 
% TODO: put in array
$
\text{edges} = 2 \times (\text{nodes} - \text{leaves})
= 2 \times (2^{d+1}-1 - 2^{d})
= 2^{d+2} - 2 - 2^{d+1}
= 2^{d+1} - 2
$ 

Hence a simple counting argument gives us the same result as the one 
we worked out using the geometry sum formula. 

\frmrule


\begin{example}
We will use another counting argument to show that the number of 
edges is one less than the number of nodes. 
\end{example}




\section{Repeated Expansions}




By doing a repeated exansion, $\text{leaves}_d = 2(2(2[ \dotsm 2(1) \dotsm ]))$.
Hence we have $\text{leaves}_d = 2^{d}$

By doing a repeated exansion, $\text{nodes}_i = 
2(\text{nodes}_{i-1} - \text{nodes}_{i-2} - \text{nodes}_{i-2} ... - \text{nodes}_0)$.

Hence we have $\text{leaves}_d = $


We can show that the \textit{sum of the elements in a given row} of pascals triangle is a \textit{power of two}
using \textit{repeated expansion}. \textit{Pascal's identity} allows us to
notice that $\sum^{n}_{r=0} C(n,r) = 2 \sum^{n-1}_{r=0} C(n-1, r)$

% TODO: put in array
We get $\sum^{N}_{r=0} C(N,r) 
= 2 (\sum^{n-1}_{r=0} C(N-1, r))
= 2 (2(\sum^{n-2}_{r=0} C(N-2, r)))
= 2 (2(2(\sum^{n-3}_{r=0} C(N-3, r))))
= \dotsm
= 2 (2(2[ \dotsm 2(\sum^{1}_{r=0} C(1, r))] ))
= 2^{N}
$


We can also use repeated expansion to solve \textit{recurrence relations}.

We can also use repeated expansion to solve \textit{series involving subtraction}.
This technique is known as \textit{difference of two sums}. 
The series are often called \textit{telescoping series}. 