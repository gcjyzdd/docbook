
\chapter{$\mathcal{L}_1$ Language}


\section{Introducing $\mathcal{L}_1$}

Here we define a new language $\mathcal{L}_1$ to begin our study of 
object-oriented language. 
$\mathcal{L}_1$ has support for classes that contain fields and methods. 
There is no support for inheritence, abstract classes, overriding
or interfaces. 

The work in this chapter will be on giving a formal description of $\mathcal{L}_1$ 
consists of
\begin{itemize}   
\renewcommand{\labelitemi}{$\Box$}
\item \textbf{Syntax} 
\item \textbf{Operational Semantics} 
\item \textbf{Type System} We will see how to prove \textit{type soundness}, 
demonstrated through an important theorem called the \textit{subject reduction theorem}.
\end{itemize} 

The \textit{structure} of an $\mathcal{L}_1$ program is defined by the function
\highlightdef{ $Prog = ClassId \rightarrow 
((FieldId \rightarrow type) 
\times (MethodId \rightarrow meth))$ }

where $meth$ and $type$ have a syntax defined by:
\begin{flalign*}
meth &::= type\, m (type\, x) \{ x \} &\\
type &::= \text{bool} | m \\
e    &::= \text{if}\, e \text{then}\, e\, \text{else}\, e | e.f | e.f := e 
               | e.m(e) | \text{new}\, c | x | \text{this} | \text{true} 
               | \text{false} | \text{null}
\end{flalign*}
with the identifier conventions: $c \in ClassId$  for class identifiers,
$f \in FieldId$ for field identifiers and $m \in MethId$  for class 
identifiers.

\frmrule


\begin{example}
Here is an example program
\begin{lstlisting}
class Book {
	bool good;
	Person owner;
	bool readBy(Person x) {this.good := true}
}
class Person {
	Book like;
	Book meet (Person x) {this.like := x.like}
}
\end{lstlisting}

which has the following representation:


\[ \begin{array}{lllll}
P_{PB} & = & \text{Book}   & \mapsto & ((\text{good} \mapsto \text{bool}, 
                                   \text{owner} \mapsto \text{Person}), \\
       &   &               &         &  (\text{readBy} \mapsto \text{bool readBy(Person x)}
                                    \{ \text{this.good} := \text{true} \})), \\
       &   & \text{Person} & \mapsto & ((\text{like} \mapsto \text{Book}), \\
       &   &               &         &  (\text{meet} \mapsto \text{Book meet(Person x)}
                                    \{ \text{this.like} := \text{x.like} \}))	 \\
\end{array}\] 


\end{example}



\begin{example}
For the following program, find the representation in $\mathcal{L}_1$.

\begin{lstlisting}
class B {
	D m (E x) { true }
	C f
}
\end{lstlisting}
\end{example}

\frmrule

\[ \begin{array}{lllll}
P_B & = & \text{B}   & \mapsto & ((\text{f} \mapsto \text{C}), \\ 
    &   &            &         &  (\text{m} \mapsto \text{D m (E x)}
                                    \{ \text{true} \})) \\
\end{array}\] 

\frmrule

For program $P$, and identifiers $c$, $f$, $m$, we define 
a \textit{field lookup function} and a \textit{method lookup functions}.

\highlightdef{The \textbf{Field Lookup Function} 
$\mathcal{F}(\text{P},\text{c},\text{f}) 
= \text{P}(\text{c}) \downarrow_{1} (\text{f})$ \\
The \textbf{Method Lookup Function} 
$\mathcal{M}(\text{P},\text{c},\text{m}) 
= \text{P}(\text{c}) \downarrow_{2} (\text{m})$}

We also define the \textit{field set function} as 
$\mathcal{Fs}(\text{P},\text{c}) = \{ f\, |\, 
\mathcal{F}(\text{P},\text{c}, \text{f})\, \text{is defined} \}$
and the \textit{method set function} as
$\mathcal{Ms}(\text{P},\text{c}) = \{ m\, |\, 
\mathcal{M}(\text{P},\text{c}, \text{m})\, \text{is defined} \}$
If you give the field set function a program and class, it gives 
you the fields inside that class. If you give the method set 
function a program and a class, it outputs the methods inside the class. 

\section{Stack Frames and Heaps}

We now begin defining our runtime environment for our language. 
Most object oriented language have the notion of \textit{objects}, 
\textit{references}, \textit{stack frames} and a \text{heap}. 
We will give a model for all of these.

First we define an \textit{address} as a greek 
iota symbol subscripted with a natural number. 
For example, the first address is $\iota_0$, second address is $\iota_1$, and so on.
We therefore define the set all possible addresses, the \textit{address set}, as follows.

\highlightdef{\textbf{Address Set}: $addr = \{ \iota_i\, |\, i \in \{0,1,2,...\} \}$ }

Under this scheme, we can work with a \textit{countable infinite} 
number of memory addresses. 
But we will see soon that when we define operational semantics
we will involve a \textit{finite} a number of computation 
steps limiting the computation to using a finite number of addresses. 
So this model doesn't limit the applicability to real machines.

Next we define a \textit{value} to be either true, false, null, \textit{or 
an address} (in the same way addresses were a previously defined). 
So the set of all possible values, 
the \textit{value set}, can be defined as follows.

\highlightdef{\textbf{Value Set}: $val = \{ \text{true, false, null} \} \cup addr $}

A field can only have values that are taken from this value set. 
In fact, this coindices exactly with how we define an \textit{object}. 
An object can be seen as an instance 
of a class where we assign values to the fields of the objects. 
The values assigned have to be from the value set.

\highlightdef{\textbf{Objects}: $object = ClassId \times (FieldId \rightarrow val)$}

Recall that we said a value is either true, false, null, or 
an address. Well, a value that is an address is called a
 \textit{pointer} and say that the value points to the contents of memory.
In our model, \textit{pointers only point to addresses on heap}. 
Furthermore, the \textit{heap can only contain objects}.

In other words, a pointer cannot point to other pointers nor can they point to the primitive language values: true. false, null. Pointers are implicit – there are 
no pointers to pointers. To formalise this, we define our memory, the stack 
frame $\phi$ and heap $\chi$.

\highlightdef{\textbf{Heap}: $heap = addr \rightarrow object$}

The heap is a function that maps address to objects. 
This definition tells us that the heap can only store objects. 
This is definition is also how we deference pointers. When we 
dereference a pointer, the result is an object. This tells us that 
pointers can only point to objects.

Finally we have a stack frame.
\highlightdef{\textbf{Stack Frame}: $stack frame = addr \times val$ }
The stack frame is a tuple. The first component gives an address for the this
pointer in the current execution context. We give an object on the heap for this 
to point to. The second component of the tuple gives a value for x in the 
current execution context. Recall that right now, our methods always have exactly
one parameter called x. The second component of the stack frame gives a value to this x. 

\frmrule

\begin{example}
Below shows how we an example of a stack frame $\phi_0$ and heap $\chi_0$ 
for some execution context of $P_{BP}$. 


Below shows how we might visualise the stack frame and heap.

\end{example}

\frmrule

\begin{example}
Below shows a visualising of the stack frame and heap for 
some execution context of $P_{BP}$.
Write out formally the heap $\chi_9$ and frame $\phi_9$ 
which correspond to the following diagram:

\end{example}


\frmrule



\section{Updating Execution Contexts}

To describe an execution context update, we use the notation 
$A[s \mapsto t]$ which says informally \textit{take A, and update
$s$ to $t$}. Here $A$ is can be anything that makes up our execution context 
(like the heap) and $s$ is a particular stateful element of $A$ that is to be 
updated to the new element, $t$. This notation was used in 
\textit{Models of Computation} when we defined the operation semantics of the 
while language.

In $\mathcal{L}_1$ we are interesting in updating two things, (i) fields inside objects and 
(ii) the heap. Before we do this, recall that we defined an object as a 
function with signature $object = ClassId \times (FieldId \rightarrow val)$ 
and we defined the heap as a function with signature
$heap = addr \rightarrow object$. With these signatures in mind lets see 
how to define updating.

\begin{itemize}   
\renewcommand{\labelitemi}{$\Box$}
\item \textbf{Updating Heap}
Given heap $\chi$, we define $\chi[\iota \mapsto o]$ as a new, updated heap.
This says informally, \textit{take the heap $\chi$ and update the 
object at address $\iota$ to have object $o$}.

Firstly we have: $\chi[\iota \mapsto o](\iota) = o$. 
This tells us that new heap has the object $o$ at address $\iota$.
We also require, $\chi[\iota \mapsto o](\iota') = \chi(\iota')$ if $\iota' \neq \iota$. 
This says that the address that are not $\iota$ must stay the same.
The update only updates on field at a time. We want to change just $\iota$ and leave 
the other addresses alone.
\item \textbf{Updating Fields} 
Given object $o$, we define $o[f \mapsto v]$ as a new, updated object.
This says informally, \textit{take the object $o$ and update the 
value of its field $f$ to have a new value $v$}.

First the update must satisfy: $o[f \mapsto v] \downarrow_1 = o \downarrow_1$. 
This is a requirement that the new object has the same class id as the old one.
Next the update must change the field: $o[f \mapsto v] \downarrow_2 (f) = v$.
The field $f$ of the new object must have value $v$. 
Finally we require, $o[f \mapsto v] \downarrow_2 (f') = f'$ if $f' \neq f$. 
This says that the fields that are not $f$ must stay the same.
The update only updates on field at a time.
\end{itemize}

\highlightdef{\textbf{Updating}:
$o[f \mapsto v]$ describes \textit{field update} and
$\chi[z \mapsto v]$ describes \textit{heap update}}


Let's see some examples of how we can use these two definitions 
to update a given execution context.

\frmrule

% Slide 19 typo, update express has extra left bracket. 
% Diagram has \chi_9 but should read \chi_1, also 
% it should have good:true and owner has the pointer (not good)

\begin{example}
Modify the diagram below, so that it describes the 
update: $$\chi_1 = \chi_0[\iota_3
\mapsto (\chi_0(\iota_3)[\text{good} \mapsto \text{true}])]$$

\begin{figure}[h]
\begin{tikzpicture}[
  title/.style={},
  object/.style={rectangle split, rectangle split draw splits=false,
  draw, text width=2cm}
]

  % text labels
  \node (stackflabel) {$\phi_0$};
  \node  [below=1cm of stackflabel] (heaplabel) {$\chi_0$};

  % stack frame
  \node[rectangle split, rectangle split horizontal,
  rectangle split parts=2, rectangle split draw splits=false, draw, 
  text width=1cm, anchor=center, text centered, right=0.2cm of stackflabel]
  (stackframe){ this \nodepart{two} x};

  % objects
  \node[object,rectangle split parts=3, below=0.5cm of stackframe] 
  (book){ \nodepart{one} :Book \nodepart{two} good: false
    \nodepart{three} owner: };
  \node[object,rectangle split parts=2, right=1cm of book] 
  (person1) {\nodepart{one} :Person \nodepart{two} like: null};
  \node[object, rectangle split parts=2, right=1cm of person1] 
  (person2) { \nodepart{one} :Person \nodepart{two} like:};

  % draw lines on first nodepart splits for the objects
  \draw(book.one split west) -- (book.one split east);
  \draw(person1.one split west) -- (person1.one split east);
  \draw(person2.one split west) -- (person2.one split east);

  % draw lines connecting stack frame and objects. Respectively:
  % this to book, x to person1, book to person2, person2 to person2
  \path[draw,>=triangle 45, ->] (stackframe.one south) -- ++(0,0.1) 
      -- ++(0,-0.4) -| (person2.north);
  \path[draw,>=triangle 45, ->] (stackframe.two east) -- ++(-0.2,0) 
      -| (person1.north);
  \path[draw,>=triangle 45, ->] (book.three east) -- ++(-0.2,0) 
      -- ++(0.6,0) -- ++(0,-0.6) -| (person2.south);
  \path[draw,>=triangle 45, ->] (person2.two east) -- ++(-0.2,0) 
      -- ++(1,0) |- (person2.east);
 

\end{tikzpicture}
\end{figure} 

\end{example}

\frmrule

\begin{figure}[h]
\begin{tikzpicture}[
  title/.style={},
  object/.style={rectangle split, rectangle split draw splits=false,
  draw, text width=2cm}
]

  % text labels
  \node (stackflabel) {$\phi_0$};
  \node  [below=1cm of stackflabel] (heaplabel) {$\chi_0$};

  % stack frame
  \node[rectangle split, rectangle split horizontal,
  rectangle split parts=2, rectangle split draw splits=false, draw, 
  text width=1cm, anchor=center, text centered, right=0.2cm of stackflabel]
  (stackframe){ this \nodepart{two} x};

  % objects
  \node[object,rectangle split parts=3, below=0.5cm of stackframe] 
  (book){ \nodepart{one} :Book \nodepart{two} good: true
    \nodepart{three} owner: };
  \node[object,rectangle split parts=2, right=1cm of book] 
  (person1) {\nodepart{one} :Person \nodepart{two} like: null};
  \node[object, rectangle split parts=2, right=1cm of person1] 
  (person2) { \nodepart{one} :Person \nodepart{two} like:};

  % draw lines on first nodepart splits for the objects
  \draw(book.one split west) -- (book.one split east);
  \draw(person1.one split west) -- (person1.one split east);
  \draw(person2.one split west) -- (person2.one split east);

  % draw lines connecting stack frame and objects. Respectively:
  % this to book, x to person1, book to person2, person2 to person2
  \path[draw,>=triangle 45, ->] (stackframe.one south) -- ++(0,0.1) 
      -- ++(0,-0.4) -| (person2.north);
  \path[draw,>=triangle 45, ->] (stackframe.two east) -- ++(-0.2,0) 
      -| (person1.north);
  \path[draw,>=triangle 45, ->] (book.three east) -- ++(-0.2,0) 
      -- ++(0.6,0) -- ++(0,-0.6) -| (person2.south);
  \path[draw,>=triangle 45, ->] (person2.two east) -- ++(-0.2,0) 
      -- ++(1,0) |- (person2.east);
 

\end{tikzpicture}
\end{figure} 

The only change is that the book.good is now true instead of false.

Now if we were to write out formally the new heap.
\[ \begin{array}{lllll}
\chi_1(\iota_3) & = & ... \\ 
\chi_1(\iota_4) & = & ... \\
\chi_1(\iota_5) & = & ... \\
\end{array}\] 

\frameans{Complete the definition of $\chi_1$}{
$\chi_1(\iota_3) = (\text{Book}, (\text{good} \mapsto \text{true}),
\text{owner} \mapsto \iota_5)) $ \\ 
$\chi_1(\iota_4) = (\text{Person}, (\text{like} \mapsto \iota_3))$ \\
$\chi_1(\iota_5) = (\text{Person}, (\text{like} \mapsto \text{null}))$
}



Note: there is no operation which changes the class of an object.
Once an object has been created at runtime, it is has a \textit{static binding}
to the class. Also notice that there is no operation to change the stack frame.
In object-oriented languages, changing the values of this isn't usually done.
Also, changing directly the values of arguments isn't too common. 



\section{Inference Rules Revision}

We need to be describe the type system and the operational semantics of $\mathcal{L}_1$.
We will do this using \textit{inference rules}. An
inference rules have a set of \textit{premises}, and one \textit{conclusion}.
There can be any number of premises (including zero), 
but there is always exactly one conclusion. 
We use a line to separate premises from the conclusion. 
Premises (if any) are put above the line, the conclusion is put below the line.

\inference[\textsc{rulename}]{P_1 & P_2 &  ... & P_n }{Q}

This can be read as \textit{if you can establish $P_1$, $P_2$ ... $P_n$, 
then you can establish $Q$}". We often give an interfence rule a name 
that is written on the side or nearby.

Such inference rules can be used to give an \textit{axiomatic definition}.
Once we have a set of axioms defined, we can combine 
them to prove \textit{theorems}. For example, below shows 
axiomatic definitions for the natural numbers. 
It is a recursive definition that makes use of a successor function.

\inference*[\textsc{zero}]
{}
{\text{Z is a natural number}}

\inference*[\textsc{succ}]
{\text{n is a natural number}}
{\text{S(n) is a natural number}}

By using these two axiomatic definitions, we
can give a derivation for the theorem: \textit{$S(S(Z))$ is a natural number}.
As you can see from the derivation, the idea is that we combine the axioms 
together in a tree like structure to arrive at the result we want. 
We match conclusion of one rule to the premise of another. 

Given axioms, we can write axioms that represent operations.
For example, we can write a definition for addition using two rules.

\inference*[\textsc{addzero}]
{}
{\text{add(Z,n) = n}}

\inference*[\textsc{addsucc}]
{}
{\text{add(Z,n) = n}}


 
We can also give an axiomatic definitions for the addition relation.



\section{Operational Semantics I}

We now define the operational semantics of $\mathcal{L}_1$.
We will use \textit{big-step semantics}.
Each premise and conclusion is written in terms 
of a \textit{full evaluation}. 
We can think of this as an evaluation or an execution.

\highlightdef{We define \textbf{Big-Step Semantics} relation, $\leadsto$}

We are given an input expression $e \in expr$ and our context
of stack frame $\phi$ and heap $\chi$. This expression can 
be evaluated to some value $v$. 
By writing $(e, \phi, \chi) \leadsto (v,\phi, \chi')$ we are 
saying that we can work out the value and 
the state of the stack and heap afterwards. 
When we evaluate this, 
the stack frame won't change but the heap may.
Since we know the stack won't change we can write 
$(e, \phi, \chi) \leadsto (v, \chi')$ instead.

\highlightdef{\textbf{Expression Evaluation}:
$(e, \phi, \chi) \leadsto (v, \chi')$ denotes
that $e$ evaluates to $v$ where the execution context starts 
at $(\phi, \chi)$ and ends up at $(\phi, \chi')$}

Below we introduce a few definitions to get us started. 
These rules (plus a few others later) make up operation semantics of 
$\mathcal{L}_1$.

\inference*[\textsc{val}]
{}
{(v, \phi, \chi) \leadsto (v,\chi) }

\inference*[\textsc{this}]
{}
{(\text{this}, (\iota_i, v), \chi) \leadsto (\iota_i,\chi) }

\inference*[\textsc{param}]
{}
{(x, (\iota_i, v), \chi) \leadsto (v,\chi) }

\inference[\textsc{if-true}]
{(e, \phi, \chi) \leadsto (\text{true}, \chi') & 
(e_1, \phi, \chi) \leadsto (v,\chi'')}
{(\text{if} \,e\, \text{then} \,e_1\, \text{else} \,e_2, \phi, \chi') 
\leadsto (v,\chi'') }

\inference[\textsc{if-false}]
{(e, \phi, \chi) \leadsto (\text{false}, \chi') & 
(e_2, \phi, \chi) \leadsto (v,\chi'')}
{(\text{if} \,e\, \text{then} \,e_1\, \text{else} \,e_2, \phi, \chi) 
\leadsto (v,\chi'') }

\inference[\textsc{field}]
{(\text{e}, \phi, \chi) \leadsto (\iota, \chi')}
{(\text{e.f}, \phi, \chi) \leadsto (\chi'(\iota)\downarrow_{2}(f),\chi') }


\frmrule

\begin{example}
Write out the derivation for evaluating the following expression
under the given evaluation context, $\phi_0$, $\chi_0$. 

$$\text{if} \,(\text{this.good})\, \text{x.like} \, \text{else} \, \text{false}$$

\begin{figure}[h]
\begin{tikzpicture}[
  title/.style={},
  object/.style={rectangle split, rectangle split draw splits=false,
  draw, text width=2cm}
]

  % text labels
  \node (stackflabel) {$\phi_0$};
  \node  [below=1cm of stackflabel] (heaplabel) {$\chi_0$};

  % stack frame
  \node[rectangle split, rectangle split horizontal,
  rectangle split parts=2, rectangle split draw splits=false, draw, 
  text width=1cm, anchor=center, text centered, right=0.2cm of stackflabel]
  (stackframe){ this \nodepart{two} x};

  % objects
  \node[object,rectangle split parts=3, below=0.5cm of stackframe] 
  (book){ \nodepart{one} :Book \nodepart{two} good: true
    \nodepart{three} owner: };
  \node[object,rectangle split parts=2, right=1cm of book] 
  (person1) {\nodepart{one} :Person \nodepart{two} like:};
  \node[object, rectangle split parts=2, right=1cm of person1] 
  (person2) { \nodepart{one} :Person \nodepart{two} like:null};

  % draw lines on first nodepart splits for the objects
  \draw(book.one split west) -- (book.one split east);
  \draw(person1.one split west) -- (person1.one split east);
  \draw(person2.one split west) -- (person2.one split east);

  % draw lines connecting stack frame and objects. Respectively:
  % this to book, x to person1, book to person2, person1 to book
  \path[draw,>=triangle 45, ->] (stackframe.one south) -- ++(0,0.1) 
      -- ++(0,-0.3) -| (book.north);
  \path[draw,>=triangle 45, ->] (stackframe.two east) -- ++(-0.2,0) 
      -| (person1.north);
  \path[draw,>=triangle 45, ->] (book.three south) -- ++(0,0.1) 
      -- ++(0,-0.4) -| (person2.south);
  \path[draw,>=triangle 45, ->] (person1.three south) -- ++(0,0.1) 
      -- ++(0,-0.4) -- ++(-1.5,0) |- (book.east);
 

\end{tikzpicture}
\end{figure} 

\frmrule

It helps to work from the bottom up.
The expression only pattern-matches the bottom of 
\textsc{if-true} and \textsc{if-false}. We can 
see from the diagram that \text{this.good} is true.
So we can guess that the last rule must be \textsc{if-true}. 
We just need to work out how to branch upwards.
Hopefully we will reach the base cases allowing our derivation tree complete. 


\inference[\textsc{if-true}]
{
   \dotsm & \dotsm
}
{
(\text{if} \,(\text{this.good})\, \text{x.like} \, \text{else} \, 
\text{false}, \phi_0, \chi_0)
\leadsto
\dotsm
}

We can start by evaluating $\text{this.good}$.

\inference[\textsc{if-true}]
{
   (\text{this.good}, \phi_0, \chi_0) \leadsto \dotsm & \dotsm
}
{
(\text{if} \,(\text{this.good})\, \text{x.like} \, \text{else} \, 
\text{false}, \phi_0, \chi_0)
\leadsto
\dotsm
}

\frmrule

If we do pattern-matching on conclusions for $\text{this.good}$,
the only matching conclusion is that in the \textsc{field} rule.
So we should branch upwards using the \textsc{field} rule.


\inference[\textsc{if-true}]
{
   \inference[\textsc{field}]
   {\dotsm}
   {(\text{this.good}, \phi_0, \chi_0) \leadsto \dotsm}
   & \dotsm
}
{
(\text{if} \,(\text{this.good})\, \text{x.like} \, \text{else} \, 
\text{false}, \phi_0, \chi_0)
\leadsto
\dotsm
}

This means we need to evaluate $\text{this}$.

\inference[\textsc{if-true}]
{
   \inference[\textsc{field}]
   {(\text{this}, \phi_0, \chi_0) \leadsto \dotsm}
   {(\text{this.good}, \phi_0, \chi_0) \leadsto \dotsm}
   & \dotsm
}
{
(\text{if} \,(\text{this.good})\, \text{x.like} \, \text{else} \, 
\text{false}, \phi_0, \chi_0)
\leadsto
\dotsm
}

\frmrule

If we do pattern-matching on conclusions for $\text{this}$,
the only matching conclusion is that in the \textsc{this} rule.
So we should branch upwards using the \textsc{this} rule.

\inference[\textsc{if-true}]
{
  \inference[\textsc{field}]
  {
    \inference[\textsc{this}]
    {}
    {(\text{this}, \phi_0, \chi_0) \leadsto \dotsm}
  }
  {
  (\text{this.good}, \phi_0, \chi_0) \leadsto \dotsm
  }
  & \dotsm
}
{
(\text{if} \,(\text{this.good})\, \text{x.like} \, \text{else} \, 
\text{false}, \phi_0, \chi_0)
\leadsto
\dotsm
}

Given that $\phi_0 = (\iota_3, \iota_4)$,
by using the \textsc{this} rule we can
complete the evaluation for this.

\inference[\textsc{if-true}]
{
  \inference[\textsc{field}]
  {
    \inference[\textsc{this}]
    {}
    {(\text{this}, \phi_0, \chi_0) \leadsto (\iota_3, \chi_0)}
  }
  {
  (\text{this.good}, \phi_0, \chi_0) \leadsto \dotsm
  }
  & \dotsm
}
{
(\text{if} \,(\text{this.good})\, \text{x.like} \, \text{else} \, 
\text{false}, \phi_0, \chi_0)
\leadsto
\dotsm
}

Now we can move downwards and complete the missing evaluations. 
According to \textsc{field}, the field access evaluates to 
$\chi_0(\iota_3) \downarrow_{2} (\text{good})$ which equals ...

\frameans{}{true}

So we have 

\inference[\textsc{if-true}]
{
  \inference[\textsc{field}]
  {
    \inference[\textsc{this}]
    {}
    {(\text{this}, \phi_0, \chi_0) \leadsto (\iota_3, \chi_0)}
  }
  {
  (\text{this.good}, \phi_0, \chi_0) \leadsto (\text{true}, \chi_0)
  }
  & \dotsm
}
{
(\text{if} \,(\text{this.good})\, \text{x.like} \, \text{else} \, 
\text{false}, \phi_0, \chi_0)
\leadsto
\dotsm
}


\end{example}



Now that we've done the left subtree for \textsc{if-true},
let's do the right subtree. This begins by evaluating $\text{x.like}$.

\inference[\textsc{if-true}]
{
  \inference[\textsc{field}]
  {
    \inference[\textsc{this}]
    {}
    {(\text{this}, \phi_0, \chi_0) \leadsto (\iota_3, \chi_0)}
  }
  {
  (\text{this.good}, \phi_0, \chi_0) \leadsto (\text{true}, \chi_0)
  }
  & 
  (\text{x.like}, \phi_0, \chi_0) \leadsto \dotsm
}
{
(\text{if} \,(\text{this.good})\, \text{x.like} \, \text{else} \, 
\text{false}, \phi_0, \chi_0)
\leadsto
\dotsm
}

\frmrule

If we do pattern-matching on conclusions on the expression $\text{x.like}$,
the only matching conclusion is that in the \textsc{field} rule.
So we will branch upwards using the \textsc{field} rule.

\inference[\textsc{if-true}]
{
  \inference[\textsc{field}]
  {
    \inference[\textsc{this}]
    {}
    {(\text{this}, \phi_0, \chi_0) \leadsto (\iota_3, \chi_0)}
  }
  {
  (\text{this.good}, \phi_0, \chi_0) \leadsto (\text{true}, \chi_0)
  }
  &
  \inference[\textsc{field}]
  {
  \dotsm
  }
  {
  (\text{x.like}, \phi_0, \chi_0) \leadsto \dotsm
  } 
}
{
(\text{if} \,(\text{this.good})\, \text{x.like} \, \text{else} \, 
\text{false}, \phi_0, \chi_0)
\leadsto
\dotsm
}

This means we need to evaluate x.

\inference[\textsc{if-true}]
{
  \inference[\textsc{field}]
  {
    \inference[\textsc{this}]
    {}
    {(\text{this}, \phi_0, \chi_0) \leadsto (\iota_3, \chi_0)}
  }
  {
  (\text{this.good}, \phi_0, \chi_0) \leadsto (\text{true}, \chi_0)
  }
  &
  \inference[\textsc{field}]
  {
  (\text{x}, \phi_0, \chi_0) \leadsto \dotsm
  }
  {
  (\text{x.like}, \phi_0, \chi_0) \leadsto \dotsm
  } 
}
{
(\text{if} \,(\text{this.good})\, \text{x.like} \, \text{else} \, 
\text{false}, \phi_0, \chi_0)
\leadsto
\dotsm
}

\frmrule

Now we do pattern matching checking which 
rules match expression $x$ in their conclusions.
The only matching conclusion is in the \textsc{param} rule.
So we will branch upwards using the \textsc{param} rule.

\inference[\textsc{if-true}]
{
  \inference[\textsc{field}]
  {
    \inference[\textsc{this}]
    {}
    {(\text{this}, \phi_0, \chi_0) \leadsto (\iota_3, \chi_0)}
  }
  {
  (\text{this.good}, \phi_0, \chi_0) \leadsto (\text{true}, \chi_0)
  }
  &
  \inference[\textsc{field}]
  {
    \inference[\textsc{param}]
    {}
    {(\text{x}, \phi_0, \chi_0) \leadsto \dotsm}
  }
  {
  (\text{x.like}, \phi_0, \chi_0) \leadsto \dotsm
  } 
}
{
(\text{if} \,(\text{this.good})\, \text{x.like} \, \text{else} \, 
\text{false}, \phi_0, \chi_0)
\leadsto
\dotsm
}

Given that $\phi_0 = (\iota_3, \iota_4)$,
by using the \textsc{param} rule we can
complete the evaluation for x.

\inference[\textsc{if-true}]
{
  \inference[\textsc{field}]
  {
    \inference[\textsc{this}]
    {}
    {(\text{this}, \phi_0, \chi_0) \leadsto (\iota_3, \chi_0)}
  }
  {
    (\text{this.good}, \phi_0, \chi_0) \leadsto (\text{true}, \chi_0)
  }
  &
  \inference[\textsc{field}]
  {
    \inference[\textsc{param}]
    {}
    {(\text{x}, \phi_0, \chi_0) \leadsto (\iota_4, \chi_0)}
  }
  {
  (\text{x.like}, \phi_0, \chi_0) \leadsto \dotsm
  } 
}
{
(\text{if} \,(\text{this.good})\, \text{x.like} \, \text{else} \, 
\text{false}, \phi_0, \chi_0)
\leadsto
\dotsm
}

\frmrule

Now we can move downwards and complete the missing evaluations. 
According to \textsc{field}, the field access evaluates to 
$\chi_0(\iota_4) \downarrow_{2} (\text{like})$ which equals ...

\frameans{}{$\iota_3$}

So the field evaluation gives 

\inference[\textsc{if-true}]
{
  \inference[\textsc{field}]
  {
    \inference[\textsc{this}]
    {}
    {(\text{this}, \phi_0, \chi_0) \leadsto (\iota_3, \chi_0)}
  }
  {
    (\text{this.good}, \phi_0, \chi_0) \leadsto (\text{true}, \chi_0)
  }
  &
  \inference[\textsc{field}]
  {
    \inference[\textsc{param}]
    {}
    {(\text{x}, \phi_0, \chi_0) \leadsto (\iota_4, \chi_0)}
  }
  {
  (\text{x.like}, \phi_0, \chi_0) \leadsto (\iota_3, \chi_0)
  } 
}
{
(\text{if} \,(\text{this.good})\, \text{x.like} \, \text{else} \, 
\text{false}, \phi_0, \chi_0)
\leadsto
\dotsm
}

We can finish the evaluation of the if statement.

\inference[\textsc{if-true}]
{
  \inference[\textsc{field}]
  {
    \inference[\textsc{this}]
    {}
    {(\text{this}, \phi_0, \chi_0) \leadsto (\iota_3, \chi_0)}
  }
  {
    (\text{this.good}, \phi_0, \chi_0) \leadsto (\text{true}, \chi_0)
  }
  &
  \inference[\textsc{field}]
  {
    \inference[\textsc{param}]
    {}
    {(\text{x}, \phi_0, \chi_0) \leadsto (\iota_4, \chi_0)}
  }
  {
  (\text{x.like}, \phi_0, \chi_0) \leadsto (\iota_3, \chi_0)
  } 
}
{
(\text{if} \,(\text{this.good})\, \text{x.like} \, \text{else} \, 
\text{false}, \phi_0, \chi_0)
\leadsto
(\iota_3, \chi_0)
}


\section{Operational Semantics II}


We now introduce more rules.

\begin{prooftree}
\AxiomC{$(e,\phi, \chi) \leadsto (\iota, \chi')$}
\AxiomC{$(e',\phi, \chi') \leadsto (v, \chi'')$}
\LeftLabel{\textsc{assign}}
\RightLabel{$\begin{smallmatrix}
                    \chi'(\iota) \text{ is defined} \\
                    \chi''' = \chi''[\iota \mapsto \chi''(\iota)[f \mapsto v]]
                  \end{smallmatrix}$}
\BinaryInfC{$(\text{e.f := e'},\phi, \chi) \leadsto (v, \chi''')$}
\end{prooftree}

\begin{prooftree}
\def\defaultHypSeparation{\hskip .01in}
\AxiomC{$(e_0,\phi, \chi_1) \leadsto (\iota, \chi_2)$}
\AxiomC{$(e_1,\phi, \chi_2) \leadsto (v, \chi_3)$}
\AxiomC{$(e_2,\phi, \chi_3) \leadsto (v, \chi_4)$}
\LeftLabel{\textsc{method}}
\RightLabel{$\begin{smallmatrix}
                    \chi_3(\iota) \downarrow_1 = \text{c} \\
                    \mathcal{M}(\text{P},\text{c}, \text{m})\, = t_1 m (t_2 x) \{ e_2 \}
                  \end{smallmatrix}$}
\TrinaryInfC{$(\text{e}_0.\text{m}(\text{e}_1),\phi, \chi) \leadsto (v, \chi_4)$}
\end{prooftree}


\begin{prooftree}
\def\defaultHypSeparation{\hskip .01in}
\AxiomC{ }
\LeftLabel{\textsc{new}}
\RightLabel{$\begin{smallmatrix}
              \chi' = \chi[\iota \mapsto (c, (f_1 \mapsto v_1, \dotsm, f_r \mapsto v_r))] \\
              \mathcal{Fs}(\text{P},\text{c})\, = \{f_1, \dotsm, f_r\} \\
              \forall l \in 1, ..., r: \; v_l \text{ is init val for } 
              \mathcal{F}(\text{P},\text{c},\text{f}_1)
            \end{smallmatrix}$}
\UnaryInfC{$(\text{new c},\phi, \chi) \leadsto (\iota, \chi')$}
\end{prooftree}

\rotatebox{90}{\parbox{4.5cm}{Long Header Over Several Lines}}



\begin{example}
Write out the derivation for executing \\
(a) $\text{x.like.good := false}, \phi_0, \chi_0$ \\
(b) $\text{this.owner.meet(x)}, \phi_0, \chi_0$ \\
for the following execution context:

\begin{figure}[h]
\begin{tikzpicture}[
  title/.style={},
  object/.style={rectangle split, rectangle split draw splits=false,
  draw, text width=2cm}
]

  % text labels
  \node (stackflabel) {$\phi_0$};
  \node  [below=1cm of stackflabel] (heaplabel) {$\chi_0$};

  % stack frame
  \node[rectangle split, rectangle split horizontal,
  rectangle split parts=2, rectangle split draw splits=false, draw, 
  text width=1cm, anchor=center, text centered, right=0.2cm of stackflabel]
  (stackframe){ this \nodepart{two} x};

  % objects
  \node[object,rectangle split parts=3, below=0.5cm of stackframe] 
  (book){ \nodepart{one} :Book \nodepart{two} good: true
    \nodepart{three} owner: };
  \node[object,rectangle split parts=2, right=1cm of book] 
  (person1) {\nodepart{one} :Person \nodepart{two} like:};
  \node[object, rectangle split parts=2, right=1cm of person1] 
  (person2) { \nodepart{one} :Person \nodepart{two} like:null};

  % draw lines on first nodepart splits for the objects
  \draw(book.one split west) -- (book.one split east);
  \draw(person1.one split west) -- (person1.one split east);
  \draw(person2.one split west) -- (person2.one split east);

  % draw lines connecting stack frame and objects. Respectively:
  % this to book, x to person1, book to person2, person1 to book
  \path[draw,>=triangle 45, ->] (stackframe.one south) -- ++(0,0.1) 
      -- ++(0,-0.3) -| (book.north);
  \path[draw,>=triangle 45, ->] (stackframe.two east) -- ++(-0.2,0) 
      -| (person1.north);
  \path[draw,>=triangle 45, ->] (book.three south) -- ++(0,0.1) 
      -- ++(0,-0.4) -| (person2.south);
  \path[draw,>=triangle 45, ->] (person1.three south) -- ++(0,0.1) 
      -- ++(0,-0.4) -- ++(-1.5,0) |- (book.east);
 

\end{tikzpicture}
\end{figure} 


\end{example}


\frmrule

There is no rule that changes a object completely.
We can only update objects, on a field-by-field basis.
Also notice that we can add objects to the heap using \textsc{new},
but there are no rules which remove objects from the heap. 
We could end up with objects that are never referenced. 
An extension to L1 could be to implement a garbage-collecting scheme.




\section{Operational Semantics III}


\section{Proving Determinism}


\section{Type System}