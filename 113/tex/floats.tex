

\chapter{Floating-point Arithmetic}


\section{Fixed Point Arithmetic}

\begin{figure}[h]
\begin{tikzpicture} [
   every node/.style={node_sty, align=center},
   edge from parent/.style={edge_sty},
   edge from parent path={(\tikzparentnode.south) 
   |- ($(\tikzparentnode.south)!0.5!(\tikzchildnode.north)$) -| (\tikzchildnode.north)},
   level 1/.style={sibling distance=4cm, level distance = 0.9cm},
   ch/.style={text width=2.5cm}
]
   \node 
   (root) {Representing Real numbers in Binary}
      child {node[ch] {Dyadic rationals}}
      child {node[ch] {Recurring values}}
      child {node[ch] {Irrational values}};
   
\end{tikzpicture}
\end{figure}

\frmrule

\begin{example}
Show that 0.1 is \textit{not} a dyadic rational number.
\[ 
\begin{array}{ll}
0.1 & = \frac{0.1}{1} = ... = \frac{1.6}{16} \\
    & = \left(\frac{1}{16}\right) + \frac{0.6}{16} = \left(\frac{1}{16}\right) + \frac{1.2}{32} \\
    & = \left(\frac{1}{16} + \frac{1}{32}\right) + \frac{0.2}{32} = ... = 
    \left(\frac{1}{16} + \frac{1}{32}\right) + \frac{1.6}{256} = 
    \left(\frac{1}{16} + \frac{1}{32}\right) + \frac{1}{16} \times \frac{\text{\textbf{1.6}}}{\textbf{16}}
\end{array}
\] 


\end{example}

\begin{example}
\textbf{(a)} Show that 42.6875 is a \textit{dyadic rational number} \\
\textbf{(b)} Hence write 42.6875 in binary

$42.6875 = 32 + 10.6875 = 32 + 8 + 2.6875 = 32 + 8 + 2 + 0.6875 = 32 + 8 + 2 + ....$
\end{example}


\frmrule

\begin{example}
What is the the binary: 0.11111111111111111111... (recurring ones) equal to?
\end{example}


\frmrule

We can obtain $x'$ by either \textit{rounding} or \textit{truncation}.


\frmrule

\section{Introducing IEEE Standard}

Previously we have looked at how floating point binary can represent certain rational numbers. 
However there are many formats possible. 
Now we will look certain representations that are standardised by:

\highlightdef{\textbf{IEEE}: The Institute of Electrical and Electronics Engineers}

\textsc{ieee} (pronounced \textit{eye-triple-E})
stands for \textit{The Institute of Electrical and Electronics Engineers}. 
They are a not-for-profit organisation that specialises in \textit{laying down industry standards}, 
as well as holding conferences and providing a medium for the publication of academic papers. 

\textsc{ieee} have formalised what is, by far, the \textit{most popular floating-point representations}
used in computer systems. The standard defines the \textit{storage format} of binary floating point numbers 
(how the fields are stored in memory), the exact \textit{semantics of arithmetic operations} (including how to handle 
certain few edge cases) and rules for \textit{error conditions}
(overflow, underflow, truncation rules, ...).



\begin{tikzpicture}
  [every node/.style={node_sty2, align=center}]
  \node (A) {\textsc{ieee} standard for floats};
  \coordinate (a1) at ($(A.south)+(-0.5cm, 0.0cm)$); 
  \coordinate (a2) at ($(A.south)+(0.5cm, 0.0cm)$); 
  %
  \node [below left=0.2cm and -0.5cm of A, text width=2.5cm]  (B) {Representation};
  \node [below=0.2cm of B, text width=2.5cm] (C) {Special Values};
  \draw[edge_sty] (a1) |- (B.east);
  \draw[edge_sty] (a1) |- (C.east);
  %
  \node [below right=0.2cm and -0.5cm of A, text width=2.5cm] (D) {Operations};
  \node [below=0.2cm of D, text width=2.5cm] (E) {Error conditions};
  \draw[edge_sty] (a2) |- (D.west);
  \draw[edge_sty] (a2) |- (E.west);
\end{tikzpicture}



\frmrule



\section{Operations}

\begin{example}
Which is larger, -0 or +0?
\end{example}




\highlightdef{Addition/subtraction requires two exponents to be compared. 
\textsc{ieee} uses excess-n as it allows exponents to be compared more easily. }

Checking whether $a < b$ is simpler in excess-n compared to two's complement. 


\highlightdef{\textbf{Loss of Precision} can occur when an integer is converted to a float}
This is a common warning message for beginner \textsc{c} programmers. 




\section{Special Values}


\highlightdef{\textbf{Special Values}: +0, -0,  +\textsc{inf}, -\textsc{inf}, \textsc{nan} }


\begin{lstlisting}
int main (int argc, char **argv) 
{ 
  float a = 1.0/0.0; 
  float b = a * -100; 
  float c = b/a; 
  printf ("\nValue of a = %f\n\n", a); 
  printf ("\nValue of b = %f\n\n", b); 
  printf ("\nValue of c = %f\n\n", c); 
} 
\end{lstlisting}


\section{Floating Point Systems}

We now look at floating point systems in general.



\section{Error Analysis}


\highlightdef{\textbf{Relative Error}: }


\highlightdef{\textbf{Absolute Error}: }



\highlightdef{Floating-point arithmetic is \textit{not associative}}
% http://stackoverflow.com/questions/6430448/why-doesnt-gcc-optimize-aaaaaa-to-aaaaaa?rq=1
But it is \textit{commutative}. 

\begin{example}
The example below demonstrates that floating-point arithmetic is not associative.

\begin{lstlisting}
int main(int intc, char **argv) 
{
    int x = 1;
    for (int i = 1; i < 10000000; i++) {
       x += 1/i;
    }
    int y = 1;
    for (int i = 10000000; i >= 0; i--) {
       y += 1/i;
    }
    printf("x: %d\n", x);
    printf("y: %d\n", y);
    printf("abs(x-y): %d\n", abs(x-y));
}
\end{lstlisting}
\end{example}

\frmrule 

% Real-world: most machines have a floating point unit
% why? what does a floating point unit do?
% relate to the work done

\frmrule 



\highlightdef{The \textbf{Machine Epsilon} measures the \textit{accuracy} 
of a floating-point system}

Although we call this value the \textit{machine} epsilon, 
it doesn't necessarily characterise the entire system. 
Not all parts of the machine use floating point arithmetic. 
We call this value the \textit{machine} epsilon 
to highlight this number is a result of the \textit{limitation} of the machine 
and the fact that we do not have the resources to have perfect accuracy.  


Below shows code to find the machine epsilon. 

\begin{lstlisting}
int main(int intc, char **argv) 
{
    float machEps = 1.0f;
    do {
        machEps /= 2.0f;
    }
    while ((float) 1.0f + (machEps/2.0f) != 1.0f);
    printf("\nCalculated Machine epsilon: %g\n\n", machEps); 
    return 0;
}
\end{lstlisting}




