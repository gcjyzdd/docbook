

\chapter{Memory}


\section{Introducing DRAM Structure}


A typical DRAM memory is laid out as a square
array of memory cells with an equal number of rows
and columns. Each memory cell stores one bit. The
bits are addressed by using half of the bits (the most
significant half) to select a row and the other half to
select a column.

\begin{example}

How many rows and columns would there be
in the memory cell array in a 16 M by 1 (16 Mbit) DRAM?

\end{example} 


\section{Byte Ordering}

We already have a prioirity on what byte is the most important 
or significant. This is the most significant byte. However 

\highlightdef{\textbf{Endianness} tells us \textit{what direction}
\textit{multibyte data} fills memory starting from the \textsc{msbyte} and finishing at the \textsc{lsbyte}}

A \textit{little}-endian means that we are filling the memory, 
byte-by-byte, moving towards the \textit{little}-end or \textit{lower} addresses.
Whereas \textit{big}-endian means that we are filling the memory, 
moving towards the \textit{big}-end (higher addresses).



\begin{example}
One way to think about byte ordering is to look at memory as an array. 
In Big-endian we fill the array in general from top-left to bottom-right. 
In Little-endian we fill the array in general from bottom-right to top-left. 
\end{example}

\begin{example}
Another way is to look at memory as a magazine. 
Big-endian is the normal way where words fill the page 
from the top left to the bottom right. Little-endian is 
perhaps stranger because words fill the page from the 
bottom-right to the top-left.
\end{example}

\highlightdef{Big-endian is the more natural/intuitive choice}

Yet there are systems that use little endian. %intel x86



\section{RAM Internals}





\highlightdef{A \textit{refresh} is a recharging of the \textsc{ram} capacitors}


\highlightdef{\textbf{DDR} stands for \textit{double data rate}}
\textsc{ddr} allows synchronous data access on both edges of the clock.