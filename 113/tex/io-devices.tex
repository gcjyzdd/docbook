

\chapter{IO Devices}


\section{Introducing IO Devices}

\highlightdef{An \textbf{IO Device} is an 
\textit{an additional piece of hardware} added to memory and cpu}

Perhaps the most important IO device is the \textit{hard disk}. 
First we will look at device controllers which are arguably the heart of the
communication between cpu and devices. 

\section{Device Controllers}

Every io-device has, inside, a \textit{device controller}.  
The device controller is simply \textit{more hardware}, 
but this hardware has a particular purpose. 
Device controllers are placed on an io device to give the cpu an interface for communicating
with the device. In other words, it’s the io-device’s way of making it open to be 
accessed or to be \textit{controlled} by the cpu.

\begin{figure}[h]
\begin{tikzpicture}[
  title/.style={},
  client/.style={rectangle, draw, minimum height=2em},
]
  \node [client] (dev) at (0, 10) {IO Controller};
  \node [title] (ctr) at (0, 9.25) {IO Device};
  \node [draw=black, inner sep=5pt, outer sep=5pt, fit={(dev) (ctr)}] {};
\end{tikzpicture}
\end{figure} 

The processor communicates with this controller but is unaware of how the device 
operates externally. To let the cpu communicate with the device, the io-device provides 
\textit{registers}. If the cpu wants to talk to the device, the cpu writes 
to registers inside the device controller. 

\begin{figure}[h]
\begin{tikzpicture}[
  title/.style={},
  client/.style={rectangle, draw, minimum height=2em},
]

  \node [client] (reg) at (0, 1.5) {Registers};
  \node [title] (ctr) at (0, 0.75) {IO Controller};
  \node [title] (dev) at (0, 0) {IO Device};
  \node [draw=black, inner sep=5pt, outer sep=5pt, fit={(ctr) (reg)}] (ctrfill) {};
  \node [draw=black, inner sep=5pt, outer sep=5pt, fit={(dev) (ctrfill)}] {};



\end{tikzpicture}
\end{figure} 

Hopefully, if the device isn’t too busy, it will somehow know that the cpu said 
something and read its own device control registers. 
If the device wants to talk to the cpu, the device writes its device control registers 
and somehow, the cpu will know about this write and read the contents ... if the cpu isn’t too busy. 
\highlightdef{A \textbf{Device Controller} is hardware inside an io device that contains
\textit{registers} in order to \textit{provide communication with the cpu}.}
Device controllers are also known as \textit{io-controllers}, 
\textit{io-adapters} or \textit{io-modules}. 

Although the registers inside a device controller provide a shared storage between cpu 
and the device, there is a difficulty in making one side aware that the other has said 
something. We will refer to this problem as the \textit{CPU-Device Scheduling Problem}. 
Later we will look at schemes that attempt solve this problem.

Very often the device controller is a very small part of the actual hardware device 
(like a small chip or block stuck on the back). 
Sometimes the device controller is the majority of the hardware device 
(like a network card).

\highlightdef{The \textbf{IO Address Space} is the \textit{processor's collection}
of \textit{every controller register} of every device}

The device controller’s registers appear to the processor as normal memory locations. 
Hence the cpu can easily manipulate the device controller’s registers by simply 
using instructions such as mov, and, or, test. 
The operand be will an io-address-space memory address.

\section{Buses and Bridges}

\highlightdef{An \textbf{IO Bus} is a network of io devices.}
A computer usually has many io bus networks. However, there is a need for separatimh networks 
of io-devices because devices operate \textit{at different speeds}. 
Some examples of common standardised io-bus networks are \textit{PCI}, 
\textit{IDE}, \textit{SCSI}, \textit{USB}, \textit{ISA} and \textit{Firewire}. 
When an io-bus network of one type is connected to another type, we have require a \textit{bridge}. 
Every device will be connected to some io-bus network and will be able to indirectly communicate with the processor. 

The cpu can access one or more io devices through one or more networks.
There is one network that the cpu is inside. The cpu has direct access to all the io devices 
inside this network. This network is known as the \textit{local bus}. 

\highlightdef{The \textbf{Local Bus} is the \textit{io-bus} the cpu has 
\textit{direct access} to}
where \textit{direct} means that no bridges are involved in communication. 
Communication is done through on network only. 

\section{CPU-Device Scheduling Problem}

\begin{figure}[h]
\begin{tikzpicture} [
   every node/.style={node_sty, align=center},
   edge from parent/.style={edge_sty},
   edge from parent path={(\tikzparentnode.south) 
   |- ($(\tikzparentnode.south)!0.5!(\tikzchildnode.north)$) -| (\tikzchildnode.north)},
   level 1/.style={sibling distance=2cm, level distance = 0.6cm},
   level 2/.style={sibling distance=3.5cm, level distance = 0.9cm}
]
   \node 
   (root) [text width=18em] {CPU-Device Scheduling Problem}
   child{
      node (sol) {Solutions}
      child {node (pio) {Programmed IO}}
      child {node(intio) {Interrupt-driven IO}}
      child {node(intio) [text width=9em] {IO Mediator}}
   };
   
\end{tikzpicture}
\end{figure}


\begin{itemize}   
\renewcommand{\labelitemi}{$\Box$}
\item \textbf{Programmed IO} 
The cpu pauses its fetch execute cycle to check the device controller’s control register port. 
If the register has a ready status then the cpu initiates a data transfer. 
\item \textbf{Interrupt-driven IO} 
Rather than the cpu checking the device controller's register on each cycle, it 
checks it's own specially reserved register called the \textit{interrupt flag register}. 
When a device is ready, it writes a value into the processors interrupt flag register.
At the end of a fetch-execute cycle, the processor will see that the device is ready 
by reading the \textit{interrupt flag register}. 
\item \textbf{IO Mediator} 
We will later see that adding more another processor or hardware controller dedicated 
to IO transfer can solve the problem. This processor/controller has the responsibility 
of being a mediator and passing messages from device to processor and vice versa. 
\end{itemize}

Programmed IO is also called \textit{polling}. Although this is slightly misleading because 
the processor does polling in some sense in Interrupt-driven IO as well. The processor 
has to poll the interrupt flag register. However interrupt-driven IO is much preferred.
Firstly, the processor only has to check \textit{one register}. 
It's own flag register. In programmed IO, it has to check the register 
of \textit{each device} to see if it's ready. 

Programmed IO results in poor utilisation of the CPU. 
We waste the cpu’s usage by busy waiting. We more cycles to checking control register than 
we would if we checked a register inside the cpu. 
When have multiple devices – devices must take turns communicating with the cpu. 
If one device takes too long then we can no longer guarantee good response times.

If we know our computer has only one device, then programmed-io may be a solution. A washing machine may for example use programmed-io. However, most desktop computers, use a large array of io-devices. And in general, most general-purpose computers require the luxury of having multiple devices. And so Programmed-IO isn’t seen in these situations. 


\section{Interrupt-Driven IO}

Processor designers realised the issues brought up with Programmed-IO.
And so processors are now designed to provide a calling mechanism called an Interrupt.


\highlightdef{An \textbf{Interrupt} is a calling 
mechanism provided by the processor that is given to devices }

An interrupt allows devices to call the processor’s attention to a 
change in the device controller’s registers.

When a device calls the processor, this interrupts the processor 
from its usual work of performing the fetch execute cycle. 
The cpu now performs the fetch-execute cycle for instructions that the device wants. 

How does the CPU know which interrupt handler to call?
How is an interrupt handler actually called?
What happens when we return from a handler?
How is data passed between the top and bottom halves of the device driver?
What happens if another interrupt is signalled while the interrupt handler is executing?




\section{OS and Drivers}



\begin{tikzpicture}[thick, data/.style={rectangle,fill=red!12}]
  \node[rectangle split,rectangle split parts=6,draw,anchor=center] (compiler) {
    \nodepart{one} cpu
    \nodepart{two} local bus
    \nodepart{three} io bus 1
    \nodepart{four} io bus 2
    \nodepart{five} ...
    \nodepart{six} io device
  };
\end{tikzpicture}


The operating system schedules io-requests in an attempt to maximise their throughput (based on the io- device’s characteristics).


\highlightdef{A \textbf{Device Driver} is software ...}


\highlightdef{The \textbf{Top Half} is responsible for ...}
\highlightdef{The \textbf{Bottom Half} is responsible for ...}


Synchronous I/O read() or write() will block a user process until its completion 
OS overlaps synchronous I/O with another process 
Asynchronous I/O read() or write() will not block a user process 
user process can do other things before I/O completion 
I/O completion will notify the user proces
