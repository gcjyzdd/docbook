

\chapter{IO Devices}


\section{Introducing IO Devices}

\highlightdef{An \textbf{IO Device} is an 
\textit{an additional piece of hardware} added to memory and cpu}

Perhaps the most important IO device is the \textit{hard disk}. 
First we will look at device controllers which are arguably the heart of the
communication between cpu and devices. 

\section{Device Controllers}

Every \textsc{io}-device has, inside, a \textit{device controller}.  
The device controller is simply \textit{more hardware}, 
but this hardware has a particular purpose. 
Device controllers are placed on an \textsc{io} device to give the cpu an interface for communicating
with the device. In other words, it's the \textsc{io}-device's way of making it open to be 
accessed or to be \textit{controlled} by the cpu.

\begin{figure}[h]
\begin{tikzpicture}[
  title/.style={},
  client/.style={rectangle, draw, minimum height=2em},
]
  \node [client, fill=white] (ctr) at (0, 10) {IO Controller};
  \node [title] (dev) at (0, 9.25) {IO Device};
  \begin{pgfonlayer}{background} 
      \node [draw=black, fill=black!10,drop shadow, inner sep=5pt, outer sep=5pt, fit={(dev) (ctr)}] {};
  \end{pgfonlayer}  
\end{tikzpicture}
\end{figure} 

The processor communicates with this controller but is unaware of how the device 
operates externally. To let the cpu communicate with the device, the \textsc{io}-device provides 
\textit{registers}. If the cpu wants to talk to the device, the cpu writes 
to registers inside the device controller. 

\begin{figure}[h]
\begin{tikzpicture}[
  title/.style={},
  client/.style={rectangle, draw, minimum height=2em},
]
  \node [client] (reg) at (0, 1.5) {Registers};
  \node [title] (ctr) at (0, 0.75) {IO Controller};
  \node [title] (dev) at (0, 0) {IO Device};
  
  \begin{pgfonlayer}{background}
      \node [draw=black, inner sep=5pt, outer sep=5pt, fill=white, fit={(ctr) (reg)}] (ctrfill) {};
  \end{pgfonlayer}
  \begin{pgfonlayer}{background1}
      \node [draw=black, fill=black!10,drop shadow, inner sep=5pt, outer sep=5pt, fit={(dev) (ctrfill)}] {};
  \end{pgfonlayer}
\end{tikzpicture}
\end{figure} 

Hopefully, if the device isn’t too busy, it will somehow know that the cpu said 
something and read its own device control registers. 
If the device wants to talk to the cpu, the device writes its device control registers 
and somehow, the cpu will know about this write and read the contents ... if the cpu isn’t too busy. 
\highlightdef{A \textbf{Device Controller} is hardware inside an io device that contains
\textit{registers} in order to \textit{provide communication with the cpu}.}
Device controllers are also known as \textit{io-controllers}, 
\textit{io-adapters} or \textit{io-modules}. 

Although the registers inside a device controller provide a shared storage between cpu 
and the device, there is a difficulty in making one side aware that the other has said 
something. We will refer to this problem as the \textit{CPU-Device Scheduling Problem}. 
Later we will look at schemes that attempt solve this problem.

Very often the device controller is a very small part of the actual hardware device 
(like a small chip or block stuck on the back). 
Sometimes the device controller is the majority of the hardware device 
(like a network card).


\section{CPU-IO Interface}

How do we make that interface available to programs

\begin{figure}[h]
\begin{tikzpicture} [
   every node/.style={node_sty, align=center},
   edge from parent/.style={edge_sty},
   edge from parent path={(\tikzparentnode.south) 
   |- ($(\tikzparentnode.south)!0.5!(\tikzchildnode.north)$) -| (\tikzchildnode.north)},
   level 1/.style={sibling distance=4cm, level distance = 0.9cm}
]
   \node (root) [text width=18em] {\textsc{cpu}-\textsc{io} Interface}
      child {node[text width=9em] {Memory Map}}
      child {node[text width=9em] {\textsc{io} Instructions}};
\end{tikzpicture}
\end{figure}


\begin{itemize}   
\renewcommand{\labelitemi}{$\Box$}
\item \textbf{Memory Map} 
The device controller’s registers appear to the processor as normal memory locations. 
Hence the cpu can easily manipulate the device controller’s registers by simply 
using instructions such as \textsc{mov}, \textsc{and}, \textsc{or}, \textsc{test}.
\item \textbf{IO Instructions} 
\textsc{io} registers have their own (very small) address space. 
The architecture must provide some method for accessing it. 
This is usually done by proving special \textsc{io} instructions in the processor's 
instruction set such as \textsc{in} and \textsc{out}. 
Each de
\end{itemize}

Intel x86 supports both memory maps and \textsc{io} instructions. 
It supports \textsc{io} instructions mapping by providing
64K 8-bit \textsc{io} ports numbered 0 to 65535


\highlightdef{The \textbf{IO Address Space} is allows the processor 
to choose which device for \textsc{io} Instructions}



\frmrule

\begin{example}
Below shows an example using \textsc{io} instructions in x86.

\begin{lstlisting}
in  ax, 10 ; copy bits from reg at address 10 into ax
out 20, ah ; copy bits from ah into reg at address 20
\end{lstlisting}
The first instruction copies 16 bits from the io controller register
that is mapped to io address 10 into the processor's register ax. 
The second instruction copies 8 bits from the processor's register al
into the io controller register that is mapped to io address 20. 

\end{example}

\frmrule

\begin{example}
Below shows an example of separate-memory mapping in IA-32.

\begin{lstlisting}
mov 30, ax ; copy bits from ax to reg at address 23
\end{lstlisting}
The first instruction copies 16 bits from the processor's register ax
into the io controller register that is mapped to physical address 30. 
This is a hardware mapping.
\end{example}

\frmrule

\begin{example}
Below shows another way of doing separate-memory mapping.
Here we use general purpose registers to tell the processor 
what address and value to use for io.
\begin{lstlisting}
mov dx, 40    ; Register address = 40
mov ax, VALUE ; Register value
out           ; Perform the io. Write to the device reg.
\end{lstlisting}
The first instruction copies the port into register dx. 
This tells the processor we want the register located at 
io memory address 40. The second instruction copies 
into ax the value we want to put into the register.
Now the processor has access all the information it needs, 
we perform the io with the out instruction. 
\end{example}


\section{Buses and Bridges}


Devices operating on buses can be divided into two categories, \textit{bus master} and \textit{bus slave}. 
Devices that are \textit{bus masters} are capable of initialising a \textit{bus transfer}
whereas \textit{bus slaves} are incapable of this. Because a bus can have many masters, there may 
be \textit{conflicts} over who has may initialise a bus transfer for a given time. 
A separate hardware circuit called a \textit{bus arbitrator} acts as an impartial 
and decides which master actually has control depending on the given circumstances. 

Different standards use different arbitration schemes. 
For example, \textsc{pci} uses a centralized arbitration scheme whereas 


\frmrule

\begin{example}
A master, $M$ wants to read data. It makes request to the bus arbitrator who grants $M$ access. 
Now $M$ has control it initialises a \textit{bus transfer}. $M$ puts an address on the bus 
along with a \textsc{read} assertion. The slave $S$ reads the address and retrieves the data. 
$S$ puts data onto the bus and $M$ reads this. 
\end{example}

\frmrule





\highlightdef{An \textbf{IO Bus} is communication medium for networked \textsc{io} devices.}

A computer usually has many io bus networks. However, there is a need for separating networks 
of \textsc{io} devices because devices operate \textit{at different speeds}. 
A bus however has a \textit{fixed clock rate}. 

Some examples of common standardised \textsc{io} bus networks are \textsc{isa}, \textsc{pci}, 
\textsc{scsi} (often pronounced: \textit{scuzzy}), \textsc{ide} (aka \textsc{ata}), \textsc{usb}, and \textit{Firewire}. 
When an \textsc{io} bus network of one type is connected to another type, we have require a \textit{bridge}. 
Every device will be connected to some \textsc{io} bus network and will be able to indirectly 
communicate with the processor.


\highlightdef{\textbf{Bus arbitration} avoids \textit{conflicting simultaneous use} 
of the bus by two masters}. 


\highlightdef{A potential \textbf{bottleneck} occurs when: bus speed $<$ device speed }

The \textsc{pci} bus is the successor to \textsc{isa} bus.



\highlightdef{Bus reads and writes are \textbf{transaction-based}}

The cpu can access one or more \textsc{io} devices through one or more networks.
There is one network that the cpu is inside. The cpu has direct access to all the \textsc{io} devices 
inside this network. This network is known as the \textit{local bus}. 

\highlightdef{The \textbf{Local Bus} is the \textsc{io} bus the \textit{cpu has direct access} to}

where \textit{direct} means that no bridges are involved in communication. 
Communication is done through on network only.

How does a device stay quiet?

\highlightdef{\textbf{Tristate}: a third state for when a bus-connected device is physically turned off}

This state doesn't maintain one nor does it maintain zero but rather, 
floats between the two binary states causing it to be, electrically, disconnected 
from the bus. On a bus that can have multiple bus masters, the address data bus must 
be driven by \textit{tristate outputs} and the data bus driven by \textit{tristate transceivers}.


\highlightdef{\textbf{Graphics cards} are usually connected to \textsc{pci} or \textsc{agp}}



\begin{example}
Modern Intel-x86 systems have a Northbridge/Southbridge divide. Why is this?
\end{example}


\section{CPU-Device Scheduling Problem}

\begin{figure}[h]
\begin{tikzpicture} [
   every node/.style={node_sty, align=center},
   edge from parent/.style={edge_sty},
   edge from parent path={(\tikzparentnode.south) 
   |- ($(\tikzparentnode.south)!0.5!(\tikzchildnode.north)$) -| (\tikzchildnode.north)},
   level 1/.style={sibling distance=2cm, level distance = 0.6cm},
   level 2/.style={sibling distance=3.5cm, level distance = 0.9cm}
]
   \node 
   (root) [text width=18em] {CPU-Device Scheduling Problem}
   child{
      node (sol) {Solutions}
      child {node (pio) {Programmed \textsc{io}}}
      child {node(intio) {Interrupt-driven \textsc{io}}}
      child {node(intio) [text width=9em] {\textsc{io} Mediator}}
   };
   
\end{tikzpicture}
\end{figure}


\begin{itemize}   
\renewcommand{\labelitemi}{$\Box$}
\item \textbf{Programmed IO} 
The cpu pauses its fetch execute cycle to check the device controller’s control register port. 
If the register has a ready status then the cpu initiates a data transfer. 
\item \textbf{Interrupt-driven IO} 
Rather than the cpu checking the device controller's register on each cycle, it 
checks it's own specially reserved register called the \textit{interrupt flag register}. 
When a device is ready, it writes a value into the processors interrupt flag register.
At the end of a fetch-execute cycle, the processor will see that the device is ready 
by reading the \textit{interrupt flag register}. 
\item \textbf{IO Mediator} 
We will later see that adding more another processor or hardware controller dedicated 
to \textsc{io} transfer can solve the problem. This processor/controller has the responsibility 
of being a mediator and passing messages from device to processor and vice versa. 
\end{itemize}

Programmed \textsc{io} is also called \textit{polling}. Although this is slightly misleading because 
the processor does polling in some sense in Interrupt-driven \textsc{io} as well. The processor 
has to poll the interrupt flag register. However interrupt-driven \textsc{io} is much preferred.
Firstly, the processor only has to check \textit{one register}. 
It's own flag register. In programmed \textsc{io}, it has to check the register 
of \textit{each device} to see if it's ready. 

Programmed \textsc{io} results in poor utilisation of the CPU. 
We waste the cpu’s usage by busy waiting. We more cycles to checking control register than 
we would if we checked a register inside the cpu. 
When have multiple devices – devices must take turns communicating with the cpu. 
If one device takes too long then we can no longer guarantee good response times.

If we know our computer has only one device, then programmed-io may be a solution. 
A washing machine may for example use programmed-io. However, most desktop computers, 
use a large array of io-devices. And in general, most general-purpose computers require 
the luxury of having multiple devices. And so Programmed-\textsc{io} isn’t seen in 
these situations. 


\section{Interrupt-Driven IO}

Processor designers realised the issues brought up with Programmed-\textsc{io}.
And so processors and low-level architectures are now designed with a
\textit{calling mechanism} called \textit{interrupts}.


\highlightdef{\textbf{Interrupts} are a \textit{callback 
mechanism} built into the processor's hardware}

An interrupt allows devices to call the processor’s attention to a 
change in the device controller’s registers.

When a device calls the processor, this interrupts the processor 
from its usual work of performing the fetch execute cycle. 
The cpu now performs the fetch-execute cycle for instructions that the device wants. 

How does the CPU know which interrupt handler to call?
How is an interrupt handler actually called?
What happens when we return from a handler?
How is data passed between the top and bottom halves of the device driver?
What happens if another interrupt is signalled while the interrupt handler is executing?

\begin{itemize}   
\renewcommand{\labelitemi}{$\Box$}
\item \textbf{1. Lookup} 
\item \textbf{2. Execution Context Switch}
\end{itemize}




\section{OS and Drivers}



\begin{tikzpicture}[thick, data/.style={rectangle,fill=red!12}]
  \node[rectangle split,rectangle split parts=6,draw,anchor=center, inner sep=8pt,fill=black!5,drop shadow] (path) {
    \nodepart{one} cpu
    \nodepart{two} local bus
    \nodepart{three} io bus 1
    \nodepart{four} io bus 2
    \nodepart{five} ...
    \nodepart{six} io device
  };
\end{tikzpicture}


The operating system schedules io-requests in an attempt to maximise their throughput (based on the \textsc{io} device’s characteristics).


\highlightdef{A \textbf{Device Driver} is software ...}

\highlightdef{Device Drivers are needed since \textsc{io} controllers vary.}

\highlightdef{The \textbf{Top Half} is responsible for ...}
\highlightdef{The \textbf{Bottom Half} is responsible for ...}


Synchronous \textsc{io} read() or write() will block a user process until its completion 
OS overlaps synchronous \textsc{io} with another process 
Asynchronous \textsc{io} read() or write() will not block a user process 
user process can do other things before \textsc{io} completion 
\textsc{io} completion will notify the user proces
