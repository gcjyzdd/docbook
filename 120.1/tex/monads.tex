

\chapter{Data Structures}


We careful and notice the difference between \lstinline{data} and \lstinline{type} declarations.
A type declaration defines a new name for an existing data type.
A data declaration defines a new data structure that consists of a collection of constructors for building that data type.



\chapter{Typeclasses}

We are familar with types. Types have been important for writing function 
signatures helping both us and the compiler describe what a function needs and what it gives.
Types are also useful in data structures where data declarations can build upon types. 	
This section will introduce two new structures that build on our knowledge of types, 
\textit{typeclasses} and \textit{typeinstances}. 

\highlightdef{A \textbf{typeclass} is similar to an \textit{interface} in Java. }

Be careful. Typeclasses are sometimes simply called \textit{classes}. 
You should definetely not compare typeclasses with 
classes from object-oriented programming. Instead, you should think of typeclasses 
as being similar to an \textit{interface with a generic type}.

A typeclass has the general structure:

\begin{lstlisting}
class T a where
   f1 :: [type of f1]
   f2 :: [type of f2]
   f3 :: [type of f3]
   ...
   fn :: [type of fn]
\end{lstlisting}

To put it simply, a typeclass is a \textit{collection of function signatures}.
First we begin the typeclass definition with the \lstinline{class} keyword
We write \lstinline{class} rather than \lstinline{typeclass} because it's shorter and easier.
Next, we declare that this typeclass is called \lstinline{T}. 
Then inside, we have a collection of function signatures. 

We are simply stating that \lstinline{f1}
has a certain signature. We only give it's names and it's type. 
There is \textit{no body} for \lstinline{f1}, it's \textit{just the signature}. 
Similarly for \lstinline{f2}, \lstinline{f3}, \lstinline{...}, we have given just their name 
and what their type is.

Finally, notice there is an \lstinline{a} just after \lstinline{class T}. 
This is the \textit{typeclass parameter}. The typeclass parameter may be used in any of the 
type definitions for the functions. We will shortly see why this is useful. 

Now that we have a typeclass \lstinline{T} defined, we can begin to define 
\textit{typeinstances} of \lstinline{T}. 

\highlightdef{An \textbf{typeinstance} is similar to a \textit{class} in Java.}

So, when we say \lstinline{M} is a typeinstance of \lstinline{T} 
we can compare this to how a class implements the methods of an interface. 
Below shows a typeinstance \lstinline{M} of typeclass \lstinline{T}.

\begin{lstlisting}
instance T M where
   f1 = [body of f1]
   f2 = [body of f2]
   f3 = [body of f3]
   ...
   fn = [body of fn]
\end{lstlisting}

As we can see, \lstinline{M} gives the body of the functions that are in. 
It must do this. There cannot be any unimplemented function bodies.   



Here is another example of a type class. 
\begin{lstlisting}
data Location = London | France | Tokyo
data Food = Pizza | Chocolate | Salad
data Drink = Water | Juice | Milk
class Alive a :: where
   walk :: Location -> String
   run :: Location -> String
   eat :: Food -> String
   drink :: Drink -> String
   count :: int -> String
\end{lstlisting}



\begin{itemize}	
\renewcommand{\labelitemi}{$\Box$}
\item \textbf{Definition} We first define a collection:
\lstinline{class Cool a}
(==)                  :: a -> a -> Bool
\item \textbf{Implementation} We first define a collection:
\lstinline{class Cool a}

\end{itemize}




\chapter{Monads}



\section{Pure and Impure Functions}

Functional programming languages divides into two: \textit{Pure Functional} programming languages 
and \textit{Impure Functional} programming languages. Haskell is an example of a \textit{pure} functional 
programming language. Standard ML and Clojure are examples of \textit{impure} functional programming languages.

Impure languages, such as Scheme and Standard ML, augment lambda calculus
with a number of possible eects, such as assignment, exceptions, or continu-
ations. Pure languages are easier to reason about and may benet from lazy
evaluation, while impure languages oer eciency benets and sometimes make
possible a more compact mode of expression.


\highlightdef{A \textbf{pure function} always returns the same result for a given argument}

All functions in Haskell are pure functions. There is no way to define an impure function.
However we can model how an impure function behave using pure functions. 
We think of an impure function as a function returning \textit{two values}. 
It returns a pure part $p$ and an impure part $i$. A given argument $x$ maps to some $p$ and $i$.
The pure part, $p$ is always the same value for the given $x$. However the $i$ can vary for a single $x$. 

\begin{lstlisting}
f :: x -> (p, i)
f = ...
\end{lstlisting}