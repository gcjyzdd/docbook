

\chapter{Data Structures}


We careful and notice the difference between \lstinline{data} and \lstinline{type} declarations.
A type declaration defines a new name for an existing data type.
A data declaration defines a new data structure that consists of a collection of constructors for building that data type.


\chapter{Typeclasses}

We are familar with types. Types have been important for writing function 
signatures helping both us and the compiler describe what a function needs and what it gives.
Types are also useful in data structures where data declarations can build upon types. 	
This section will introduce two new structures that build on our knowledge of types, 
\textit{typeclasses} and \textit{typeinstance declarations}. 

\highlightdef{A \textbf{typeclass} is similar to an \textit{interface} in Java. }

Be careful. Typeclasses are sometimes simply called \textit{classes}. 
You should definitely not compare typeclasses with 
classes from object-oriented programming. Instead, you should think of typeclasses 
as being similar to an \textit{Java interface with a generic type}. 
Interfaces are a bit like a todo list where we purposely don't say how exactly 
to do the things in the list - because they can be done in so many ways. 

A typical typeclass has the general structure:

\begin{lstlisting}
class T a where
   f1 :: [type of f1]
   f2 :: [type of f2]
   f3 :: [type of f3]
   ...
   fn :: [type of fn]
\end{lstlisting}

To put it simply, a typical typeclass is a \textit{collection of function signatures}.
First we begin the typeclass definition with the \lstinline{class} keyword.
We write \lstinline{class} rather than \lstinline{typeclass} because it's shorter and easier.
Next, we declare that this typeclass is called \lstinline{T}. 
For now, ignore the \lstinline{a} and look inside, we have the collection of function signatures. 

We are simply stating that \lstinline{f1}
has a certain signature. We only give its names and its type. 
There is \textit{no body} for \lstinline{f1}, it's \textit{just the signature}. 
Similarly for \lstinline{f2}, \lstinline{f3}, \lstinline{...}, we have given just their name 
and what their type is.

Finally, recall that there is an \lstinline{a} just after \lstinline{class T}. 
This is the \textit{typeclass parameter}. The typeclass parameter may be used in any of the 
type definitions for the functions. We will shortly see why this is useful. 

Now that we have a typeclass \lstinline{T} defined, we can begin to define 
\textit{typeinstances} of \lstinline{T}. 

\highlightdef{A \textbf{typeinstance declaration} is similar to a \textit{class} in Java.}

So, when we say \lstinline{M} is a typeinstance of \lstinline{T} 
we can compare this to how a class implements the methods of an interface. 
Below shows a typeinstance declaration that uses typeclass \lstinline{T}.

\begin{lstlisting}
instance T M where
   f1 = [body of f1]
   f2 = [body of f2]
   f3 = [body of f3]
   ...
   fn = [body of fn]
\end{lstlisting}

As we can see, we have declared the bodies of all the functions that are in 
the typeclass \lstinline{T}. All bodies must be there. 
There cannot be any unimplemented function bodies. This is similar to 
how a class that implements an interface must implement all the methods.

You mind think that \lstinline{M} is like a name of a new class being defined. 
Don't. It turns out that \lstinline{M} is \textit{not} the name of the typeinstance. 
Type instance declarations do not define a \text{new} name or identifer.
This typeinstance declaration is actually expecting 
an \textit{already defined type} \lstinline{M}.
When we write \lstinline{instance T M} we are saying
that \textit{given the signatures in M, these bodies tell you how them for type \lstinline{T}}. 

We are giving a concrete implementation on how to do  \lstinline{f1, f2, ...} 
for concrete T. Remember the typeclass gave us a vague collection 
of methods \lstinline{f1, f2, ...} on a general type \lstinline{a}. 
An instance declaration gives concrete bodies that work for concrete type T.

Also notice that a single typeclass declaration allows for many instance declarations. 
They are many ways to give bodies for the functions \lstinline{f1, f2, ...}. 
It depends on what our concrete type $N$ is and what bodies work for that type. 

\begin{lstlisting}
instance T N where
   f1 = [body of f1]
   f2 = [body of f2]
   f3 = [body of f3]
   ...
   fn = [body of fn]
\end{lstlisting}

This is why it was useful to define the \textit{typeclass parameter}
\lstinline{a} just after \lstinline{class T}. It forces us to write 
 code that pertains to a particular concrete type. 
This is an important part of how Haskell does code resuse and how Haskell 
does support programming to an interface rather than to an implementation. 




Here is another example of a type class. 
\begin{lstlisting}
data Location = London | France | Tokyo
data Food = Pizza | Chocolate | Salad
data Drink = Water | Juice | Milk
class Alive a :: where
   walk :: Location -> String
   run :: Location -> String
   eat :: Food -> String
   drink :: Drink -> String
   count :: int -> String
\end{lstlisting}






\chapter{Functors and Monads}

\section{Pure and Impure Functions}

Functional programming languages divide into two: \textit{Pure Functional} programming languages 
and \textit{Impure Functional} programming languages. Haskell is an example of a \textit{pure} functional programming language. 
Standard ML and Clojure are examples of \textit{impure} functional programming languages.

Impure languages are flexible and provide features such as assignment, exceptions, 
continuations at a lost cost to write. By low cost I mean that that it's easier to learn how to write code with these features or the features are provided in more carefree manner. 
This trades away ease of mathematically reasoning with the language.
Pure languages on the other hand are easier to reason about, benifit from lazy evaluation
but certain features are most costly to write.

\highlightdef{A \textbf{pure function} always returns the same result for a given argument}

All functions in Haskell are pure functions. This is what makes Haskell a pure language.
There is no way to define an impure function in Haskell.
However we can model how an impure function behave safely and purely in Haskell
using an idea of {wrapping values} in a data structure. 
We wrap arguments and outputs in what we call a data structure which we will call an
 \textit{context}. For now, think of this as a box.

A good design pattern for storing state is by carrying it around as an extra 
argument.
\highlightdef{We can model impurities by wrapping values in an \textit{context}}
How can wrapping values in boxes model impure functions? Let's see.



Below shows how an example of how we might like to wrap up a general value \lstinline{a}.
\begin{lstlisting}
data Box a = OK a | Empty
\end{lstlisting}

A \textit{boxed \lstinline{a}}, i.e. \lstinline{Box a}, is more 
expressive than just the value \lstinline{a} and we'll use it to model 
function application involving impure function.

\begin{lstlisting}
fnmap :: (Box -> Box) -> Box -> Box 
fnmap f (OK a) 
   | [condition1] = OK (f a)
   | [condition2] = Empty
fnmap f (Empty)
   | always = Empty
\end{lstlisting}


\begin{itemize}   
\renewcommand{\labelitemi}{$\Box$}
\item In the first case we have a value in the box.
We perform function application and the value changes inside the box to the output 
of the function. This is similar to how when impure function gives a conistent result.
\item In the second case,  we have a value in the box but
 the value can simply disappears from it's box. This is similar 
to how the output of an impure function gives a value that is inconsistent or nonsense. 
\item In the third case, we have a box where the value has 
already disappeared. Once the value has disappeared, it cannot come back. 
\end{itemize}



\section{Functors}



\section{Category Theory Introduction}


\highlightdef{A \textbf{Category} is a collection of objects and arrows between them...}
... where there are certain constraints on how arrows can be drawn.


\highlightdef{A \textbf{Monoid} is a category with \textit{one object}}
A monoid my have possibly many arrows or just one arrow. 

\begin{example}
$C = (\mathbb{N}, \square)$ is a monoid where the composition law is given by $n \square m = nm$.
Here, $Arr(C)$ is the set of natural numbers. 

\end{example}


\highlightdef{A \textbf{Functor} is a mapping from one category to another ...}
... that preserves the arrows in a particular way. 

the source and target of every arrow, the identity arrow of every node, 
and, the composition law.


