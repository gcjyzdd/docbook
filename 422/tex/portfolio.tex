
\chapter{Portfolio Theory I}







\section{Asset Returns}


\highlightdef{\textbf{Asset}: Investment instrument that can be bought/sold}



\section{Portfolios}

The word \textit{return} can refer to either the total return
or the rate of return. So whenever caution needed to ensure
which interpretation is meant. In these notes, 
it will be clear from the context.

$r = R - 1$ \\
$= \sum^{n}_{i = 1} w_i R_i  - 1$ \\
$= \sum^{n}_{i = 1} w_i R_i  - \sum^{n}_{i=1}w_i$ \\
$= \sum^{n}_{i = 1} w_i (R_i - 1)$ \\
$= \sum^{n}_{i = 1} w_i r_i $ \\

%\textit{total}: $R = \sum^{n}_{i = 1}w_i R_i$,

Let $\mathbf{r} = (r_1,r_2,...,r_n)$ be the vector of return rates
and $\mathbf{w} = (w_1,w_2,...,w_n)$ be the vector of weights. 
The $i$th component of these vectors gives the $i$th asset's return rate/weight. 


\highlightdef{
\textbf{Portfolios}:\\
\textit{Return Rate}:  $r_P = \bm{w}^T\bm{r}$ \\
\textit{Expected Return Rate}:  $\overline{r}_P = \bm{w}^T\bm{\overline{r}}$ \\
\textit{Variance}:  $\sigma_P = \bm{w}^T\bm{\Sigma}\bm{w}$
}



From now on, we will be working with \textit{return rates} (as apposed to total returns). 
The idea is that we will develop a model that can find 
optimal return rates. Once we have optimal return rates, $r^{*}_i$ 
we can simply find the optimal returns $R^{*}_i = 1 + r^{*}_i$.









\section{Portfolio Diagram}



\highlightdef{\textbf{Portfolio Diagram}: 
A plot ($\overline{r}, \sigma$) of return vs risk. 
\\Each coordinate corresponds to a particular portfolio.
}
Here portfolio equivalence is up to return and risk. 

Each point corresponds to a vector of weights. 


\highlightdef{
\textbf{Minimum Variance Set}: bullet \\
\textbf{Efficient Frontier}: curve
}


For two assets, the curve goes through both points. 

For three assets, 


\section{Risk-free and Short Selling}

If both borrowing and lending are allowed, an infinite
triangular region is obtained.

If only lending is allowed, the region will have a
triangular front end, but will curve for larger $\sigma$


\highlightdef{The covariance of the risk-free return $r_f$ with $r_i$ must be zero.}
That is $\text{cov}(r_f, r_i)$. 


\section{Markowitz Model I}




Given a target $\overline{r}$, we want to scan right until 
we hit the the efficient frontier. 




We differentiate the Lagrangian $L(\bm{w}, \lambda, \mu)$, with respect to $\bm{w}$.

$$\frac{\partial L}{\partial \bm{w}} = 
\frac{\partial}{\partial \bm{w}} \left(\frac{1}{2} \bm{w}^T\bm{\Sigma}\bm{w} 
- \lambda \left[\bm{w}^{T}\overline{\bm{r}} - \overline{r}_P\right]
- \mu \left[\bm{w}^{T}\bm{e} - 1 \right] \right)$$

$$= 
\left(\frac{1}{2} \cdot 2 \cdot \bm{\Sigma}\bm{w} 
- \lambda \left[\overline{\bm{r}} \right]
- \mu \left[\bm{e}\right] \right)
$$

All of the $\bm{w}^{T}$'s disappear. All constant's disappear. 
Also note that a factor of 2 appear when we differentiated $\bm{w}^T\bm{\Sigma}\bm{w}$. 
It is for this very reason that we added a $\frac{1}{2}$ in the original formulation of the problem.
Because we added the $\frac{1}{2}$ in front, it will nicely cancel with this 
factor of 2. With this, optimality condition of the Lagrangian can be neatly expressed as:

$$
\bm{\Sigma}\bm{w} - \lambda \overline{\bm{r}} - \mu \bm{e} = \bm{0}
$$

Overall, our optimality conditions are: 
(i) $\bm{\Sigma}\bm{w} - \lambda \overline{\bm{r}} - \mu \bm{e} = \bm{0}$
(ii) $\bm{r}^{\top}\bm{w} = \overline{r}_P$
(iii) $\bm{e}^{\top}\bm{w} = 1$

We can express this conditions in a single vector/matrix equation.
We have a $(n+2) \times (n+2)$ matrix multiplying a $(n+2)$-component column vector 
to give another $(n+2)$-component column vector. 

The solution requires us to invert the matrix. 
This matrix is invertible only when $\bm{\Sigma}$ has \textit{full rank}
and when $\overline{\bm{r}}$ is \textit{not a multiple} of $\overline{\bm{e}}$. 
The final solution is a linear program that can be solved using an algorithm like
\textit{Gaussian elimination}. 

\highlightdef{We \textit{cannot} find the Markowitz solution for portfolios with risk-free
assets, the covariance matrix \textit{does not} have full rank. }

In the next section, we will look at the Two-fund theorem that will 
help us deal with risk-free assets.


\frmrule

\begin{example}

\end{example}

\frmrule

So far, we looked at the solving the Markowitz Model \textit{with short-selling}. 
The solution was a linear program. However, if we now remove short-selling, 
the solution cannot be reduced to a set of linear equations. 
Instead we have a \textit{quadratic program}. The solution 
can be found via computer programs (e.g. quadprog in Matlab). 

The difference is that we added the constraint $\bm{w} \geqslant \bm{0}$. 

\frmrule

\begin{example}
What if we choose an $\overline{r}$ that's below the bullet and 
obtain the Markowitz solution.  
\end{example}

\frmrule

Investors seeking efficient portfolios need only invest in
combinations of two efficient funds.

Under the assumptions of the mean-variance model,
there is no need for anyone to buy individual stocks.


\section{Markowitz Model II}


\highlightdef{\textbf{Affine Function}: $f(x) = \bm{s}{x} + \bm{t}$ }

The vector is taking linear combinations of 
the columns of the inverse matrix. We don't know what 
these columns are, but the point is that such a linear combiation 
will be affine function of $\overline{r}_P$. 

$\bm{A}^{-1} \bm{x} = 0\bm{A} - \overline{r}_P \bm{b} - \bm{c} = -\bm{b}\overline{r}_P - \bm{c}$

This affine property of the Markowitz solution allows us to easily 
prove an important theorem called the \textit{Two-fund Theorem}. 


Let two portfolios $P'$ and $P''$ with Markowitz solutions $f(\overline{r}_{P}')$ and $f(\overline{r}_{P}')$,
we have that for a portfolio, $P$ that is combination $P'$ and $P''$, 
$\overline{r}_{P} = \alpha \overline{r}_{P}' + (1 - \alpha) \overline{r}_{P}''$,
then optimal solution can be found easily by:

\highlightdef{\textbf{Two-fund Theorem}: $f(\overline{r}_{P}) = \alpha f(\overline{r}_{P}') + (1 - \alpha) f(\overline{r}_{P}'')$ }

It is simply the linear combination of the already given Markowitz solutions.
We use the weighting of portfolios for the weighting of the Markowitz solutions. 

To prove the theorem, we only see that: 
$f(\overline{r}_{P}) = \alpha f(\overline{r}_{P}') + (1 - \alpha) f(\overline{r}_{P}'')$

\[
\begin{array}{ll}
f(\alpha r' + (1 - \alpha)r'') 
&= -\bm{b} (\alpha r' + (1-\alpha)r'') - \bm{c} \\
&=  \alpha (- \bm{b} r' - \bm{c}) + (1 - \alpha)(-\bm{b}r'' - \bm{c}) - \bm{c} + \alpha \bm{c} + (1-\alpha)\bm{c} \\
&=  \alpha (- \bm{b} r' - \bm{c}) + (1 - \alpha)(-\bm{b}r'' - \bm{c}) - (1-\alpha)\bm{c} + (1-\alpha)\bm{c}  \\
&=  \alpha (- \bm{b} r' - \bm{c}) + (1 - \alpha)(-\bm{b}r'' - \bm{c})  \\
&=  \alpha f(r') + (1 - \alpha)f(r'')  \\
\end{array}
\]


Notice that it is necessary for the weights to sum to 1. 
The result doesn't hold for any linear combination of portofolios. 
To see this, let's repeat the but now, we are take a 
\textit{general combination}. 


\[
\begin{array}{ll}
f(\alpha r' + \beta r'') 
&= -\bm{b} (\alpha r' + \beta r'') - \bm{c} \\
&=  \alpha (- \bm{b} r' - \bm{c}) + \beta(-\bm{b}r'' - \bm{c}) - \bm{c} + \alpha \bm{c} + \beta\bm{c} \\
&=  \alpha (- \bm{b} r' - \bm{c}) + \beta(-\bm{b}r'' - \bm{c}) - \bm{c} ( 1- [\alpha + \beta])  \\
&=  \alpha (- \bm{b} r' - \bm{c}) + \beta(-\bm{b}r'' - \bm{c}) \text{ exactly when: } \alpha+\beta = 1 \\
&=  \alpha f(r') + \beta f(r'') \text{ exactly when: } \alpha+\beta = 1 \\
\end{array}
\]

\frmrule

Recall that we could not solve the Markowitz problem 
for assets that were risk-free. The covariance matrix has 
a column of zeros, implying has linearly dependent columns, 
and so we cannot invert the matrix to find the solution. 

With the establishment of the two-fund theorem, 
we can now consider portfolios that have risk-free assets. 
The idea is that we'll first form a portfolio using weight $\alpha$ 
for the risk-free asset and $1-\alpha$ for the risky asset. 

This portfolio has a return-risk given by:
$$\overline{r}_P = \bm{w}^{T}\overline{\bm{r}} = \alpha r_f + (1-\alpha)\overline{r}$$
$$\sigma^{2}_P = \bm{w}^{T} \bm{\Sigma}\bm{w} = (1-\alpha)^{2}\sigma^{2}$$





\section{Revision: Statistical Estimators}


