
\chapter{Fixed-Income Securities I}



\section{Introducing Bonds and Securities}


You can lend money to the US government and in return they give you an IOU and 
promise to pay you back the money, along with interest at some future date. 

\begin{figure}[h]
\begin{tikzpicture} [
  edge from parent/.style={edge_sty},
  edge from parent path={(\tikzparentnode.south) -- ++(0, -0.3cm) -| (\tikzchildnode.north)},
  level/.style = {level distance = 0.9cm, every node/.style={draw, node_sty2}},
  level 1/.style={sibling distance=3cm, text width = 1.5cm},
  level 2/.style={sibling distance=2cm}
]
  \node (root) [node_sty2] {US Treasury}
    child {node(bills) {bills}}
    child {node(notes) {notes}}
    child {node(bonds) {bonds}};
  \node[below=0.2cm of bills, align=center] {\textit{$<$ 1 year}};
  \node[below=0.2cm of notes, align=center] {\textit{1 to 10 years}};
  \node[below=0.2cm of bonds, align=center] {\textit{$>$ 10 years}};
\end{tikzpicture}
\end{figure}


\begin{itemize}
\item \textbf{Face Value} $F$, sometimes called \textit{par value}
\item \textbf{Subannual periods}: pay $m$ times per year 
\item \textbf{Subannual Payments}: given annual rate $r$, after $m$ subannual periods, 
we have paid $rF$. We call $C = rF$ the \textit{yearly coupon amount}. 
We pay $C/m = rF/m$ at every subannual period. 
\item \textbf{Maturity}: $n$ how many years we are making payments
\end{itemize}

Given face value $F$, rate $r$, maturity $n$, and subannual periods $m$,
\highlightdef{\textbf{Bond}: pay $Fr/m$, $m$ times per year for $n$ years}

\section{Spot Rates}


So far we have only been dealing with a single 
interest rate, $r$. This has been constant over time. 
But what is our interest rate varies over time?
We can model a cash flow as having a whole 
family of interest rates at any point in time. 
These interest rates we call \textit{spot rates}.  

\highlightdef{
  \textbf{Spot Rate of Interest}: The \textit{spot rate} is the 
  annualized interest rate charged for 
  borrowing money from the present until time $t$.
}

By considering different maturity times, $t$, that is, 
looking at holding money now up to time $t$, we derive 
different interest rates. These are the \textit{spot rates}. 

Both the interest and the principle (original amount borrowed) 
are paid at time $t$. The spot rate is denoted $s_t$. 
For example, $s_1$ is the 1-year interst rate - the rate paid for borrowing money for one years. 
$s_2$ is the rate paid for borrowing money for two years. 

However spot rates are expressed on an \textit{annualized} basis.
So for example, if you make an agreement to pay a rate of $s_2$ 
for a two-year deposit of amount $A$ compounded yearly, it 
will actually repay $(1 + s_2)^2 \cdot A$ at the end of two years. 
Your money grows by factor $(1 + s_2)^2$, not $(1 + s_2)$. 

Spot rates are always yearly rate. We may have done a recalculation from some other periodicity 
but, overall, $s_t$ denotes the $t$th \textit{year}. If you mount $A$ compounded yearly, 
you get a return of $(1 + s_t)^t \cdot A$ (not $(1 + s_t) \cdot A$). 

\begin{figure}[h]
\begin{tikzpicture}
\matrix [matrix of nodes, column sep=5mm, row sep=0cm]
{
  \textbf{Yearly} & \textbf{Periods} & \textbf{Continuous} \\
  $(1+ s_t)^t$ & $d_t = (1 + s_t/m)^{mt}$ & $e^{s_t t}$ \\
};
\end{tikzpicture}
\end{figure}


Given any cash flow stream $[x_0, x_1, x_2, ..., x_n]$, we can compute the 
present value and future value using $s_t$
and $d_t$, the corresponding discount factor ($= 1/s_t$). 

\highlightdef{
  \textbf{Future Value}: $\text{FV} = x_0 s_0 + x_1 s_1 + x_2 s_2 + ... + x_n s_n$
}

\frmrule 

\begin{example}
Consider: \textit{an 8\% bond with a face value of \pounds100 maturing in 10 years}. \\
(a) Give the cash flow stream. \\
(b) For the spot rates, $s_t$, given below, compute the present value, $\text{PV}$.
 $$[5.571, 6.088, 6.555, 6.978, 7.361, 7.707, 8.020, 8.304, 8.561, 8.793]$$
\end{example}



\frmrule 

Rather than having the spot rates given to us in 
the form $[s_0, s_1, s_2, ..., s_n]$, it is quite useful to 
have them presented in \textit{graphical form}. This 
graph is called a \textit{spot rate curve}.

\highlightdef{\textbf{Spot-rate Curve}: A graph with $y$-axis: $s(t)$, \textit{spot rate}, 
and $x$-axis: $t$, \textit{time}. }

This is often called the \textit{yield curve}. 
Spot rate curves are usually \textit{upwards sloped}. 
In other words, if you lend your money for longer periods of time, 
then a higher interest rate is paid. Intuitively, if you lend money 
for a longer period of time, then there is a higher risk. So a higher 
return is expected. Conversely, shorter periods of time have a lower 
risk. Only so much can happen in a month in comparison to several years. 
With less risk comes a lower interest rate. 

\frmrule 

\begin{example}
A typical spot-rate curve has its $x$-axis starting from $t = 1$. \\
Why don't we plot the $x$-axis starting from $t = 0$, rather than from $t = 1$? 

\frameans{Hint: Will there be anything interesting about $s_0$?}{$s_0 = 1$, \textit{always}. No point plotting.}
\end{example}

\frmrule 

For the given three given modes of computing spot rates, we can define 
the corresponding discount factors, $d_t$. 

\begin{figure}[h]
\begin{tikzpicture}
\matrix [matrix of nodes, column sep=5mm, row sep=0cm]
{
  \textbf{Yearly} & \textbf{Periods} & \textbf{Continuous} \\
  $d = \frac{1}{(1+ s_t)^t}$ & $d_t = \frac{1}{(1 + s_t/m)^{mt}}$ & $d_t = e^{-s_t t}$ \\
};
\end{tikzpicture}
\end{figure}

Where for yearly, $t \in \mathbb{N}$, for periods we have $t \in \{0,1,..m\}$ 
and for continuous, $t \in \mathbb{R}$. 

\highlightdef{
  \textbf{Present Value}: $\text{PV} = x_0 d_0 + x_1 d_1 + x_2 d_2 + ... + x_n d_n$
}


$d_t$ acts like a price for cash received at time $t$. And so 
the present value is summing up the quantities, $x_t$ multiplied by 
the prices, $d_t$ for having that quantity. This summation is over 
all times $t = 0....n$, that is - a summation from now, up to 
the relevant future time $n$. 



\section{Forward Rates}

\highlightdef{
  \textbf{Forward Rate}: The \textit{forward rate}, $f_{t_1, t_2}$, is an 
  interest rate for money to be borrowed between two dates \textit{in the future}, 
  $t_1$ and $t_2$ ($t_1 < t_2$).  
}
where the terms of agreement are made \textit{today}. 
It is the rate charged for \textit{borrowing money at} $t_1$ which is to be repaid 
with interest at $t_2$. Furthermore, this rate of interest, $f_{t_1, t_2}$ is 
agreed upon \textit{in advanced} at $t = 0$ (today). 


\frmrule

\begin{example}
Suppose that we know the spot rates $s_1$ and $s_2$. That is, 
we know how much interest to charge to the borrowing of money 
now (on the spot) to be repaid at time $t = 1$, 
and the borrowing of money now (on the spot) to be repaid at time $t = 2$. 
\begin{itemize}
\item We leave $\pounds 1$ in a two year account. By the definition of spot rates, 
the growth factor for a deposit made after $t$ years is $(1 + s_t)^t$.
So for two years, our  $\pounds 1$ grows to $\pounds 1 \times (1 + s_2)^2$. 
\item Alternatively, we make two simulateously agreements. The first agreement 
is on putting $\pounds 1$ in an account for just one year (rather than two). 
The second agreement is 
that money will be borrowed between two dates \textit{in the future},
for $t = 1$ and $t = 2$. How much interest do we charge? This is precisely the \textit{forward rate}, 
$f_{1,2}$. This agreement is made \textit{now}, and it details the borrowing of money 
between \textit{two future dates}.

How much is borrowed? Well suppose we lend the contents of our account 
at the end of year 1. Recall that we put  $\pounds 1$ in for one year.
By the definition of spot rates, 
the growth factor for a deposit made after $t$ years is $(1 + s_t)^t$.
So at $t = 1$, our  $\pounds 1$ will have grown to $\pounds 1 \times (1 + s_1)^1$.
Let's lend this amount for the time between $t = 1$ to $t = 2$. We are repaid 
this amount at time $t = 2$ plus an interest given by $f_{1,2}$. In other words, we 
have a total of $\pounds 1 \times (1 + s_1)^1 \times (1 + f_{1,2})$. 
\end{itemize}
\end{example}

We claim that these two options \textit{are the same investment}. That is, 
after two years, both options give the same return on investment. Hence:
$$(1+s_2)^2 = (1 +s_1)^{1}(1 + f_{1,2})$$

Why should these two options give an equal return? This is an application 
of the \textit{comparison principle} and so our justification requires 
an \textit{arbitrage argument}. This is like a proof by contradiction.

Assume the two options \textit{did not} give the same return on investment. 
Then either $(1+s_2)^2 > (1 +s_1)^{1}(1 + f_{1,2})$ or $(1+s_2)^2 < (1 +s_1)^{1}(1 + f_{1,2})$
\begin{itemize}
\item $(1+s_2)^2 > (1 +s_1)^{1}(1 + f_{1,2})$
\item $(1+s_2)^2 < (1 +s_1)^{1}(1 + f_{1,2})$. In this case, the second investment 
is better. An arbitrageur could also do option two and \textit{lend} money to somebody else. 
But, he could also \textit{borrow} money. Somebody else will be willing to lend under option 1, 
and so the arbitrageur will borrow money from this person. \\
\begin{itemize}
\item $t = 0$ 

Arbitrageur, borrows from somebody, $X_1$, offering same agreement terms as option 1 
(the arbitrageur is on the \textit{borrowing} side, rather than the on the \textit{lending} side).\\ 
$X_1$: $X_1$ deposits $\pounds 1$ in the arbitrageur's account at $t = 0$. 
and arbitrageur agrees to pay $\pounds 1 \times (1 + s_2)$ at $t = 2$.

\frmrule 

Immediately after this agreement has been made (still at $t = 0$), the arbitrageur carries out option 2. 

$X_2$: Lend $\pounds 1$ to $X_2$ now and $X_2$ will repay us a total of $(1 + s_1)^1$ at $t = 1$. 

$X_3$: Lend to $X_3$ for time $t = 1$ to $t = 2$, the amount returned from lending the $\pounds 1$
to $X_2$ be repaid at $t = 2$ with interest rate $f_{1,2}$.  That is, we agree now to lend $\pounds (1 + s_1)$
to $X_3$ and $X_3$ will pay us $\pounds (1 + s_1)(1 + f_{1,2})$. 

\frmrule \\
\item $t = 1$ \\
$X_2$ pays arbitrageur $\pounds (1+s_1)^1$. \\
Arbitrageur pays $X_3$ $\pounds (1+s_1)^1$. \\
Total cash is $0$. \\
\frmrule 
\item 
$t = 2$ \\
$X_3$ pays arbitrageur $\pounds (1 + s_1)(1 + f_{1,2})$. \\
Arbitrageur pays $X_1$ $\pounds (1+s_2)^2$. \\
Total profit of $(1 + s_1)(1 + f_{1,2}) - (1+s_2)^2 > 0$. 
\end{itemize}
\end{itemize}

\frmrule

This arbitrage scheme could be carried out with at any magnitude. 
At $t = 0$, we could have several triples of people $(X_1, X_2, X_3)$, say
$(X_1',X_2',X_3'),  (X_1'',X_2'',X_3''),  ...$ and made the same agreements with 
those triples. 
If we had $k$ triples, we have $k$ times the profit. So, assuming 
a countless number of people, we would in theory, 
could make a countless amount of money - all from no initial capital. 
So it must be \textit{impossible to implement the scheme in the market}. And so 
we have a contradiction. If were possible - 
a vast number of arbitrageurs would take advantage of it and it would affect the 
market so greatly that it would make the scheme no longer possible 
(gaps in rates would close). 

\frmrule



\begin{example}
Suppose that the spot rates for years 1 and 2 are $s_1 = 7\%$ and $s_2 = 8\%$. 
What must the forward rate, $f_{1,2}$ be?

Firstly, the comparison principle tells us that $(1 + s_2)^2 = (1 + s_1)^1(1 + f_{1,2})$. \\
So:
$$f_{1,2} =  \frac{(1 + s_2)^2}{(1 + s_1)^1} - 1$$

Hence $f_{1,2} = ............$

\frameans{The forward rate is ....}{$9.01\%$}

\end{example}


\frmrule

Let us generalise for any two future times $t = i$ and $t = j$, where $i < j$.

\begin{example}
Suppose that we know the spot rates $s_i$ and $s_j$. That is, 
we know how much interest to charge to the borrowing of money 
now to be repaid at time $t = i$, and the borrowing of money now to be repaid at time $t = j$. 
\begin{itemize}
\item \textit{Option 1:} We leave $\pounds 1$ in an $j$ year account. $\pounds 1$ grows to $\pounds 1 \times (1 + s_j)^j$. 
\item \textit{Option 2:} Make two simultaneous agreements. First agreement:
put $\pounds 1$ in an account for just $i$ years (rather than $j$). 
At $t = i$, we will be paid $\pounds (1 + s_i)^i$.
Second agreement: lend $\pounds (1 + s_i)^i$
at $t = i$ and  at $t = j$ be repaid $\pounds (1 + s_i)^i (1 + f_{i,j})^{j-i}$. 
\end{itemize}
\end{example}

By the comparison principle, $$(1 + s_j)^j = (1 + s_i)^i (1 + f_{i,j})^{j-i}$$
Hence

$$f_{i,j} =  \left[\frac{(1 + s_j)^j}{(1 + s_i)^i} \right]^{1/(j-i)} - 1$$


\frmrule





\section{Expectation Dynamics I}




\highlightdef{
  \textbf{Predicting the Future I}: Spot rates can be calculate 
  from forward rates.
}


The claim is that next applicable forward rate (i.e. one that starts 
at time $t = 1$ and goes to some time $t = j > 1$), $f_{1,j}$ is 
a very good prediction of next year's spot rate of borrowing 
on the spot (next year) for $j-1$ years. 

\highlightdef{
  \textbf{Unbiased Expectations Hypothesis}:
  $f_{1,j} = E[s'_{j-1}]$
}
where $s'_{j-1}$ is next year's spot rate. 
Note that times $(1,j)$ now become times $(0,j-1)$ near year. 
This explains why there is a $j-1$ underneath $s'$ (it is \textit{not} $s_j$).



According to the \textit{unbiased expectations hypothesis}, 
the forward rate $f_{1,j}$ is \textit{exactly} equal to 
the market expectation of what the $j-1$-year spot rate will be next year. 
The expectation of what an not yet known value can be inferred 
from already known existing values. 

\frmrule

\begin{example}
Earlier we considered a situation where $s_1 = 7\%$, $s_2 = 8\%$. 
We found that the implied forward rate was $f_{1,2} = 9.01\%$. 
According to the \textit{unbiased expectations hypothesis}, 
this value of $9.01\%$ is the markets's expected value of next year's 
1-year spot rate $s_1'$. 
\end{example}

\frmrule

In general, the current spot rate curve, $[s_1, s_2, ..., s_n]$, by 
the comparison principle (CP),
leads of a set of implicit forward rates $f_{1,2}, f_{1,3}, ..., f_{1,n}$ 
which, by the \textit{unbiased expectations hypothesis} (EH),
define the expected spot rate curve for next year $[s_1', s_2', ..., s_{n-1}']$.


\begin{figure}[h]
\begin{tikzpicture}
\node (s1) {$[s_1, s_2, ..., s_n]$};
\node [right = 2cm of s1] (f) {$f_{1,2}, f_{1,3}, ..., f_{1,n}$};
\node [right = 2cm of f] (s2) {$[s_1', s_2', ..., s_{n-1}']$};
\draw[<->, shorten <=10pt, shorten >=10pt] (s1) -- node[above]{CP} (f);
\draw[<->, shorten <=10pt, shorten >=10pt] (f) -- node[above]{EH} (s2);
\end{tikzpicture}
\end{figure}

This argument flows in the reverse direction too. 
That is, if we are given an expected spot rate curve, $[s_1', s_2', ..., s_{n-1}']$, 
then this determines the forward rates $f_{1,2}, f_{1,3}, ..., f_{1,n}$
which in turn, determine the current year's spot rate curve, $[s_1, s_2, ..., s_n]$.
So overall, the current spot rate curve implies an expectation of what the 
spot rate curve will be next year. And, (turning the argument around) the 
expectation of next year's spot rate curve determines what this year's curve is.
So there is a certain intertwining between today's spot rate curve 
and the expectations of next year's spot rate curve. 

\frmrule

\begin{figure}[h]
\begin{tikzpicture}
\matrix [matrix of math nodes, column sep=5mm, row sep=0cm]
{
  f_0,1 & f_2 & f_1 \\
  f_0,1 & f_2 & f_1 \\
  f_1 & f_1 & f_1 \\
};
\end{tikzpicture}
\end{figure}



\section{Expectation Dynamics II}

What we have been doing is calculating interest rates for future times 
from already known spot rates. In some ways, this is like predicting 
the future. 

\highlightdef{
  \textbf{Predicting the Future I}: Forward rates can be calculated 
  from spot rates.
}

\frmrule

Unknown forward rates calculated from known spot rates. 
These forward rates are called \textit{implied forward rates}.
They are merely an \textit{implicit} value that can be derived from spot rate curves. 
They are not a perfect representation of what happens in the future.
That is, they are subject to \textit{market imperfections}. 


\highlightdef{
  \textbf{Implicit Forward Rate}: The \textit{Implicit forward rate} 
  between times $t_1$ and $t_2$ (where $t_1 < t_2$) is the rate of interest 
  between those two times that is \textit{consistent with a given spot rate curve}.
}
As will all forward rates, the terms of agreement are made \textit{today}. 
It is \textit{today}'s forward rate between $t_1$ and $t_2$. 
It is a forcast of the future spot rate between $t_1$ and $t_2$. 
We don't know what the spot rate is going to be. Future interest 
rates as unknown. But implicit 
in today's prices is a forcast of this spot rate and we call 
this forcast the \textit{implicit forward rate}. It captures 
the market's best guess at what the future spot rate will be.

\frmrule

\begin{example}
Assume the spot rates for the current year are:
$$[6.00, 6.45, 6.80, 7.10, 7.36, 7.56, 7.77]$$
\end{example}

\frmrule


\section{Short Rates}


\highlightdef{\textbf{Short Rate}: A \textit{short rate} is a forward rate spanning one year}

$$(1 + f_{i,j})^{j-i} = (1 + r_i)(1 + r_{i+1})...(1 + r_{j-1})$$

$$1 + r_t = \frac{(1 + f_{0,t})^{t}}{(1 + f_{0,t-1})^{t-1}}$$


% To summarise:

% \begin{itemize}
% \item \textbf{Spot Rate} $s_t$ (already known)
% \item \textbf{Future Spot Rate} (don't know) $s_t'$
% Sometimes called the \textit{Future Rate} or the \textit{Future Expected Spot Rate}.
% \item \textbf{Forward Rate} $f_{i,j}$ (already known)
% \item \textbf{Implicit Forward Rate} (don't know) $f_{i,j}$ (calculated from $s_i$ and $s_j$)
% \end{itemize}


