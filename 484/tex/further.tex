
\chapter{Further Topics}


\section{Quantum Cryptography}


\section{Quantum Error Correction}



\highlightdef{\textbf{3-Qubit Bit-Flip Code}: }

We need to keep in mind that we have to measure to determine if there was an error.






 \begin{tikzpicture}[thick]
    % `operator' will only be used by Hadamard (H) gates here.
    % `phase' is used for controlled phase gates (dots).
    % `surround' is used for the background box.
    \tikzstyle{operator} = [draw,fill=white,minimum size=1.5em] 
    \tikzstyle{phase} = [draw,fill,shape=circle,minimum size=5pt,inner sep=0pt]
    \tikzstyle{control} = [draw,shape=circle,minimum size=5pt,inner sep=0pt]
    \tikzstyle{surround} = [fill=blue!10,thick,draw=black,rounded corners=2mm]
    %
    \matrix[row sep=0.4cm, column sep=0.8cm] (circuit) {
    % First row.
    \node (q1) {\ket{\phi}}; &
    \node[phase] (p1) {}; &
    \node[phase] (p2) {}; &
    \coordinate (end1); \\
    % Second row.
    %
    \node (q2) {\ket{0}}; &
    \node[control] (c1) {}; &
    &
    \coordinate (end2);\\
    % Third row.
    %
    \node (q3) {\ket{0}}; &
    &
    \node[control] (c2) {}; &
    \coordinate (end3); \\
    };
    % Draw bracket on right with resultant state.
    \draw[decorate,decoration={brace},thick]
        ($(circuit.north east)-(0cm,0.3cm)$)
        to node[midway,right] (bracket) {$\ket{\phi'}$}
        ($(circuit.south east)+(0cm,0.3cm)$);
    \begin{pgfonlayer}{background}
        % Draw lines.
        \draw[thick] (q1) -- (end1)  (q2) -- (end2) (q3) -- (end3) (p1) -- (c1) (p2) -- (c2);
        % Draw control vertical lines.
        \draw[thick] (c1.north) -- (c1.south);
        \draw[thick] (c2.north) -- (c2.south);


    \end{pgfonlayer}
    %
\end{tikzpicture}




The initial state $\ket{\phi} = \alpha \ket{0} + \beta \ket{1}$ is converted by
the circuit into $\ket{\phi'} = \alpha \ket{000} + \beta \ket{111}$.
Then each qubit of $\ket{000}$ is sent independently through the bit flip channel.
Each qubit of $\ket{111}$ is sent independently through the bit flip channel. 
This gives us $\ket{\phi''}$.

We then apply a two-stage error correction method. 
We first make a measurement using the four measurement
operators. Each measurement operator is gives a basis for the errors. 

These operators are called \textit{error syndromes}.
Depending on which error syndrome is measured we then recover the original
state.

$M_0 = \ket{000}\bra{000} + \ket{111}\bra{111}$ \\
$M_1 = \ket{100}\bra{100} + \ket{011}\bra{011}$ \\
$M_2 = \ket{010}\bra{010} + \ket{101}\bra{101}$ \\
$M_3 = \ket{001}\bra{001} + \ket{110}\bra{110}$ \\



\begin{example}
Suppose, for example, that a bit flip occurred, on the second qubit.
Then $\ket{\phi} = \alpha \ket{0} + \beta \ket{1}$ is converted by
the circuit into $\ket{\phi'} = \alpha \ket{000} + \beta \ket{111}$.
Then $\ket{\phi'} = \alpha \ket{000} + \beta \ket{111}$ is converted to 
$\ket{\phi''} = \alpha \ket{010} + \beta \ket{101}$. 

Now we determine $p(2) = \bra{\phi''} M^{H}_{2} M_{2} \ket{\phi''} = ...$

\end{example}

\frameans{}{$1$}

So it is \textit{certain} that $j = 2$. 

The state of the system will not change $M_2 \ket{\phi''} = \ket{\phi''}$
Therefore, we finally flip the second qubit two to recover $\ket{\phi'}$.



\begin{example}
Show that these four measurement operators 
satisfy the \textit{General Measurement Principle}: $\sum^{3}_{j=0} M^{H}_{j} M_{j} = I$
\end{example}


\section{Teleportation Quantum Computation}


\section{One-way Quantum Computation}

\section{Measurement Calculus}