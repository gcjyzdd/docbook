
\chapter{Quantum Algorithms I}


\section{Parallel Hadamard Gates}


\highlightdef{$H\ket{x} = \frac{1}{\sqrt{2}} \left[(-1)^{x \wedge 0} \ket{0} + (-1)^{x \wedge 1} \ket{1} \right]$  }

This may seem like a useless way to write the Hadamard operation but it was done in preparation 
for understanding how \textit{parallel} Hadamard gates operate. 



\highlightdef{$[x_0 x_1 \dotsm x_n, \; y_0 y_1 \dotsm y_n] = 
(x_0 \wedge y_0) \oplus (x_1 \wedge y_1) \oplus \dotsm \oplus (x_n \wedge y_n)$}

\frmrule

\begin{example}
Determine $[\textsf{0011}, \; \textsf{0101}]$ 
\end{example}

\frameans{}{$0$}


\frmrule

\begin{example}
Show that $(-1)^{u \wedge v}\ket{i} \otimes (-1)^{x \wedge y}\ket{j} = (-1)^{[ux,vy]}\ket{ij}$ 
where $u,v,x,y,i,j$ are bits.
\end{example}

\frmrule

$(-1)^{u \wedge v} (-1)^{x \wedge y} \ket{i} \ket{j} = $\\
$(-1)^{u \wedge v + x \wedge y} \ket{ij} = $\\
$(-1)^{[u \wedge v + x \wedge y]\text{mod } 2} \ket{ij} = $\\
$(-1)^{u \wedge v \oplus x \wedge y} \ket{ij} = $
$(-1)^{[ux, vy]} \ket{ij} $


\frmrule

\begin{example}
Show that $(-1)^{[ux,vy]}\ket{ij} \otimes (-1)^{s \wedge t}\ket{k} = (-1)^{[uxs,vyt]}\ket{ijk}$ 
\end{example}

\frmrule

After doing the previous examples, it's now not too hard to what happens 
in the general case. 

$(-1)^{x_1 \wedge y_1}\ket{i_1} \otimes
(-1)^{x_2 \wedge y_2}\ket{i_2} \otimes
\dotsm \otimes
(-1)^{x_n \wedge y_n}\ket{i_n} = $ 

$(-1)^{x_1 \wedge y_1 + x_2 \wedge y_2 + \dotsm + x_n \wedge y_n}\ket{i_1 i_2 \dotsm i_n} = $

$(-1)^{[x_1 \wedge y_1 + x_2 \wedge y_2 + \dotsm + x_n \wedge y_n] \text{mod } 2}\ket{i_1 i_2 \dotsm i_n}$

$(-1)^{(x_1 \wedge y_1) \oplus (x_2 \wedge y_2) \oplus \dotsm \oplus (x_n \wedge y_n)}
\ket{i_1 i_2 \dotsm i_n} =$

$(-1)^{[x_1 x_2 \dotsm x_n, \; y_1 y_2 \dotsm y_n]}
\ket{i_1 i_2 \dotsm i_n}$

We will use this next.

\frmrule


\begin{example}
Determine $H^{\otimes n}\ket{x}$.
\end{example}


$H^{\otimes n}\ket{x}$

$= 
\frac{1}{\sqrt{2}} \left[(-1)^{x_1 \wedge 0} \ket{0} + (-1)^{x_1 \wedge 1} \ket{1} \right] \otimes
\frac{1}{\sqrt{2}} \left[(-1)^{x_2 \wedge 0} \ket{0} + (-1)^{x_2 \wedge 1} \ket{1} \right] \otimes
\dotsm 
\\ \otimes \frac{1}{\sqrt{2}} \left[(-1)^{x_n \wedge 0} \ket{0} + (-1)^{x_n \wedge 1} \ket{1} \right]
$

$= \sum_{y_1 y_2 \dotsm y_n \in \{0,1\}^n}
\frac{1}{\sqrt{2}} [(-1)^{x_1 \wedge y_1}\ket{y_1}] \otimes \frac{1}{\sqrt{2}} [(-1)^{x_2 \wedge y_2}\ket{y_2}] 
\otimes \dotsm \otimes \frac{1}{\sqrt{2}} [(-1)^{x_n \wedge y_n}\ket{y_n}]
$

$= 
\frac{1}{\sqrt{2}^n} \sum_{y_1 y_2 \dotsm y_n \in \{0,1\}^n}
(-1)^{x_1 \wedge y_1 + x_2 \wedge y_2 + \dotsm + x_n \wedge y_n} \ket{y_1 y_2 \dotsm y_n}
$

$= 
\frac{1}{\sqrt{2}^n} \sum_{y_1 y_2 \dotsm y_n \in \{0,1\}^n}
(-1)^{[x_1 \wedge y_1 + x_2 \wedge y_2 + \dotsm + x_n \wedge y_n] \text{mod } 2 } \ket{y_1 y_2 \dotsm y_n}
$

$= 
\frac{1}{\sqrt{2}^n} \sum_{y_1 y_2 \dotsm y_n \in \{0,1\}^n}
(-1)^{(x_1 \wedge y_1) \oplus (x_2 \wedge y_2) \oplus \dotsm \oplus (x_n \wedge y_n)} \ket{y_1 y_2 \dotsm y_n}
$

$= 
\frac{1}{\sqrt{2}^n} \sum_{y_1 y_2 \dotsm y_n \in \{0,1\}^n}
(-1)^{[x_1 x_2 \dotsm x_n, \; y_1 y_2 \dotsm y_n]} \ket{y_1 y_2 \dotsm y_n}
$



\highlightdef{
\textbf{Parallel Hadamard}: 
$H^{\otimes n}\ket{\textsf{x}} = 
\frac{1}{\sqrt{2}^n} \sum_{\textsf{y} \in \{0,1\}^n}
(-1)^{[\textsf{x}, \; \textsf{y}]} \ket{\textsf{y}}$  }

This is known as \textit{Fourier Samping}. 

\frmrule

\begin{example}
Using the previous result, what is $H^{\otimes n}\ket{\textsf{00..0}}$?
\end{example}

We insert $\textsf{x} = \textsf{00..0}$ into the Parallel Hadamard equation to get:\\
$H^{\otimes n}\ket{\textsf{00..0}} = 
\frac{1}{\sqrt{2}^n} \sum_{\textsf{y} \in \{0,1\}^n}
(-1)^{[\textsf{00..0}, \; \textsf{y}]} \ket{\textsf{y}}$
$ = \frac{1}{\sqrt{2}^n} \sum_{\textsf{y} \in \{0,1\}^n} \ket{\textsf{y}}$

\frmrule

This is a special case of Fourier Sampling where here, instead of 
using a general $\ket{\textsf{x}}$, we are using $\textsf{x} = \textsf{00..0}$. 
Notice how this gives a superposition of all $2^n$ computational basis vectors 
for an $n$ qubit system.

\highlightdef{An \textbf{$n$-Qubit Superposition}
is given by $H^{\otimes n}\ket{\textsf{00..0}}$
}
In other words, by connecting $n$ Hadamards in parallel each with input $\ket{0}$. 


\section{Bernstein-Vazirani Algorithm}

The \textit{parity problem}.

\highlightdef{
\textbf{Parity Problem}: 
Can we determine whether a function $f: \{0,1\}^n \rightarrow \{0,1\}$ 
is \textit{balanced or not} using $U_f$ only once in a given circuit $D$
}




\section{Deutsch Algorithm}

\highlightdef{
\textbf{Balanced Problem}: 
Can we determine whether a function $f: \{0,1\} \rightarrow \{0,1\}$ 
is \textit{balanced or not} using $U_f$ only once in a given circuit $D$
}

\highlightdef{The \textbf{Deutsch Algorithm} solves the \textit{Balanced Problem}}



\highlightdef{
\textbf{Generalized Balanced Problem}: 
Can we determine whether a function $f: \{0,1\}^n \rightarrow \{0,1\}$ 
is \textit{balanced or not} using $U_f$ only once in a given circuit $D$
}

\highlightdef{The \textbf{Deutsch-Jorza Algorithm} solves \textit{Generalized Balanced Problem}}


