
\chapter{Hilbert Spaces}


\section{Linear Maps}

\begin{example}
Below we define the \textit{two-point butterfly} operation.
$$(a,b,w) \mapsto (a + wb, a-wb)$$
Show that this is a \textit{linear} map from $\mathbb{C}^3 \rightarrow \mathbb{C}^2$. 
\end{example}


\highlightdef{\textbf{Dual}: 
The dual of a vector is a linear map that performs the inner product
}


\highlightdef{\textbf{Bilinear map}: 
$B(u,v)$, fixing $u$ gives a linear map $B_u(v)$ and fixing $v$ gives a linear map $B_v(u)$
}
Bilinear maps will become useful later when we look at \textit{tensor products}.

\section{Inner Products}


\highlightdef{\textbf{Hermitian Inner Product}: }.

This is often called the \textit{complex inner product}. 




\highlightdef{$\bra{u}\ket{v}$ is the same as $\braket{u}{v}$  }.

One might think that $\bra{u}\ket{v}$ is the multiplication of $u$ with $v$.
After all, we have a row vector next to a column vector. 
However this operation is not useful in Hilbert spaces.  
Instead we treat $\bra{u}\ket{v}$ as $\braket{u}{v}$. The complex 
inner product of two vectors is used far more and we will be seeing it often. 
The multiplication of $u$ with $v$ using dot products is not seen at all. 

We join the two vertical lines of $\bra{u}\ket{v}$ to form one vertical line. 
in $\braket{u}{v}$. We will rarely write $\bra{u}\ket{v}$ because it is 
slightly shorter to write $\braket{u}{v}$. 