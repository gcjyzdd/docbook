
\chapter{Quantum Algorithms II}


\section{Fourier Transform}


\highlightdef{\textbf{Root of Unity}: $\omega^j_n = e^{2\pi i\frac{j}{n}}$}

\highlightdef{\textbf{Phasor}: 
$w^j_n = (\omega^0_n,\omega^j_n,\omega^{2j}_n,...,\omega^{(n-1)j}_n) 
= \sum^{n-1}_{k=0} \omega^{kj}_n \ket{k}$}

\highlightdef{\textbf{Vandermonde Matrix}: $V_{n \times n} = [w^0_n w^1_n \dotsm w^{n-1}_n]$}

\begin{example}
Determine the inner product of $w^{j_1}$ and $w^{j_2}$ for $j_1 \neq j_2$. 
\end{example}

\begin{example}
Determine the inner product of $w^{j_1}$ and $w^{j_2}$ for $j_1 = j_2$. 
\end{example}


\begin{example}
Prove that $V_{n \times n}$ is unitary.
\end{example}


\begin{example}
Prove that $V_{n \times n}$ is unitary.
\end{example}


\highlightdef{\textbf{Quantum Fourier Transform}: $F = \frac{1}{\sqrt{2^n}}V_{2^n \times 2^n}$}

\begin{example}
What is the matrix for $F$ for $n = 1,2,3$?
\end{example}

\begin{example}
Prove that $F$ is unitary.
\end{example}




\begin{example}
Determine $\sum_{\textsf{t}_1 \textsf{t}_2 \dotsm \textsf{t}_n \in \{0,1\}^n} 
c^{\textsf{t}_1 \textsf{t}_2 \dotsm \textsf{t}_n} 
\ket{\textsf{t}_1 \textsf{t}_2 \dotsm \textsf{t}_n} = ...$ 
written in the form of a tensor product with $n$ factors.
\end{example}

\frameans{}{$\bigotimes^n_{l=1} \left(\ket{0} + c^{2^{n-l}} \ket{1} \right)$ }


\begin{example}
Determine $\sum_{\textsf{t}_1 \textsf{t}_2 \dotsm \textsf{t}_n \in \{0,1\}^n} 
c^{d \textsf{t}_1 \textsf{t}_2 \dotsm \textsf{t}_n} 
\ket{\textsf{t}_1 \textsf{t}_2 \dotsm \textsf{t}_n} = ...$ 
written in the form of a tensor product with $n$ factors.
\end{example}

\frameans{}{$\bigotimes^n_{l=1} \left(\ket{0} + c^{d2^{n-l}} \ket{1} \right)$ }


\frmrule

\[ 
\begin{array}{ll}
F\ket{j} = \frac{1}{\sqrt{2^n}} V_{2^n \times 2^n} \ket{j} & \text{def of } F  \\
         = \frac{1}{\sqrt{2^n}} w^{j}_{2^n} & \text{def of } V_{2^n \times 2^n}; j\text{th col} \\
         = \frac{1}{\sqrt{2^n}} \sum^{2^n-1}_{k=0} \omega^{kj}_{2^n} \ket{k} & \text{def of } w^{j}_{n}\\
         = \frac{1}{\sqrt{2^n}} \sum^{2^n-1}_{k=0} e^{2\pi i\frac{kj}{2^n}} \ket{k} & \text{def of } \omega^{j}_{n} \\
         = \frac{1}{\sqrt{2^n}} \sum_{\textsf{k}_1 \textsf{k}_2 \dotsm \textsf{k}_n \in \{0,1\}^n} 
                   e^{\frac{2 \pi ij}{2^n} \times \textsf{k}_1 \textsf{k}_2 \dotsm \textsf{k}_n} 
                   \ket{\textsf{k}_1 \textsf{k}_2 \dotsm \textsf{k}_n} & \text{writing }k\text{ in binary} \\
        %
		= \frac{1}{\sqrt{2^n}} \bigotimes^n_{l=1} (\ket{0} + e^{\frac{2\pi i j}{2^n} 2^{n-l}} \ket{1})
		& \text{prev ex with } c = e, d = \frac{2\pi i j}{2^n} \\
		%
		= \frac{1}{\sqrt{2^n}} \bigotimes^n_{l=1} (\ket{0} + e^{\frac{2\pi i j}{2^l}} \ket{1})
		 & \text{cancelling }2^n \\
		= \frac{1}{\sqrt{2^n}} \bigotimes^n_{l=1} (\ket{0} + e^{\frac{2\pi i}{2^l} \sum^{n}_{r=1} 2^{n-r}
         \textsf{j}_r} \ket{1}) & \text{writing }j\text{ in binary} \\
		%
		= \frac{1}{\sqrt{2^n}} \bigotimes^n_{l=1} (\ket{0} + e^{\frac{2\pi i}{2^l} 
		[\sum^{l}_{r=1} 2^{n-r} \textsf{j}_r + \sum^{n}_{r=l+1} 2^{n-r} \textsf{j}_r]}  \ket{1}) & 
		\text{breaking up least sig } l \text{ bits} \\
		%
		= \frac{1}{\sqrt{2^n}} \bigotimes^n_{l=1} (\ket{0} + e^{2\pi i
		[\sum^{l}_{r=1} 2^{n-r-l} \textsf{j}_r + \sum^{n}_{r=l+1} 2^{n-l+1} \textsf{j}_r]}  \ket{1}) & 
		\text{binary shift right by } l \text{ bits} \\
		%
		= \frac{1}{\sqrt{2^n}} \bigotimes^n_{l=1} (\ket{0} + e^{2\pi i
		[K + \sum^{n}_{r=l+1} 2^{n-r-l} \textsf{j}_r]}  \ket{1}) & \text{where } K\text{ is an integer} \\
		%
		= \frac{1}{\sqrt{2^n}} \bigotimes^n_{l=1} (\ket{0} + e^{2\pi i
		\sum^{n}_{r=l+1} 2^{n-r-l} \textsf{j}_r}  \ket{1})  & e^{2\pi i K} = 1 \text{ for any integer } K \\
        %
        = \frac{1}{\sqrt{2^n}} \bigotimes^n_{l=1} (\ket{0} + e^{2\pi i
        \sum^{n-l-1}_{r=0} 2^{n-r+1} \textsf{j}_{r+l+1}}  \ket{1})  & \text{reindex summation }
\end{array}
\] 


\highlightdef{\textbf{QFT Output}: $F\ket{\textsf{j}_1 \textsf{j}_2 \dotsm \textsf{j}_n} 
= \frac{1}{\sqrt{2^n}} \bigotimes^n_{l=1} 
(\ket{0} + e^{2\pi i \sum^{l}_{r=1} 2^{r-l-1} \textsf{j}_r} \ket{1})$}

To implement the circuit we need a new gate, the \textit{controlled phase shift} gate.
This is a two qubit gate that adjusts the relative phase of $\ket{1}$ on the first qubit provided 
the second qubit is $\ket{1}$.

$$
R_k = \begin{bmatrix}
       1 & 0 & 0 & 0                        \\[0.3em]
       0 & e^{2\pi i \frac{1}{2^k}} & 0 & 0 \\[0.3em]
       0 & 0 & 0 & 0 \\[0.3em]
       0 & 0 & 0 & 1
     \end{bmatrix}
$$

In the \textsc{qft}, we are interested in adjusting the relative phase by factors of 
$e^{2\pi i \frac{1}{2^k}}$. In other words doing bit shift right by some $k$. 
The goal is to get  $e^{2\pi i \sum^{l}_{r=1} 2^{r} \textsf{j}_r }$ for each line in our 
parallel circuit using our gate $R_k$.

First line corresponds to $l = 1$. We need $\ket{\textsf{j}_1} \mapsto e^{2\pi i \sum^{1}_{r=1} 2^{r-1} \textsf{j}_1}$
which simplifies to $\ket{\textsf{j}_1} \mapsto e^{2\pi i 2^{1-1} \textsf{j}_1}$

\begin{example}
Given that $F$ is unitary, what is the matrix for the inverse fourier transform.
\end{example}


\section{Phase Estimation}




\section{Order Finding}

\section{Period Finding}


\highlightdef{$f_{a,N}(x) = a^x \text{mod } N$}

We want to find the order of $a$, the smallest 


\section{Discrete Logarithms}



A \textit{Discrete logarithm}, $dlog$, is borrowing the the idea of logarithm with real numbers 
and applying them to \textit{multiplicative cyclic groups}. 
They are sometimes called \textit{indexes}.

\begin{example}
Consider the group $\mathbb{Z}^{*}_{5}$ that has generator 2. \\ 
The discrete log of 1, $dlog(1) = 4$ because $2^4 = 1 \text{mod } 5$.
\end{example}

\frmrule

\section{Hidden Subgroup Problem}



\section{Grover's Algorithm}

\begin{example}
Show that the averaging operator is \textit{unitary}.
\end{example}


\section{Shor's Algorithm}