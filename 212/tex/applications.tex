\chapter{Applications}


\section{DNS I}

\frmrule 

\textit{Namespace tree.}


\frmrule 

\textit{Domains}

A domain is a set of zones. 
Because the zones are non-overlapping, there will be some namespace node 
that has the highest level. 
We often use uniquely identify the domain using this zone. 

\highlightdef{\textbf{Domain}: subtree of the namespace tree}

\frmrule 

\textit{Zones and the Delegate Relation}

We partition the nodes of the namespace tree into \textit{zones}. 
if two nodes are in the same zone, we require that there is a unique 
path connecting them - \textit{and} that every node along 
such a path is in the zone. Notice that this constaint tells 
us that every zone is a tree. 
The namespace node that is root of this tree is denoted by $d(Z)$. 
It is often caled the \textit{domain of the zone}. 


\highlightdef{
\textbf{Zone Partition}: a partition $\mathcal{Z}$ of a namespace nodes \\
where each block, $Z \in \mathcal{Z}$ satisfies:\\
(i) $n,n' \in Z$ implies unique path from $n$ to $n'$\\
(ii) every node along unique path is in $Z$
}

Recall from Discrete Maths that, a partition $\mathcal{P}$ of $S$ satisfies the following:\\
(i) $X \in \mathcal{P}$ implies $X \neq \emptyset$ (\textit{no block is empty})\\
(ii) $X,Y \in \mathcal{P}$ and $X \neq Y$ implies $X \cap Y = \emptyset$ (\textit{the blocks don't overlap})\\
(ii) $\cup_{X \in \mathcal{P}} = S$ (\textit{the blocks join to make $S$})\\

\highlightdef{
\textbf{Zone}: a block of a namespace node zone partition
}

Because the zones form a partition, there is a corresponding 
equivalence relation. But there is actually another underlying relation 
called the \textit{delegation relation}.
The Delegation Relation is a relation $D \subseteq Z \times Z$ 
on the set of zones $Z$. 
If we have $(z_1,z_2) \in D$ then we require that
all the name nodes in $z_2$ are children of name nodes in $z_1$. 
That is for each $n \in z_1$, there is a unique path to every $n' \in z_2$. 

This constraint forces the delegations relation to itself \textit{form a tree}. 
Thus can call the delegation relation, a \textit{delegation tree}.
The zone at the root of the delegation tree is called the \textit{root zone}.  

$(z_1,z_2) \in D$ means that $z_2$ was created 
by $z_1$ to take the delegated responsibility.

\frmrule 

\begin{example}
For the following namespace tree, explain whether the following partitions 
are \textit{zone partitions}. \\
(a) $\mathcal{Z} = \{\{a,b,c\},\{d,e\},\{f\}\}$
\end{example}


\frmrule 

\begin{example}
For the following namespace tree
(a) show that the given following partitions is a \textit{zone partitions}. \\
(b) find the underlying delegation relation\\
(c) state which zone is the root zone\\
\end{example}


\frmrule 

\textit{Recursive Lookup vs Iterative Lookup}

\frmrule 

\textit{DNS Resource Records (RRs)}

Each zone $z_i$ maintains a resource record. 
Each zone $z_i$ has a non-empty set of nameservers $S_i$. 
One server in $NS_p \in S_i$ is chosen to be the \textit{primary server}. 
All the remaining servers are in $S_i - \{S_i\}$ and are called \textit{secondary servers}. 
The servers are responsible for maintaining the resource record as well as 
handling requests to read the record. 

A DNS resource record (RR) is a three-tuple $(\textsf{n},\textsf{ttl},\textsf{v})$
where\\
\begin{itemize}
\item \textsf{n} the domain name node that is at the root of the zone, $d(Z)$
\item \textsf{ttl} \textit{time to live} - the number of seconds for which the RR may be cached. After this time 
expires, the record is considered to be out-of-date 
\item \textsf{v} - the domain name node name
% \item \textsf{t} \textit{type} - various types of records can be stored. We will mainly store 
% records for the domain name of the zone. But the RR can store other types of file that we will 
% see later are useful for more advanced DNS features. 
\end{itemize}

\frmrule 

\begin{example}
For the following namespace tree with the zones partition shown, 
give the DNS resource records for each zone. 
\end{example}

\frmrule 



\section{DNS II}

ARP Ping and DNS


\frmrule 

\textit{DNS Uses UDP and not TCP}

What is means is that if a DNS packet is lost, there is no automatic recover. 
At first it may seem like this causes a problem. 


\frmrule

In addition to being subject to loss, DNS packets have a maximum length,
potentially as low as 576 bytes. What happens when a DNS name to be looked
up exceeds this length? Can it be sent in two packets?

\frmrule

Can a machine with a single DNS name have multiple IP addresses? How
could this occur?

\frmrule

Can a computer have two DNS names that fall in different top-level domains?
If so, give a plausible example. If not, explain why not.


\section{Websites: HTTP I}
\section{Websites: HTTP II}
\section{Websites: HTTP III}

\section{Emails: Introduction}


\section{Emails: SMTP}

\section{Emails: POP3}