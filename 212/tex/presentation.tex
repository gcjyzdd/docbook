
\chapter{Character Coding}


\section{Character Encoding}

There are several entities that need to be formally defined.
First we will define the difference between a \textit{character} and a \textit{glyph}.

\begin{itemize}   
\renewcommand{\labelitemi}{$\Box$}
\item \textbf{Character} We give a formal definition of the term \textit{character} for 
encoding. Here, a \textit{character} is defined is an abstract name or description for a shape or form. A character has \textit{no concrete shape}. You cannot write a character down on paper.
A character cannot be printed on a computer screen. It's independent of any specific rendered image or font.
\item \textbf{Glyph} A \textit{glyph} is a concrete visual representation that we can assign
to a character. 
\item \textbf{Code Point} Each glyph is assign a unique natural number 
called a \textit{code point} or or \textit{code position}. 
\item \textbf{Binary Encoding} Each glyph has a unique binary sequence or digits 
assigned. This does not necessarily have to be the binary representation of the 
code point. The binary encoding must also determine  the number of leading zeros.
Some glyphs may be given different length binary encodings. 
\end{itemize}

A single character can have many glyphs associated.

\begin{example}
We define the character: \textit{the letter a}. 
It is just a name or description. A glyph is a concrete way of writing 
that can be assigned to \textit{the letter a}. 
\end{example}

A single glyph can have many characters associated. 

\begin{example}
We define two characters: \textit{the latin letter a}, \textit{the german letter a}. 
As an example, the following glyph can be assigned to \textit{the latin letter a}
or it can be assigned to \textit{the german letter a}. 
\end{example}

We can think of glyphs as being the syntax and characters as being the semantics.
Glyphs are the concrete written symbols whereas charactes are descriptions or names 
that give a meaning to these symbols. 

\begin{figure}[h]
\begin{tikzpicture}[
  title/.style={},
  entity/.style={rectangle, draw, text centered}
]
\node[entity] (ch) {Character};
\node[entity, right = 1.5cm of ch] (g) {Glyph};
\node[entity, right = 1cm of g] (cp) {Code Point};
\node[entity, right = 1cm of cp] (b) {Binary Encoding};

\draw(g) -- (cp);
\draw(cp) -- (b);
\draw(ch) -- (g);

% draw crowsfeet
\node[right = 0.2cm of ch] (cf1) {};
\node[left = 0.2cm of g] (cf2) {};
\draw(cf1.west) -- (ch.10);
\draw(cf1.west) -- (ch.0);
\draw(cf1.west) -- (ch.350);
\draw(cf2.east) -- (g.170);
\draw(cf2.east) -- (g.180);
\draw(cf2.east) -- (g.190);

\end{tikzpicture}
\end{figure} 

\frmrule




\section{ASCII}

The use of the term \textit{Extended ASCII} has been criticized. 
\begin{itemize}   
\renewcommand{\labelitemi}{$\Box$}
\item It can be mistakenly interpreted that the term \textit{Extended ASCII} 
is an update of the ASCII standard to include more than 128 characters. 
This is wrong. The ASCII standard 
contains 128 characters no more no less.
\item It can be mistakenly interpreted that the term \textit{Extended ASCII} 
unambiguously identifies a single encoding.
There are many encodings that may be considerd to \textit{extend} ASCII.
\end{itemize}
 

\frmrule


\section{CP-1252}

CP-1252 or \textit{Windows-1252} is a character encoding of the Latin alphabet, 
used by Windows in certain components.
CP-1252 is similar to ISO 8859-15 (8-bit ASCII) except that different characters 
are mapped to different code points. 

\frmrule


\section{Unicode I}





\highlightdef{\textbf{UTF} stands for \textit{Unicode Transfer Format}}


\begin{example}
For the following UTF-8 binary encoding, what is the corresponding codepoint?
\end{example}

\begin{example}
What is the UTF-8 encoding of the following codepoint?
\end{example}