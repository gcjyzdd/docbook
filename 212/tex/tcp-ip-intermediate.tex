\chapter{TCP/IP: Intermediate}




\section{Collision Detection}


CSMA/CD


\frmrule

\textit{The difference between a hub and a switch}





\section{Errors I}


\frmrule


\section{Transport Fields}

\begin{example}
Which fields of an IP header would a router change as it forwards it? 
Explain how they change and the reason for their change.

TTL: decremented at each router to stop packets circulating forever in network\\
Total length/Flags/Fragment Offset: changed to reflect the fragment of packet 
if traveling over network with smaller MTU\\
checksum: needs to be recomputed, fields change cause checksum to change
\end{example}

\frmrule

\begin{example}
Why does an IP packet perform a checksum on its \textit{header} but not its body?

Error control of the encapsulated body is the responsibility of the higher level protocol, 
not the responsibility of IP. \\
Even if IP did perform a checksum on the body, it wouldn't be able to do anything 
if it found an error. \\
Lower level protocols provide error control, 
so IP providing error control provides no services that doesn't already exist.
\end{example}



\section{TCP State Machine}


\section{TCP Congestion}


\frmrule 

\textit{Flow control vs. congestion control}

\highlightdef{
\textbf{Congestion Control}: aims not to overflow the \textit{network}\\
\textbf{Flow Control}: aims not to overflow the \textit{receiver}
}



\section{Discovering New Hosts}

DHCP