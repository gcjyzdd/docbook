\chapter{TCP/IP: Intermediate}



\section{NAT}

\frmrule 

\textit{Reserved Network Addresses}



\begin{sidenote}{The Exact Numbers}

$$\text{percentage} = \frac{s - b_0}{b_1-b_0} \times 100$$

$$\text{percentage} = \frac{\text{no. blocks}}{\text{total blocks}} \times 100$$

\end{sidenote}



\frmrule 

\textit{Statically vs Dynamically Assigned IP Addresses}


\textit{Statically Assigned}:
good for machines that are on all the time, \\
provide services that don't move \\
require another level of access control 

\textit{Dynamically Assigned}:
connect intermittently\\
connect at different times to each other \\
connect to different networks



\frmrule 

\textit{Summary}

NAT violates the IP model, which states that every IP address uniquely 
identifies a single machine worldwide. 

NAT changes the internet from connectionless to connection-oritened 
because end-to-end connections depend on the state of NAT gateways. 

NAT violates protocol layering. The network layer makes 
assumptions about the transport layer by knowing about port 
numbers and remapping them for outgoing connections. 

NAT retricts the transport layer to TCP and UDP. 
Other transport layers like RTP are not necessarily supported. 

NAT fails with protocols that include internal IP addresses in the payload

NAT only supports $< 65536$ machines living behind a NAT router with a 
single externally visible IP address. 





\section{Network Fields}

\begin{example}
Which fields of an IP header would a router change as it forwards it? 
Explain how they change and the reason for their change.

TTL: decremented at each router to stop packets circulating forever in network\\
Total length/Flags/Fragment Offset: changed to reflect the fragment of packet 
if traveling over network with smaller MTU\\
checksum: needs to be recomputed, fields change cause checksum to change
\end{example}

\frmrule

\begin{example}
Why does an IP packet perform a checksum on its \textit{header} but not its body?

Error control of the encapsulated body is the responsibility of the higher level protocol, 
not the responsibility of IP. \\
Even if IP did perform a checksum on the body, it wouldn't be able to do anything 
if it found an error. \\
Lower level protocols provide error control, 
so IP providing error control provides no services that doesn't already exist.
\end{example}


\frmrule

\textit{Fragmenting Packets}


\highlightdef{
\textbf{More Fragments Flag}: $\textsf{MF}=0$ iff the fragment is the last fragment
}

\highlightdef{
\textbf{Don't Fragment Flag}: $\textsf{DF}=1$ prevents fragmentation from occuring
}




\frmrule




\section{TCP State Machine}


\section{TCP Congestion}


\frmrule 

\textit{Flow control vs. congestion control}

\highlightdef{
\textbf{Congestion Control}: aims not to overflow the \textit{network}\\
\textbf{Flow Control}: aims not to overflow the \textit{receiver}
}



\section{Discovering New Hosts}

DHCP