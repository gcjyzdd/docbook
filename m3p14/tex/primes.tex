
\chapter{Primes}

\section{Introducing Primes}


\textbf{Proof: Fundamental Theorem of Arithmetic} \\


\frmrule


\highlightdef{ $b | (p_1 p_2 \cdots p_n + 1)$ iff $b = \pm 1$ }

\textbf{Proof: There is no largest prime} \\
(proof there are infinitely many primes). 

Assume $p$ were the largest prime number.\\
Let $q$ be the product of all these $p$ prime numbers, $1 \times \cdots \times p$\\

$q + 1$ is not divisible by any of them. This is because .....	

\frmrule

Let $p_i$ be the $ith$ prime number from 1 to $p$. 
Assume $p_i | q + 1$. \\
Then $p_i = \pm 1$. \\
$1$ and $-1$ are not prime. $p_i$ is prime. We have a contradiction about the primality of $p_i$. \\
Hence $p_i \centernot\mid q + 1$. \\
$q + 1$ is not divisible by of the $p$ prime numbers.

\frmrule

Given that $q + 1$ is not divisible by of the $p$ prime numbers, 
the prime factorisation of $q + 1$ cannot use any of the $p$ prime numbers. \\
But we know every number, including $q + 1$, has a prime factorisation. 
Having assumed $p$ is the largest prime, we cannot find a factorisation 
for $q + 1$. This contradicts the fundamental theorem of arithmetic. 
We have a contradiction.

So $p$ cannot be the largest prime number. 


\frmrule


\highlightdef{ \textbf{No. of Divisors}: $(\alpha_1 + 1) \times (\alpha_2 + 1) \times \cdots \times (\alpha_n + 1)$  }

\begin{example}
Show that a number has an odd no. of divisors iff it is a square number. 
\end{example}

\frmrule