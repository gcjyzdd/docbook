
\chapter{Rings}




\section{Definition of a Ring}

Previously we have looked at groups where we combine an operator with a set. 
It often helped to think of a group as being \textit{additive} $(S, +)$ 
or \textit{multiplicative} $(S, \times)$. There is no algebriac difference betwen them, but we can 
tell that some operations \textit{feel} like addition and other operations \textit{feel} more 
like multiplication. We now introduce a new structure that will distinguish between addition 
and multiplication.

\highlightdef{A \textbf{Ring}, $R$ is a set with addition, subtraction and multiplication}

Or more formally, a ring $R$ is a algebraic structure $(S, +, \times)$,  
with a set of elements $S$ and two operators $+$ and $\times$. The \textit{addition operator}, 
forms a commutative group $(S,+)$ with the ring's elements. The \textit{multiplication operator} 
however does \textit{not} form a group. $(S,\times)$ has to be associative. However $(S,\times)$ does not 
have to have a multiplicative identity nor does it need to have all invertible elements. 


Notice that by our definition, a ring does \textit{not} have to have $1_R$. 
But it should be noted that some authors define a ring so that they must include a $1_R$.
There have been many names given to distinguish rings with a multiplicative identity from those 
which don't have one. 

In general, the term \textit{ring} can refer to either one of these two structures. 
A ring with a $1_R$ may be called: \textit{rings with unity, rings with identity, unit rings, ...}
A ring without $1_R$ may be referred to as: \text{Pseudo-rings, rings, unit rings}.
In our book, the term \textit{ring} will \textit{always} refer to structures without $1_R$. 

\highlightdef{In this book, a ring, $R$, \textit{does not} necessarily have $1_R$}



\section{Polynomial and Imaginary Rings}

In this section, we look at how, given ring R, we can make more rings.
The idea is to use the elements of R as coefficients or components for other objects.
And these objects will form \textit{another} ring. First we will look at \textit{polynomial rings}, 
where the elements of R are the coefficients of polynomials. We will see that the polynomials 
will form another ring.


\highlightdef{The \textbf{Polynomial Ring} of $R$, $R[x]$, is a \textit{new ring} of polynomials}

...where polynomials use coefficients from $R$. 

We write $R[x_1, x_2, ..., x_n]$ to denote the variables in the polynomial. Often we are not very 
interested in what these variables are - so we just write $R[x]$ as a shorthand for the 
polynomial ring of $R$. Also note, in general, polynomials can be of \textit{any degree}. 
We can go up to whatever $n^{n}$ we want. 

\begin{example}
Explain what the difference is between $R[x]$ and $\mathbb{R}[x]$ 
\end{example}




\highlightdef{An \textbf{Imaginary Ring} of $R$, $R[i]$ is a \textit{new ring} with elements $a + bi$}

...where $a$ and $b$ are elements in $R$. 

\begin{example}
Explain what the difference is between $R[i]$ and $\mathbb{R}[i]$ 
\end{example}


The ring $\mathbb{R}[i]$ is the same as the ring $(\mathbb{C}, +, \times)$ of complex numbers. 
Recall how we proved in the first section that complex numbers form a ring under normal 
numerical addition and multiplication. 

\begin{example}
State which of the following elements are in $\mathbb{Z}[i]$ 
\end{example}

The ring $\mathbb{Z}[i]$ is called the ring of \textit{Gaussian Integers}. If we draw an Argand Diagram, 
the Gaussian Integers are the complex numbers that occupy locations on a 1xi sized grid. 


\section{Units}


\begin{example}
Show that $1+2i$ is not a unit in $\mathbb{Z}[i]$. \\
We need to solve $x(1+2i) = 1_{\mathbb{Z}[i]}$ This gives $x = $ .................

\frameans{}{$x = \frac{1}{5} - \frac{2}{5}i$}

This is not an element in $\mathbb{Z}[i]$ because ....
\frameans{}{$a$ and $b$ are not in $\mathbb{Z}$}

\end{example}
