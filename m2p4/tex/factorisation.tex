
\chapter{Factorisation}




\section{Reducibility and Associates}


\highlightdef{A proper non-unit x is \textbf{reducible} iff x can written as a product of two non-units. }

\begin{example}
Show that 6 is reducible in $\mathbb{Z}$. When we write 6 as:  $6 = 3 \times 2$.
Now 3 ....... (is/is not) a unit
And 2 ....... (is/is not) a unit.
\frameans{}{is not, is not}

So 6 can be written as a product of two non-units. Hence 6 \textit{is} reducible.
\end{example}

\begin{example}
Show that $x^{2} - 1$ is reducible in $\mathbb{R}[x]$. \\
Using the difference of two squares we get the product ...............
\frameans{}{$(x+1)(x-1)$}
$(x+1)$ is not a unit because ......................\\
$(x-1)$ is not a unit because ......................\\
Hence $x^{2} - 1$ can be written as a product of two non-units and so \textit{is} reducible. 
\end{example}

\frameans{}{There is no $y$ such that $f_{x+1}(y) = 1$ and no $y$ such that $f_{x-1}(y) = 1$ }


\highlightdef{x is \textbf{irreducible} iff it is not reducible and x is not a unit. }

Here is another case where we exclude \textit{cheaters}. It would be very easy to call 
a unit irreducible. 

\begin{example}
The irreducible elements in Z are ...............
\end{example}
\frameans{}{$\pm p$ where $p$ is a prime number}

Be careful as this is not true in general.
\highlightdef{Irreducible is \textit{not the same} as prime.}


\begin{example}
Show that -5 is irreducible in $\mathbb{Z}$. \\
Firstly, -5 is not a unit because the only units in $\mathbb{Z}$ are .....
\end{example}
\frameans{}{1 and -1}








\begin{example}
Show that (1+5i) and (-5+i) are associates in $\mathbb{Z}[i]$.
\end{example}

\begin{example}
Find an associate for (2+3i) in $\mathbb{Z}[i]$. \\
To find an associate, we can multiply (2+3i) by a unit.
\end{example}

\begin{example}
Find an associate for 10 in $\mathbb{Z}$.  \\
To find an associate, we can 10 by a unit. The only units in $\mathbb{Z}$ are ....... 
Hence the associates for 10 are .......
\end{example}
\frameans{}{units: 1, -1 \\ associates: 10, -10}


\highlightdef{Associates are \textit{not the same} as multiplicative inverses. }
This is only true when the unit connecting the associates is 1. 


\section{Unique Factorisations}


\section{Euclidean Domains}

