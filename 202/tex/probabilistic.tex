
\chapter{Randomized Algorithms}


\section{Probabilistic Algorithms}

Often when a problem is intractable, we may 
be able to solve the problem by using a style that makes it
known as a \textit{probabilistic algorithm}. This can be done rather than 
using a deterministic algorithm. As the name suggests, the correctness of 
probabilistic algorithms is based on a probability. 
The algorithm itself has been randomized and made non-deterministic. 

Although probabilistic algoriths cannot guarantee that their 
output is error-free, techniques can be applied to reduce the probability of error 
to a value that is small enough for the application.


\begin{figure}[h]
\begin{tikzpicture} [
   every node/.style={node_sty, align=center},
   edge from parent/.style={edge_sty},
   edge from parent path={(\tikzparentnode.south) 
   |- ($(\tikzparentnode.south)!0.5!(\tikzchildnode.north)$) -| (\tikzchildnode.north)},
   level 1/.style={sibling distance=5cm, level distance = 0.9cm},
   level 2/.style={sibling distance=3cm},
   ch/.style={text width=2cm}
]
   \node 
   (root) {Probabilistic Algorithms}
      child {node {Las Vegas Algorithms}}
      child {
        node {Monte Carlo Algorithms}
        child {node[ch] {Yes-biased}}
        child {node[ch] {No-biased}}
      };
   
\end{tikzpicture}
\end{figure}


\begin{itemize}   
\renewcommand{\labelitemi}{$\Box$}
\item \textbf{Monte Carlo Algorithms}
A \textit{Monte Carlo Algorithm} is a yes/no decision algorithm.
When \textit{yes-biased} Monte Carlo Algorithm outputs yes, 
the probability it is correct is 1. And \textit{yes-biased} 
Monte Carlo Algorithm outputs no, the probability it is correct is $e$ 
(there may be an error, $e \leqslant 1$). 
When \textit{no-biased} Monte Carlo Algorithm outputs no, 
the probability it is correct is 1, and when it outputs 
yes, the probability it is correct is $e$. 

\item \textbf{Las Vegas Algorithms}
A \textit{Las Vegas} algorithm may not give an answer at all. 
There are three outputs: yes, no, or null. 
When a \textit{Las Vegas} outputs yes, the probability it is correct is $1$.
When a \textit{Las Vegas} outputs no, the probability it is correct is $1$.
When a \textit{Las Vegas} outputs null, the probability it is correct is $0$.
Here the error is whether we fall into the null case. The probability that a 
yes/no answer is returned is $e$ (there may be an error, $e \leqslant 1$). 
\end{itemize}


