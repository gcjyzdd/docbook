
\chapter{String Matching}


\section{Arrays as Lists Revision}

Recall that an array that we used in languages (like Java, C and C++)
can be \textit{treated as a list} (like in Haskell) for the convenience of reasoning 
about algorithms. 

We write $A[i..j]$ to denote the sublist $[A[i], A[i+1], \cdots, A[j-1]]$. 
In other words, it is the \textit{sublist from $i$ to $j$}, following our usual 
convention on regions in regards to off-by-one errors.
That is, we include the element at $i$ and continue up to, but \textit{do not include}, element $j$.

\begin{figure}[h]
\begin{tikzpicture} [node distance=0mm]
  \begin{scope}[start chain=1, every node/.style={draw,on chain=1,minimum height=0.65cm}]
  {
    \node[minimum width=2cm] (r1) {};
    \node[minimum width=4cm, fill=black!10] (r2) {$A[i..j]$};
    \node[minimum width=2cm] (r3) {};
  }
  \end{scope}
  \node[below=0.1cm of r1.south west] {0};
  \node[below=0.1cm of r1.south east] {$i$};
  \node[below=0.1cm of r2.south east] {$j$};
  \node[below=0.1cm of r3.south east] {a.length};
  \node[left=0.3cm of r1] {$A$};
\end{tikzpicture}
\end{figure} 

\frmrule

\begin{example}
What is the length of the following lists. \\
\textbf{(a)} $A[i..i]$ \textbf{(b)} $A[i..i+1]$ \textbf{(c)} $A[1..3]$ 
\end{example}

\frameans{}{\textbf{(a)} 0 \textbf{(b)} 1 \textbf{(c)} 2  }


\begin{example}
What is the length of $A[i..j]$ ?
\end{example}

\frameans{}{$j-i$}


\section{Introducing String Matching}



\section{Knuth-Morris-Pratt Algorithm}


\section{Boyer-Moore Algorithm}