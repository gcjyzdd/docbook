
\chapter{Dynamic Programming}


\section{Recursive Decomposition}



\textbf{SubsetSum Problem}









\textbf{Independent Set Problem}


\begin{itemize}  
\renewcommand{\labelitemi}{$\Box$}
\item Assume the $i$th node \textit{is not} in the independent set. Then we can still have 
an independent set for $i+1$ nodes provided if $yn_{i}$ holds. 
\item Assume $i$th node does form an independent set. The previous node is adjacent and 
so cannot be in the independent set. 
So we know $yn_{i}$ must be false. However can have $yn_{i-2}$. 
\end{itemize} 
Hence $\text{yn}_{i+2} = \text{yn}_{i+1}$ or  $\text{yn}_{i}$

\textbf{Vertex Cover Problem}

\frmrule

\textbf{Scalar Product Problem}

\frmrule

\textbf{Knapsack Problem}

Given $m$ items with valus $v_i$ and weights $w_i$, 
and knapsack weight limit $W$, is there a knapsack of 
value $\geqslant K$ that is not too heavy. 


\textbf{CoinSum Problem}

\frmrule

\textbf{Clique Problem}

\frmrule

\textbf{Rod Cutting}

Given a rod of positive length $L$, and length-to-price conversion 
function $f: 1 \mapsto p_1, ..., L \mapsto p_L$, 
is there a \textit{cutting} whose value is $\geqslant K$. 
[A \textit{cutting} of a rod of length $L$ is a sequence $X = [l_1, ..., l_k]$ such 
that $l_i \in \{1...L\}$ and $\sum_i l_i = L$. The \textit{value} of a cutting is $\sum_i f(l_i)$ ] 



%Dependency graph is acyclic.
%There are no infinitely decreasing chains.
%There are $O(2^n)$ paths. 

\frmrule


\textbf{Maximum SubarraySum} 

Given a sequence of N numbers $X = [x_0, ... ,x_{n-1}]$ 
is a contiguous subsequence that has a sum $\geqslant K$?

In other words, the goal is to find the contiguous subarray which has 
the largest summation. Suppose we want to know $\text{dp}[i][s]$. 
If we know that $\text{dp}[i-1][s] = Y$ then $\text{dp}[i][s] = Y$ must also be true. 
Also, if $\text{dp}[i-1][s-x_i] = Y$ then $\text{dp}[i][s] = Y$. 

\frmrule


\textbf{Common Sequence Alignment Problem} 

The sequencr alignment problem is more classically known as 
the \textit{Longest Common Subsequence} problem or the LCS problem. 
It should be noted that here, a subsequence is noncontiguous. 

\highlightdef{We are looking at \textit{noncontiguous} subsequences}

Given two sequences $X = [x_0, ... ,x_{n-1}]$ and 
$Y = [y_0, ... ,y_{m-1}]$, is there a sequence $Z = [z_0, ... z_{l-1}]$ 
of length $l \geqslant K$ where we can insert zero or more blanks/wilcards to $Z$ to recover $X$ 
and insert zero or more blanks to recover $Y$.

Idea: let $\text{dp}[m][n]$ denote using $X[0,m]$ and $Y[0,n]$. 
To solve $\text{dp}[m+1][n+1]$ can we use $\text{dp}[m][n]$? 
Can we decompose the problem $\text{dp}[m+1][n+1]$ 
to a subprogram recurrence relation using by considering \textit{prefixes}? Unlike 
previous problems, the substructure is two-dimensional. 


\begin{figure}[h]
\begin{tikzpicture} [node distance=0mm]
  \begin{scope}[start chain=1, every node/.style={draw,on chain=1,minimum height=0.65cm}]
  {
    \node[minimum width=2cm] (r1) {};
    \node[minimum width=4cm, fill=black!10] (r2) {$Z[i..i+l]$};
    \node[minimum width=2cm] (r3) {};
  }
  \end{scope}
  \node[anchor=north, below=0.1cm of r1.south west] {0};
  \node[below=0.1cm of r1.south east] {$i$};
  \node[below=0.1cm of r2.south east] {$i+l$};
  \node[left=0.3cm of r1] {$X$};

  \begin{scope}[start chain=1, yshift=-2cm, every node/.style={draw,on chain=1,minimum height=0.65cm}]
  {
    \node[minimum width=3cm] (y1) {};
    \node[minimum width=4cm, fill=black!10] (y2) {$Z[j..j+l]$};
    \node[minimum width=1cm] (y3) {};
  }
  \node[draw=none, anchor=north, below=0.1cm of y1.south west] {0};
  \node[draw=none,below=0.1cm of y1.south east] {$j$};
  \node[draw=none,below=0.1cm of y2.south east] {$j+l$};
  \node[draw=none, left=0.3cm of y1] {$Y$};
  \end{scope}
\end{tikzpicture}
\end{figure} 


Suppose we are trying to solve $\text{dp}[m+1][n+1]$. 
Consider $\text{dp}[m+1][n]$ and $\text{dp}[m][n+1]$. is Y for length $l$, clearly 
$\text{dp}[m+1][n+1]$ must be Y for length $l$. Then if we know that $\text{dp}[m][n]$ is Y for 
length $l$, and that $X[m] = Y[n]$, then $\text{dp}[m][n]$ must have a Y for length $l+1$, 
and Y for length $l$. 

\frmrule

\textbf{Increasing Alignment} 

Given a sequence of numbers $X = [x_0, ... ,x_{n-1}]$ 
is there an \textit{increasing alignment}, of $X$ of length $\geqslant K$? 

[An \textit{alignment} of $X$ is obtained by deleting some number of elements (possibly zero) of $X$, 
leaving the remaining elements in their original order. 
$X = [x_1, ... x_m]$ is \textit{increasing} when $x_1 < x_2 < ... < x_m$. ]

\frmrule

Suppose that we want to solve $\text{dp}[n][l]$, we know that 
$\text{dp}[n-1][l]$ is Y, then $\text{dp}[n][l]$ is Y. 
Also if we knew that $\text{dp}[n-1][l-1]$ is Y, and 
that $x_{i-1} < x_i$ then $\text{dp}[n][l]$ is Y. 

\frmrule


\textbf{Zigzag Alignment Problem} (Top Coder: 4493/1259)

For this problem, a sequence of numbers, $Z$, is called a \textit{zig-zag sequence} 
iff the differences between successive numbers strictly alternate between positive and negative. 
The first difference (if one exists) may be either positive or negative. 
A sequence with fewer than two elements is trivially a zig-zag sequence.

For example, $Z = [1,7,4,9,2,5]$ is a zig-zag sequence because the differences (6,-3,5,-7,3) are alternately positive and negative. In contrast, $[1,4,7,2,5]$ and $[1,7,4,5,5]$ are not zig-zag sequences. 
The first is not because its first two differences are positive. 
The second is not because its last difference is zero.

Given a sequence of integers, $X$, return true iff there is a alignment of $X$ of length $\geqslant K$
that is a zig-zag sequence.  
[An \textit{alignment} of $X$ is obtained by deleting some number of elements (possibly zero) of $X$, 
leaving the remaining elements in their original order.]

\frmrule

% Zigzag Subarray (not alignment) Solution
% Suppose we want to solve $\text{dp}[n][l]$. 
% If we know that $\text{dp}[n-1][l]$ = Y, then clearly $\text{dp}[n][l]$ = Y. This is 
% the simple case where we do not include $x_i$ and can still say the answer is Y.

% Now the harder case. If we know that $\text{dp}[n-2][l-1]$ = Y,
% and that either:
% $x_{i-2} - x_{i-1} < 0$ and $x_{i-1} - x_{i} > 0$ (the last difference was negative and adding $x_i$ gives a positive 
% difference) or: $x_{i-2} - x_{i-1} > 0$ and $x_{i-1} - x_{i} < 0$ (the last difference was positive and adding $x_i$ gives 
% negative difference) - then $\text{dp}[n][l]$ = Y. 


\frmrule




\textbf{Shortest Path} 

Given an undirected weighted graph $G$ that has strictly positive weights, 
is there a path of length $\leqslant K$ from vertex 1 to vertex $n$?

\frmrule




\section{Optimal Solutions}



\textbf{Knapsack Problem}

Let $V^{*}(i,W)$ is value of the best knapsack 
given only the first $i$ items, and knapsack weight limit, $W$. 
\begin{itemize}  
\renewcommand{\labelitemi}{$\Box$}  
\item Assume item $i$ included, then $v^{*}_{[i,x]} = v^{*}_{[i-1,x-w_i]} + v_i$
\item Assume item $i$ excluded, then $v^{*}_{[i,x]} = v^{*}_{[i-1,x]}$
\end{itemize} 
Hence $v^{*}_{[i,x]} = \text{max}(v^{*}_{[i-1,x-w_i]} + v_i, v^{*}_{[i-1,x]})$
This is assuming, $w_i \leqslant x$. If we have $w_i > x$, then 
we must exclude it, $v^{*}_{[i,x]} = v^{*}_{[i-1,x]}$.


\highlightdef{
\textbf{Y/N Solutions $\rightarrow$ Optimum Solutions}:\\ 
Instead of using $\text{dp}[i][j]$ to store $YN(i,j)$, use it to store $V^{*}(i,j)$\\
Instead of using booleans to relate subproblems, use min/max 
}



\textbf{Coin Sum Problem}

Suppose we have coins $c_1, c_2, ..., c_G$, each are worth $w_1, w_2, ..., w_G$, 
You are given a target value $T$. Find the set of coins $C$ that has a 
total worth that sums to $T$ that uses the least number of coins. 


Let $S(C_T, T, v^{*}_{[C,T]})$. 
Take coin $c_1$
\begin{itemize}  
\renewcommand{\labelitemi}{$\Box$}  
\item Assume $c_1 \in C_T$ \\let $S(C_{T}-{c_1}, T-v(c_1), v^{*}_{[C_{T}-{c_1},T-v(c_1)]})$ 
then $v^{*}_{C,T} = v^{*}_{C_{T}-{c_1},T-v(c_1)} + 1$
\item Assume $c_1 \notin C_T$ \\let $S(C_{T}-{c_1}, T, v^{*}_{[C_{T}-{c_1},T]})$ 
then $v^{*}_{C,T} = v^{*}_{C_{T},T}$
\end{itemize} 
Take coin $c_2$
\begin{itemize}  
\renewcommand{\labelitemi}{$\Box$}  
\item Assume $c_1 \in C_T$ \\let $S(C_{T}-{c_2}, T-v(c_2), v^{*}_{[C_{T}-{c_2},T-v(c_2)]})$ 
then $v^{*}_{C,T} = v^{*}_{C_{T}-{c_2},T-v(c_2)} + 1$
\item Assume $c_1 \notin C_T$ \\let $S(C_{T}-{c_1}, T, v^{*}_{[C_{T}-{c_2},T]})$ 
then $v^{*}_{C,T} = v^{*}_{C_{T}-{c_2},T-v(c_2)}$
\end{itemize} 


Hence $v^{*}_n = \text{min}()$


There is a factor of $2G$ with lengths of at most $T$. 
So there are $O(T^{2G})$ paths. 



\section{Functional Solutions}



\section{Dynamic Programming}

% We may perform an exponential number of calls but we only solve the problem O(n)-times

 


\highlightdef{\textbf{Memoization}: first time you solve a subproblem,
\textit{cache its solution} in a global table}

this will give an O(1) lookup time later. 


\highlightdef{\textbf{Dynamic Programming}: solve a problem with a recursive optimum solution, 
using \textit{memoization} to prevent recomputating}

If we recompute over and over again, then we are essentially, 
doing a bruce-force. 

\begin{enumerate}
\item \textbf{Thought Experiment 1}: Find a recursive solution. 
Idea: assume the Y/N answer for the smaller problems, find 
the Y/N for the main problem. What to think: 
what exactly defines a smaller problem (what is our partial order)? 
Precisely cases of the smaller problems do we need? 
What is the relationship between the smaller problem cases to our main problem?
\item \textbf{Optimal Solution}: If we can solve a decision problem (Y/N problem)
that encodes an optimisation problem using DP, then it is likely 
we can solve the optimisation problem using DP. Can the problem be solved 
using an optimal subsolutions. Can we take max/min of the problem cases? 
\item \textbf{Function Solution}: Can we actually solve the functional 
problem. Can we find the items, nodes, coins, etc that yield the optimum?
Does the dp table implicitly give the items? Can we walk back through 
the table and recover the items?
\end{enumerate}