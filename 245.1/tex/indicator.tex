\chapter{Indicator Random Variables}



\section{Introducing Indicator Random Variables}


\section{Indicator Random Variables I}



\begin{example}
\textbf{Birthday Paradox Revisited:}\\
We revist the birthday paradox but now perform analysis using \textit{indicator random variables}. 
Suppose that there are $k$ people in a classroom. We can enumerate all $C(n,2)$ 2-sets $\{i,j\}$ 
classmates by enumerating all pairs $(i,j)$ $1 \leqslant i < j \leqslant k$. 
For each pair, define a random variable $X_{ij}$:
\[ 
\begin{array}{ll}
 X_{ij} &= \text{I}\{\text{person } i \text{ and person } j \text{ have the same birthday}\}  \\
        &= \left\{  \begin{array}{ll} 1 & \text{person } i \text{ and person } j \text{ have the same birthday}
      \\ 0 & \text{otherwise} \end{array} \\
\end{array}
\]

Now $E[X_{ij}] = \prob{\text{person } i \text{ and person } j \text{ have the same birthday}}$. 
We assume that birthdays are independent. That is, for any two people $i$ and $j$, $i < j$, and some date $r$: 
$$\prob{i \text{ has birthday } r | j \text{ has birthday } r} = \prob{i \text{ has birthday } r}$$ 

So $\prob{\text{person } i \text{ and person } j \text{ have the same birthday}}$ is equal to:
\[ 
\begin{array}{ll}
 E[X_{ij}=\sum^{n}_{r=1}\prob{i \text{ has birthday } r \cap j \text{ has birthday } r} & \\
 = \sum^{n}_{r=1}\prob{i \text{ has birthday } r} \cdot \prob{i \text{ has birthday } r | j \text{ has birthday } r} & \\
 = \sum^{n}_{r=1}\prob{i \text{ has birthday } r} \cdot \prob{i \text{ has birthday } r} &\\
 = \sum^{n}_{r=1} 1/n \cdot 1/n &\\
 = 1/n
\end{array}
\]

Let $X$ be a random variable that counts the number of pairs of individuals having the same birthday. 
We are intersted in the value of $E[X]$. Notice that using random variable algebra, 
we can say that $X = \sum^{k}_{i=1}\sum^{k}_{j=i+1}X_{ij}$. 
Hence: 
\[ 
\begin{array}{ll}
 E[X] = E[\sum^{k}_{i=1}\sum^{k}_{j=i+1}X_{ij}] & \\
 = \sum^{k}_{i=1}\sum^{k}_{j=i+1}E[X_{ij}] \\
 = ...
\end{array}
\]

\frameans{}{$k(k-1)/2n$}

Because:
\[ 
\begin{array}{ll}
 \sum^{k}_{i=1}\sum^{k}_{j=i+1}E[X_{ij}] \\
 = (1/n) \cdot \sum^{k}_{i=1}\sum^{k}_{j=i+1} 1 \\
 = (1/n) \cdot C(k,2) \\
 = (1/n) \cdot (k(k-1)/2)
\end{array}
\]

When $k(k-1) \geqslant 2n$ we have $E[X] \geqslant 1$. 
In other words, when $k(k-1) \geqslant 2n$, the expected number of 
people with the same birthday is at least 1. So we 
can solve $k(k-1) \geqslant 2n$ for $k$ to get $k \geqslant \sqrt{2n} + 1$. 
If we let $n = 365$ then $k \geqslant \sqrt{2\cdot 365} + 1$ which 
tells us we need $k \geqslant 28$ people.

\textit{Birthday Problem on Mars}: If let $n = 699$ (a year on Mars is 669 days long) then we need a classroom of $k \geqslant ...$ people.

\frameans{}{$38$}


\end{example}

We will look at the birthday problem in more detail later in the course. 

\frmrule 

\begin{example}
\textbf{Balls and Bins Revisited:}\\



Balls and bins will help us when we look at Hash Tables later in the course. 


\section{Indicator Random Variables II}

