
\chapter{Discrete Random Variables}







\section{Cumulative Functions}




\section{Transformations}


A \textit{predicate transformation}
defines a transformation between a first order logic 
sentence and a formal P expression. 
Given a sentence $\forall x : \mathbb{N} P(x)$ 
we have the following transformation.

$$\mathbb{P}[P(x)] \rightarrow \mathbb{P}[x \leqslant X]$$

We can allow more variables other than $x$ to be used. 
It is understood that these variables are quantified by a forall.




\section{Distributions}

\highlightdef{A \textbf{Probability Distribution} is 
a discrete \textsc{rv} with \textit{zero or more parameters}}


\section{Poisson Processes}


\highlightdef{A \textbf{Poisson Process} is a distribution 
$\text{Poi}(T, \lambda)$ whose whose mass 
function $p(x)$ determines 
$\mathbb{P}[
x \text{ \textit{arrivals with intervals numbered} } [t,t+T]
]$ ...}
... for all $t$ where $\lambda$ is called the \textit{arrival rate}, 
$T$ is the \textit{duration}. 
It is defined by the following:

$$
p(x) = \frac{e^{-\lambda}\lambda^{x}}{x!}
$$

Once we have fixed a $T$ and a $\lambda$, we write $X \sim \text{Poi}(T, \lambda)$ 
when $X(\omega) = $

\frmrule

\begin{example}
A router has an expected arrival rate of 200 packets per second.
Assuming a Poisson process, what is the probability that exactly
one call is received in an interval of ten seconds?
\end{example}




\frmrule

A poisson process has \textit{no state}. 
Knowing that the process has been running for time $T'$ doesn’t
affect how the distribution hehaves for the remaining time.

$\mathbb{P}[x \text{ \textit{arrivals in} } (t,t+T] | 
x \text{ \textit{arrivals in} } (t-T',t]]
= \mathbb{P}[x \text{ \textit{arrivals in} } (t,t+T]]$

This is known as the \textit{lack of memory property}. 
Attemping to storing a past history on arrivals to predict 
the future is therefore pointless. 


\section{Distribution Sums}




\section{Independent and Identically Distributed}


A collection of random variables may all have the 
\textit{same distribution function}, and, are pairwise independent 
of each other. This when we have \textit{independent and identically 
distributed random variables}. This is often abreviated as \textsc{iid}.

