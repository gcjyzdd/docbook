
\chapter{Disk and File}


\section{Introducing Disk Systems}




\highlightdef{A \textbf{Disk Cylinder} is a collection of}



\section{Unix Block Buffer Cache}

We can cache files in main memory for a speedup (compared to accessing the files 
from disk). The cache works in the usual manner, we have a cache hit case and a 
cache miss case.



\section{Unix inodes}


\begin{lstlisting}
typedef struct{
unsigned inode_number : 16; /* 2 bytes */
char file_name[14] : 112; /* 14 bytes */
} DIRECTORY_ENTRY
\end{lstlisting}

An inode number uniquely identifies an inode.
There is precisely one inode per file.

\section{Virtual File Systems}


\section{Unix Permissions}


\highlightdef{Unix has access levels: 
\textbf{r} for \textit{read},
\textbf{w} for \textit{write},
\textbf{x} for \textit{execution}}



\section{File System Recovery}

storage devices still mess up – they have
so-called bad blocks that make it hard to keep a file
system reliable.

simply backup the system regularly so that
parts of it can be restored when a bad block occurs.
The problem is how to do backups efficiently:

incremental dumps, by which changes are added
to the backup, say, every day
use doubling technique, such as doing writes to
two drives, but reading only from one.




