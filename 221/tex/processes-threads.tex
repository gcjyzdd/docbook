
\chapter{Processes and Threads}





\section{Introducing Process}


Right now, program \textit{during execution} can only be seen as a sequence of machine code instructions. Before code is executed we can see a program as source code but during 
execution there is no high level view or code to describes it's execution.

We \textit{need} this higher level view of a program running on a processor.
The execution of a program machine code instructions will lead to it requiring io devices.
Recall that use of io devices is complex and expensive. 
The operating system is resonsible for managing io devices as resources that 
are to be allocated to programs.

So, as a resource manager, an operating system should know how programs in 
execution affect resources. This is done with the idea of a \textit{process}. 
A \textit{process} is an invention by an operating system. 
It is used as a placeholder for executing programs so that we can keep a precise 
account of how program execution is affecting the use of hardware and software resources.

\highlightdef{A \textbf{Process} is the os's view of a program in execution}

As a virtual machine the os needs to provide easy access to the 
resources and as a processor-helper it
needs to provide a software-layer to communicate with devices.

In UNIX, 


\section{Execution-Schedule Management}