\chapter{Box Model}

\section{Introducing Box Model}

We introduce the CSS$/$2.1 box model. 



\highlightdef{
\textbf{Block Elements}: div, p, h1 ...\\
\textbf{Inline Elements}: span, ...
 }

Block elements are sometimes called \textit{block-level} elements.


\highlightdef{
\textbf{Nesting Rules}:\\
Blocks within blocks,\\
Inline within blocks,\\
Inline within inline.
 }

\begin{example}
Explain whether the following document tree 
has acceptable block/inline nesting

\end{example}


\highlightdef{\textsf{width}, \textsf{height} 
change the \textit{content area}
}
when a width or height is explicitly specified for any block-level element, 
it only determines the width or height of the visible element, aka, the 
\textit{content area}. We then apply
any padding, borders, and margins widths \textit{afterwards}. 

A common mistake is to includes the content, 
padding and borders within a specified width or height.
This results in a narrower or shorter rendering of a box 
than would result following the standard behavior.

The total width and height glued to a particular 
element is the \textit{computed width}. 
This is content area width/height, \textit{plus} 
the widths of the margins and borders. 



\highlightdef{
\textbf{Computed Width}: $W = w_c + w_b + w_m$ \\ 
\textbf{Computed Height}: $H = h_c + h_b + h_m$ }




\section{Collapsing Margins}




In simple terms, 
this definition indicates that when the 
vertical margins of two elements are touching, 
only the margin of the element with the largest margin value will be honored,
while the margin of the element with the smaller margin value will be collapsed to zero.


\highlightdef{\textbf{Collapsing Margins}: margin values can \textit{collapse to zero} }


\frmrule 

\textit{Collapsing Margins Between Adjacent Elements}

\begin{example}
$[(\textsf{div\#e1}, \{\textsf{margin} \mapsto 0 \}),$\\
$(\textsf{div\#e2}, \{\textsf{margin-top} \mapsto 40px, \textsf{margin-bottom} \mapsto 25px\})$\\
$(\textsf{div\#e3}, \{\textsf{margin-top} \mapsto 20px \})]$
\end{example}


\frmrule 

\textit{Collapsing Margins Between Parent and Child Elements}

\frmrule 

\textit{Negative margins}

In the case where one element has a negative margin, 
the margin values are added together to determine the final value. 

If both are negative, the greater negative value is used.


\section{IE Box Model Bug}






\frmrule 

\textit{IE7 and the collapsing margins}




\highlightdef{\textbf{Quirks Mode}: an \textit{incorrect} version of the box model}


\highlightdef{Using an \textit{incomplete doctype}, 
or having \textit{no doctype} will cause $\leqslant$ IE8 to
switch to quirks mode }


\frmrule 

\begin{example}
Explain which of the following doctypes will cause $\leqslant$ IE8 to switch to 
quirks mode.
\end{example}

\frmrule 


