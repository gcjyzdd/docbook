
\chapter{Relational Algebra}

\section{Introducing Relational Model}



\highlightdef{In the relational model, we have \textit{attributed tuples} rather than ordered tuples }


\highlightdef{\textbf{Tables} provides merely a \textit{convenient pictorial representation} for relations}




The \textit{tuple calculus} is a formalism of this idea 
of using the set-builder notation to find relations. 
A tuple calculus expression is of the form $\{t \;|\; P(t) \}$. 







\section{Relational Algebra I}



\highlightdef{\textbf{Intersection}: $A \cap B$}

\highlightdef{\textbf{Union}: $A \cup B$}



\highlightdef{\textbf{Projection}: $\pi_{\textsf{L}}(A)$}

\highlightdef{\textbf{Selection}: $\sigma_{\textsf{Q}}(A)$}





\section{Cartesian Products I}


For intersecting attributes, 
$\textsf{l}$ and $\textsf{l'}$


\highlightdef{\textbf{Cartesian Product}: $A \times B$ }



\section{Cartesian Products II}


To work out the \textit{equality cross product}, 
we take the \textit{cross product} of A and B but when we are forming the cross product, 
we only consider pairs of tuples that have the \textit{same values} for attributes in $\textsf{L}$. 


\highlightdef{\textbf{Equality Cartesian Product}: $A \equaltimes_{\textsf{L}}  B$ }

Rather than using all $|A||B|$ tuples, the equality cartesian product filters 
the tuples that are in the result based a comparison of the values for the attributes 
$\textsf{L}$. A pair $(a',b')$ is in $R$ only if 
$a\textsf{.l}_1 = b\textsf{.l}_1$, $a\textsf{.l}_2 = b\textsf{.l}_2$, ..., 
$a\textsf{.l}_n = b\textsf{.l}_n$. That is, only if they agree on values 
for all the attributes in $L$. 

\frmrule

\begin{example}
Evaluate the following \textit{equality cross product}.

$$\arraycolsep=2pt
\left\{
\ratuple {
    \textsf{a} & 1 \\
    \textsf{b} & 2
}
\ratuple {
     \textsf{a} & 3 \\
     \textsf{b} & 4
}
\ratuple {
     \textsf{a} & 5 \\
     \textsf{b} & 6
}
\right\}
\overset{\times}{=}_{\textsf{\{b\}}}
\left\{
\ratuple {
     \textsf{c} & 1 \\
     \textsf{d} & 2
}
\ratuple {
     \textsf{c} & 3 \\
     \textsf{d} & 4
}
\ratuple {
     \textsf{c} & 5 \\
     \textsf{d} & 6
}
\ratuple {
     \textsf{c} & 5 \\
     \textsf{d} & 6
}
\ratuple {
     \textsf{c} & 5 \\
     \textsf{d} & 6
}
\ratuple {
     \textsf{c} & 5 \\
     \textsf{d} & 6
}
\right\}
$$


\end{example}


\highlightdef{\textbf{Conditional Cartesian Product}: }


To work out the equality cross product




\section{Natural Joins I}


\highlightdef{Attribute Intersection: $\underline{A} \cap \underline{B}$}



\section{Natural Joins II}


\section{Further Operations}

\highlightdef{Division: $A \div B$}

\highlightdef{Group By:}

\highlightdef{Having:}
