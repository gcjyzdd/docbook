\chapter{Functional Dependencies}


\section{Functional Dependence I}


For a given relation 
$\textsf{R} \subseteq 
(\textsf{l}\text{:}X \;\times\; 
\textsf{l}'\text{:}X' \;\times\; 
...)
$ we define what it means for 
an attribute $\textsf{l}$ to \textit{functionally determine} 
another $\textsf{l}'$ in the relation. 

\highlightdef{\textbf{Functional Determine}: 
$\textsf{l}$ \textit{functionally determines} $\textsf{l}'$ in $\textsf{R}$,
iff the projection relation $p_{\textsf{l},\textsf{l}'}(A) \subseteq X \times X'$ is 
a \textit{function}. }

Let's look at an example.

\frmrule


\begin{example}
Let $X_1 = \{a,b,c,d\}$, $X_2 = \{f,g\}$, $X_3 = \{p,q,r,s\}$, $X_4 = \{u,v,w,x,y\}$\\
we define relation 
$\textsf{A} \subseteq (\textsf{a}\text{:}T_1 \;\times\; 
\textsf{b}\text{:}T_2 \;\times\; 
\textsf{c}\text{:}T_3 \;\times\; 
\textsf{d}\text{:}T_4)$ 

$$
\arraycolsep=2pt
\left\{
\ratuple {
    \textsf{a} & a \\
    \textsf{b} & g \\
    \textsf{c} & p \\
    \textsf{d} & u \\
}
\ratuple {
    \textsf{a} & a \\
    \textsf{b} & g \\
    \textsf{c} & q \\
    \textsf{d} & w \\
}
\ratuple {
    \textsf{a} & a \\
    \textsf{b} & g \\
    \textsf{c} & r \\
    \textsf{d} & y \\
}
\ratuple {
    \textsf{a} & b \\
    \textsf{b} & f \\
    \textsf{c} & s \\
    \textsf{d} & u \\
}
\ratuple {
    \textsf{a} & c \\
    \textsf{b} & g \\
    \textsf{c} & r \\
    \textsf{d} & v \\
}
\ratuple {
    \textsf{a} & c \\
    \textsf{b} & g \\
    \textsf{c} & p \\
    \textsf{d} & x \\
}
\ratuple {
    \textsf{a} & c \\
    \textsf{b} & g \\
    \textsf{c} & q \\
    \textsf{d} & x \\
}
\right\}
$$

\textbf{(a)} Find the projection relation $p_{\textsf{a},\textsf{b}}$. 
Hence show that $\textsf{a}$ functionally determines $\textsf{b}$ in \textsf{A}.\\
\textbf{(b)} Find the projection relation $p_{\textsf{a},\textsf{c}}$. 
Hence show that $\textsf{a}$ \textit{does not} functionally determines $\textsf{c}$ in \textsf{A}.
\end{example}



\begin{itemize}
\item 
To find $p_{\textsf{a},\textsf{b}}(\textsf{A})$, 
consider $\pi_{\textsf{a},{b}}(\textsf{A})$:

$$
\arraycolsep=2pt
\left\{
\ratuple {
    \textsf{a} & a \\
    \textsf{b} & g \\
}
\ratuple {
    \textsf{a} & b \\
    \textsf{b} & f \\
}
\ratuple {
    \textsf{a} & c \\
    \textsf{b} & g \\
}
\right\}
$$

Then $p_{\textsf{a},\textsf{b}}(\textsf{A})$ is given by the corresponding 
classical relation of ordered pairs:
$$
p_{\textsf{a},\textsf{b}}(\textsf{A}) = \{(a,g),(b,f),(c,g)\} \subseteq X_1 \times X_2
$$
This relation is indeed a \textit{function} because every input has exactly one associated output. 
Since $p_{\textsf{a},\textsf{b}}(\textsf{A})$ is a function,
by definition, $\textsf{a}$ functionally determines $\textsf{b}$ in \textsf{A}.
\item 
To find $p_{\textsf{a},\textsf{c}}(\textsf{A})$, 
consider $\pi_{\textsf{a},{c}}(\textsf{A})$:

$$
\arraycolsep=2pt
\left\{
\ratuple {
    \textsf{a} & a \\
    \textsf{b} & p \\
}
\ratuple {
    \textsf{a} & a \\
    \textsf{b} & q \\
}
\ratuple {
    \textsf{a} & a \\
    \textsf{b} & r \\
}
\ratuple {
    \textsf{a} & b \\
    \textsf{b} & s \\
}
\ratuple {
    \textsf{a} & c \\
    \textsf{b} & r \\
}
\ratuple {
    \textsf{a} & c \\
    \textsf{b} & p \\
}
\ratuple {
    \textsf{a} & c \\
    \textsf{b} & q \\
}
\right\}
$$

To find $p_{\textsf{a},\textsf{c}}(\textsf{A})$, we convert 
$\pi_{\textsf{a},\textsf{c}}(\textsf{A})$ from a set of unordered tuples a corresponding 
set of ordered tuples (using the order: $\textsf{a}$ then $\textsf{c}$ - as specified in the 
subscript of $p_{\textsf{a},\textsf{c}}$). This gives:
$$
p_{\textsf{a},\textsf{c}}(\textsf{A}) = \{(a,p),(a,q),(a,r),(b,s),(c,r),(c,p),(c,q)\} \subseteq X_1 \times X_3
$$
This relation is \textit{not} a function. Clearly, some inputs do not have 
a single possible output value (specifically, $a$ and $c$). 
So $\textsf{a}$ \textit{does not} functionally determines $\textsf{c}$ in \textsf{A}.
\end{itemize}


\frmrule



\highlightdef{
$A.\textsf{l}_j$ \textit{functionally dependent} $A.\textsf{l}_i$
iff $A.\textsf{l}_i$ functionally determines $A.\textsf{l}_j$
}

When $A.\textsf{l}_j$ is functionally dependent on $A.\textsf{l}_i$ 
we write $A.\textsf{l}_j \rightarrow A.\textsf{l}_i$ or 
just $\textsf{l}_j \rightarrow \textsf{l}_i$.

\frmrule

\frmrule


\section{Functional Dependence II}

We can extend the idea of functional determinance to hold for 
\textit{multiple attributes}. We generalise so that now 
a \textit{set of attributes} can functionally determine 
another \textit{sets of attributes}.

Given a relation 
$\textsf{R} 
\subseteq 
(
\textsf{l}_1\text{:}X_1 \;\times\; 
\textsf{l}_2\text{:}X_2 \;\times\; 
... \;\times\; 
\textsf{l}_n\text{:}X_n \;\times\; 
\textsf{l}'_1\text{:}X'_1 \;\times\; 
\textsf{l}'_2\text{:}X'_2 \;\times\; 
...
\textsf{l}'_n\text{:}X'_m
)
$ 
\\
Let $L = X_1 \times ... \times X_n$ 
and $L' = X'_1 \times ... \times X'_m$. Then:

\highlightdef{
\textbf{Multiple Attributes}:
$\{X_1, ..., X_n\}$ \textit{functionally determines} $\{X'_1, ..., X'_m\}$ in $\textsf{R}$,\\
iff the projection relation $p_{\textsf{l}_1,...,\textsf{l}_n, \textsf{l}'_1,...,\textsf{l}'_m}(\textsf{R}) 
\subseteq L \times L'$ \\
gives a \textit{function} 
$p_{\textsf{l}_1,...,\textsf{l}_n, \textsf{l}'_1,...,\textsf{l}'_m}(\textsf{R} )  : L \rightarrow L' $. }

In other words, given any $n$-unordered-tuple in $\textsf{R}$, $(l_1|a_1, ..., l_n|a_n)$ there exists 
a \textit{unique} $m$-unordered-tuple $(l'_1|b_1, ..., l'_m|b_m)$. 
The attributes $X_1 \times ... \times X_n$ determine $X'_1 \times ... \times X'_m$.
Let's look at an example.

\frmrule
 
\begin{example}
Let $A = \{a,b,c,d\}$, $F = \{f,g,h\}$, $P = \{p,q,r,s\}$, $U = \{u,v,w\}$, $X = \{x,y\}$\\
we define relation 
$\textsf{A} \subseteq 
(\textsf{a}\text{:}A \;\times\; 
\textsf{f}\text{:}F \;\times\; 
\textsf{p}\text{:}P \;\times\; 
\textsf{u}\text{:}U \;\times\; 
\textsf{x}\text{:}X)$ to have tuples:
$$
\arraycolsep=2pt
\left\{
\ratuple {
    \textsf{a} & a \\
    \textsf{f} & f \\
    \textsf{p} & p \\
    \textsf{u} & u \\
    \textsf{x} & x \\
}
\ratuple {
    \textsf{a} & a \\
    \textsf{f} & f \\
    \textsf{p} & p \\
    \textsf{u} & u \\
    \textsf{x} & y \\
}
\ratuple {
    \textsf{a} & b \\
    \textsf{f} & f \\
    \textsf{p} & q \\
    \textsf{u} & u \\
    \textsf{x} & x \\
}
\ratuple {
    \textsf{a} & c \\
    \textsf{f} & f \\
    \textsf{p} & q \\
    \textsf{u} & u \\
    \textsf{x} & y \\
}
\ratuple {
    \textsf{a} & c \\
    \textsf{f} & f \\
    \textsf{p} & p \\
    \textsf{u} & v \\
    \textsf{x} & x \\
}
\ratuple {
    \textsf{a} & a \\
    \textsf{f} & g \\
    \textsf{p} & r \\
    \textsf{u} & w \\
    \textsf{x} & x \\
}
\ratuple {
    \textsf{a} & a \\
    \textsf{f} & h \\
    \textsf{p} & q \\
    \textsf{u} & u \\
    \textsf{x} & x \\
}
\ratuple {
    \textsf{a} & a \\
    \textsf{f} & h \\
    \textsf{p} & q \\
    \textsf{u} & u \\
    \textsf{x} & y \\
}
\ratuple {
    \textsf{a} & d \\
    \textsf{f} & h \\
    \textsf{p} & s \\
    \textsf{u} & w \\
    \textsf{x} & y \\
}
\right\}
$$
\textbf{(a)} 
Show that $\alpha = \{\textsf{a}, \textsf{f} \}$ 
functionally determines
$\beta = \{\textsf{p}, \textsf{u} \}$
\\
\textbf{(n)} 
Show that $\alpha = \{\textsf{p}, \textsf{u} \}$ 
\textit{does not} functionally determine 
$\beta = \{\textsf{a}, \textsf{f} \}$

\end{example}


\section{Armstrong's Inference Rules}


\highlightdef{\textbf{Armstrong's Rules}: 
rules that hold for functional dependencies
}



\highlightdef{
\textbf{Soundness}: if we derive an fd then it is valid. \\
\textbf{Completeness}: if an fs is valid, then it can be derived
}

To visualise soundness and completeness, consider two
sets $R^{+}$ all the fds that hold for $R$ 
and $S^{+}$, all the fds that can possibly 
be written down starting from formulas in $S$. 




% Recall that an \textit{axiom} in logic is 
% a rule that are so fundamental 
% that they can proven from first principles or is 
% a rule that is assumed to be true because 
% any proof would seem too trivial.  



\section{Set Closures}


\section{Set Covers}