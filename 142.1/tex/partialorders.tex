
\chapter{Partial Order Relations}




\section{Introduction}



\highlightdef{\textbf{Partial Order Relation:} }

\highlightdef{\textbf{Partially Ordered Set:} }

\highlightdef{\textit{Antisymmetric} is not the same as: \textit{Not Symmetric} }



\highlightdef{\textbf{Hasse Diagram:} Diagram useful for visualising posets}

\highlightdef{
\textit{Reflexive loops} and \textit{transitive edges} are not drawn in Hasse diagrams\\
\textit{Arrowheads} are not drawn - instead we draw $b$ higher than $a$ when $a \leqslant b$}



\section{Maxima and Minima}


\highlightdef{\textbf{Maximal Element}: $x$ comparable to $y$ implies $x \geqslant y$ }
An element $x$ is maximal when it is greater than or equal to all the elements in the poset 
that are comparable to $x$.

\highlightdef{\textbf{Minimal Element:} $x$ comparable to $y$ implies $x \geqslant y$ }
An element $x$ is minimal when it is less than or equal to all the elements in the poset 
that are comparable to $x$.

\highlightdef{\textbf{Minimum Element:} $x$ comparable to $y$ and $x \geqslant y$ }
\highlightdef{\textbf{Maximum Element:} $x$ comparable to $y$ and $y \geqslant x$ }

Maximum elements sometimes called \textit{greatest elements}. 
Minimum elements sometimes called \textit{least elements}. 

\highlightdef{There can be either zero or one Minimum element\\
There can be either zero or one Maximum element}


\highlightdef{There is at least one minimal element\\
There is at least one maximal element
}

There may indeed be several minimal elements and several maximal elements. 




\highlightdef{\textbf{Total Order Relation:} }
\highlightdef{\textbf{Totally Ordered Set:} }



\begin{example}
For a poset, given that there is a path from $x$ to $y$, which of the following can we definitely say:
(a) there is a path from $y$ to $x$ (b) there is \textit{not a path} from $y$ to $x$
(c) $x$ and $y$ are \textit{incomparable}    (d) $x$ and $y$ are \textit{comparable}
\frameans{}{(b),(d)}
\end{example}


\section{Topological Sorts}


\highlightdef{\textbf{Well-ordered Poset:} $(X,\leqslant)$ is well-ordered iff $X$ is well-founded 
and $\leqslant$ is a total order relation.}

\highlightdef{\textbf{Well-founded Poset:} $(X,\leqslant)$ is well-founded iff
there are no infinitely descending chains of elements}




\highlightdef{Every directed acyclic graph can be seen as a well-founded poset}


\section{Dominating Paths}







\section{Lattices}

Take some subset of a poset. 

\highlightdef{\textbf{Upper bound}: An element of a poset subset is \textit{upper bound} iff it is the maximum in the subset}
That is a subset $Y$ of poset $P$ has $y$ as an \textit{upper bound} iff $\forall x \in Y : x \leqslant y$. 

\highlightdef{\textbf{Lower bound}: An element of a poset subset is \textit{lower bound} iff it is the minimum in the subset}
That is a subset $Y$ of poset $P$ has $y$ as an \textit{lower bound} iff $\forall x \in Y : x \leqslant y$. 

\frmrule 

\textit{Thinking informally about upper bounds using paths}





\highlightdef{The upper bound/lower bound \textit{does not} have to be in the poset subset}
And quite often lies \textit{outside} the original subset in question. 


\highlightdef{\textbf{Complete Lattice}: A poset is a \textit{complete lattice} when 
all subsets have least upper bounds as well as greatest lower bounds}




\highlightdef{\textbf{Meet} $a \wedge b = glb(\{a,b\})$,  \textbf{Join} $a \wedge b = lub(\{a,b\})$   }

\frmrule 

\textit{Introducing a Moore family}


\highlightdef{\textbf{Moore Family:} A \texit{Moore Family} is a subset $Y$ of a complete 
lattice such that  $\forall Z \subseteq Y : \meet{Z} \in Y$ }


Thus a Moore family \textit{always contains a least element}, that being $\meet{Z}$ 
and a greatest element $\meet{\emptyset}$ which is equal to the 
greatest element, $\top$ of the original complete lattice. 

A Moore family can \textit{never be empty}. 

\frmrule 

\begin{example}
Given the complete lattice $L = (P(S), \subseteq)$, where $S = \{1,2,3\}$. \\
Show that $Y = \{ \{2\}, \{1,2\}, \{2,3\}, \{1,2,3\} \}$ is a \textit{Moore family}.
\end{example}

\frmrule 

\begin{example}
Given the complete lattice $L = (P(S), \subseteq)$, where $S = \{a,b,c\}$. \\
Show that $Y = \{ \{a,b,c\}, \emptyset \}$ is a \textit{Moore family}.\\
\end{example}

\frmrule 

\begin{example}
Given the complete lattice $L = (P(S), \subseteq)$, where $S = \{x,y,z\}$. \\
Show that $Y = \{ \{ x \}, \{ y \} \}$ is \textit{not} a Moore family.
\end{example}

\frmrule 

\begin{example}
Given the complete lattice $L = (P(S), \subseteq)$, where $S = \{x,y,z\}$. \\
Show that $Y = \{ \emptyset, \{ x \}, \{ y \}, \{ x,y \} \}$ is \textit{not} a Moore family.
\end{example}


\section{Functions}




A function is 
injective \\
bijective \\
monotone  \\
additive  \\
multiplicative \\

\frmrule 

\highlightdef{\textbf{Completely Additive}: A function is \textit{completely additive} 
if for all $Y \subseteq L_1$, $f(join_1(Y)) = join_2(\{f(y) | y \in L_1 \} )$}



\section{Fixed Points}



\highlightdef{\textbf{Reductive of $f$}: The set of elements upon which $f$ is reductive }
$Red(f) = \{x \;|\; f(x) \leqslant x\}$ 

\highlightdef{\textbf{Extensive of $f$}: The set of elements upon which $f$ is extensive }
$Ext(f) = \{x \;|\; x \leqslant f(x)\}$ 


Since $L$ is a complete lattice, all subsets have greatest lower bounds 
(as well as least upper bounds). 
So the subset of $Fix(f)$ will also have a greatest lower bound. 
We shall call this the \textit{least fixed point} of $f$, written $lfp(f)$. 
Similarly $Fix(f)$ will also have a least upper bound
that we shall call the \textit{greatest fixed point} of $f$, written $gfp(f)$. 

\highlightdef{
\textbf{Least Fixed Point}: $lfp(f) = glb(Fix(f))$\\
\textbf{Greatest Fixed Point}: $gfp(f) = lub(Fix(f))$
} 


\section{M\"{o}bius Functions}


\highlightdef{\textbf{Arborescence}:
An \textit{Arborescence} is a partial ordering with a least element \\
and exactly one path from the least element to other elements }


If there is exactly one path from $x$ to $y$ then for $x \leqslant z \leqslant y$ ...
\highlightdef{
$\mu(x,z) = -1$ when $z$ is direct 
neighbour of $x$, \\ $\mu(x,z) = 0$ when $z$ is indirect neighbour  }

\section{M\"{o}bius Inversions}