\chapter{Git Internals I}




\section{Introducing Git}


\frmrule

\textit{What does git stand for?}

\highlightdef{Git has \textit{no particular abbreviation}}

\frmrule

\textit{Git vs GIT vs git}

At the command line we must, by default, type \lstinline{git} (all lowercase), but 
otherwise, \textit{all three} ways to 
write Git (Git, GIT, git) are popular and acceptable.
In this book, we will stick to using Git (and not use GIT or git).

\frmrule 


\section{Hashing I}


\frmrule 

\textit{Git objects}

In Git, an \textit{object} has particular concrete definition.

\highlightdef{\textbf{Object}: An \textit{object} is a hash of  }

A git object is either a \textit{blob object}, 
a \textit{tree object}, or a \textit{commit object}. 


\begin{figure}[h]
\begin{tikzpicture} [
  every node/.style={node_sty},
  edge from parent/.style={edge_sty},
  edge from parent path={(\tikzparentnode.south) -- ++(0, -0.2cm) -| (\tikzchildnode.north)},
  level/.style = {level distance = 0.6cm},
  level 1/.style={sibling distance=3.5cm, text width=1.5cm}
]
  \node (root) {Object}
    child{node (b) {Blob}}
    child{node (t) {Tree}}
    child{node (c) {Commit}};
\end{tikzpicture}
\end{figure}

Blob objects are often called \textit{blobs}. 
Likewise, tree objects are often called \text{trees} 


\frmrule 

\textit{Git as a content-addressable system}

\highlightdef{Git is a \textit{content-addressable system}}


Git hashes the \textit{contents} of objects. 
And so, when Git
places a file into a data structure called the \textit{object store}. 
It does this hashing based on the \textit{hash of the data} and not,
for example, on the hash of the filename. 

The files are not stored according to their filename, 
but rather by the hash of the data they contain. 
We call these files, \textit{blob objects}


and commit objects are often called \text{commits}. 

\begin{sidenote}{SHA1}
SHA1 (or SHA-1, with the dash between SHA and 1) is a \textit{cryptographic hash function}. 
\end{sidenote}






\frmrule 

\textit{Filepath of Objects - Overview}

Inside the \lstinline{.git} folder is a 
subfolder called \lstinline{objects}

\highlightdef{Each object's content address is 160 bits $=$ 40 hex digits }


\highlightdef{\textbf{Filepath}: Objects are stored in the path \lstinline{.git}$/$\lstinline{objects}$/h_0h_1/h_2,...,h_{39}$  }
where $h_0...h_{39}$ are the fourty hexidecimal digits 
that are the result of hashing the object's contents. 

Git uses the first 2 hex digits of the hash as a directory name, then the remaining 38 as a filename. 
This is so that we don't end up with too many files in a single directory. 
By inserting a / after the first two digits, we improve the efficiency for a lot of filesystems
that slow when many files in the same directory. 

The exact preimage of the hash is determined by whether 
the object is a blob object, tree object or commit object as the 
Depending on the type of object, we 
have a different structure of plaintext to hash.
The next examples demonstrate.  

\frmrule 

\begin{example}
Show that a \textit{blob object} with contents \lstinline{sweet} has a file path of:\\
\lstinline{.git/objects/aa/823728ea7d592acc69b36875a482cdf3fd5c8d}

\begin{itemize}
\item
If we take the SHA1 hash of:
$$[blob,\textsc{sp},6,\textsc{nul}] + [sweet,\textsc{lf}]$$
then we find the hash is:
$$h = aa823728ea7d592acc69b36875a482cdf3fd5c8d$$


So the path is \lstinline{.git}$/$\lstinline{objects}$/h_0h_1/h_2,...,h_{39}$
which is equal to \lstinline{.git/objects/aa/823728ea7d592acc69b36875a482cdf3fd5c8d}.

Here, \textsc{sp},\textsc{nul} and \textsc{lf} denote 
the space, zero-byte and linefeed characters respectively. We 
can find the hash using for example,\\ \lstinline{printf "blob 6\000sweet\n" | sha1sum}
\end{itemize}

\end{example}



\frmrule 

\begin{example}
Show that a \textit{tree object} with a tuple:
$(rose,\textsc{normal\textunderscore file},h_0)$
has path:\\
\lstinline{.git/objects/05/b217bb859794d08bb9e4f7f04cbda4b207fbe9}\\
where $h_0 = aa823728ea7d592acc69b36875a482cdf3fd5c8d$.\\
You may assume $\textsc{blob}$ is a constant equal to 100644.

\begin{itemize}
\item
If we take the SHA1 hash of:
$$[tree,\textsc{sp},32,\textsc{nul}] + 
[\textsc{blob},myfile,\textsc{nul},h_0]$$
then we get:
$$h = aa823728ea7d592acc69b36875a482cdf3fd5c8d$$


So the path is \lstinline{.git}$/$\lstinline{objects}$/h_0h_1/h_2,...,h_{39}$
which is equal to \lstinline{.git/objects/aa/823728ea7d592acc69b36875a482cdf3fd5c8d}.

\end{itemize}
\end{example}

\frmrule 

\textit{Tree objects as a list of tuples}

The previous example was for a tree object with \textit{one} tuple.
But, in general, tree objects have many tuples. They can be seen a \textit{list} of tuples.

\highlightdef{\textbf{Tree Object}: a \textit{tree object} is a list of tuples: $[($filetype, filename, hash$)]$}

The filetype of the tuple can vary. But we consider two cases:
\begin{itemize}
\item
$(n,\textsc{blob},h_0)$  where $n$ 
is the name of the file, $h_0$ is the hash of the blob object.
\item
$(n,\textsc{tree},h_0)$ where $n$ 
is the name of the directory, $h_0$ is the hash of the tree object.
\end{itemize}

In both cases $(n,t,h_0) \rightarrow [tree,\textsc{sp},32,\textsc{nul}] + [t,n,\textsc{nul},h_0]$
where $t$ is the filetype ($t \in \{\textsc{normal\textunderscore file}$, $\textsc{directory}\}$). 
Using these two types of tuple, we can build up a \textit{hierarchy} of tree objects. 

One tuples: $[tree,\textsc{sp},s,\textsc{nul}] + [t,n,\textsc{nul},h_0]$\\
Two tuples: $[tree,\textsc{sp},s,\textsc{nul}] + [t_0,n_0,\textsc{nul},h_0] + [t_1,n_1,\textsc{nul},h_1]$\\
Three tuples: $[tree,\textsc{sp},s,\textsc{nul}] + [t_0,n_0,\textsc{nul},h_0] + [t_1,n_1,\textsc{nul},h_1] + [t_2,n_2,\textsc{nul},h_2]$\\
$n$ tuples: $[tree,\textsc{sp},s,\textsc{nul}] +  [t_0,n_0,\textsc{nul},h_0] + ... + [t_{n-1},n_{n-1},\textsc{nul},h_{n-1}]$

That is, tree, with a space, followed by null, 
then followed by $n$ lots of $[t,n,\textsc{nul},h]$, 
one for each tuple where $n \geqslant 0$. 

\frmrule 

\begin{example}
Show that a \textit{tree object} with a tuples:\\
$[(house,\textsc{tree},h_0),(car,\textsc{blob},h_0),(dog,\textsc{blob},h_0)]$\\
has path:
\lstinline{.git/objects/05/b217bb859794d08bb9e4f7f04cbda4b207fbe9}
where \\$h_0 = aa823728ea7d592acc69b36875a482cdf3fd5c8d$,\\
$h_1 = aa823728ea7d592acc69b36875a482cdf3fd5c8d$,\\
$h_2 = aa823728ea7d592acc69b36875a482cdf3fd5c8d$.\\
You may assume $\textsc{blob}$ and $\textsc{tree}$ are constants equal to 100644
and , respectiely.
\end{example}

\frmrule 

\textit{Computing the hash of an empty list}



\frmrule 

\textit{Showing Hierarchies using Object Store Diagrams}


\highlightdef{\textbf{Tree Hierarchies}: each tree object's list of tuples: $[($filetype, filename, hash$)]$
represents the \textit{immediate children} }

That is, each one tuple for each direct child of the tree object. 
Each child may be either a tree object (recursive case) or a 
blob object (base case).



\frmrule 


\begin{sidenote}{Hash Chains}
Notice that to build our Hierarchies, we are using a hash $h_0$ to find a new hash $h$. 
This is an example of a \textit{hash chain}. Let $h(x)$ be a 
hash function, then a \textit{hash chain}
is $h(g(x))$ where $g(x)$ has $h(x')$ inside. 
Note that this definition is \textit{recursive} for $g(x)$ may itself, 
be a hash chain. 

For us, our $h(x)$ is \\
and our $g(x)$ is\\
\end{sidenote}




\frmrule 



\begin{example}


\end{example}


\frmrule 


\highlightdef{File and directory names are \textit{only stored in tree objects}.  }

A common mistake is to think that a filename is 
hashed to address a blob object. It is only the \textit{content} of the 
file that is used to form the address. This is why we call the addresses 
\textit{content} addresses.



\begin{sidenote}{git ls-tree}
For any git project, you can 
see the tuples of a tree object using \lstinline{ls-tree} (list tree). 
Two useful flags are:\\
(i) \lstinline{-r}: recursively show all subtrees (not just 
the immediate children). \\
(ii) \lstinline{-t}: output the trees while recursing 
\end{sidenote}


\frmrule 

\begin{example}
For the following blob-tree hierarchy, give an expression for 
the hash values $h_1$, $h_2$, $h_3$ of tree objects $T_1$, $T_2$, 
$T_3$ respectively. 
\end{example}



\section{Hashing II}



% "commit 158" NUL
% "tree 05b217bb859794d08bb9e4f7f04cbda4b207fbe9" LF
% "author Alice <alice@example.com> 1234567890 -0800" LF
% "committer Bob <bob@example.com> 1234567890 -0800" LF
% LF
% "Shakespeare" LF

\frmrule 

This is the first commit, so there are no parent commits, 
but later commits will always contain at least one line identifying a parent commit.

\frmrule 


\begin{sidenote}{DEFLATE}
Objects in git are 
compressed using $\textsc{deflate}$. The $\textsc{deflate}$ algorithm is a 
variation of the popular compression algorithm LZ77 
(The \textit{Lempel-Ziv 1977 Algorithm}).  $\textsc{deflate}$ is also used 
for compression in $\textsc{png}$ images.
\end{sidenote}




\frmrule 


\textit{Identical files have no effect on the Object Store}





\frmrule 

\textit{SHA1 hash collisions, will in practice, never happen }

The probability is so small. 

\begin{sidenote}{Birthday Paradox and Hash Collisions}

\end{sidenote}


\frmrule

\begin{example}
What is the maximum number of folders that can be inside 
\lstinline{.git/objects}?
\\
\textit{Note: by folders, we mean direct child folders.}\\
\textit{Hint: recall that we use hex digits $h_0h_1$ of 
hash $h_0...h_{39}$ of a git object's contents for the 
folder name. The remaining digits are used for the filename.}

There are 16 choices for $h_0$ and 16 choices 
for $h_1$. So by the multiplication principle 
of counting, we have 256 choices for $h_0h_1$. 
So there at most 256 possible folders. 
\end{example}

\frmrule

\begin{example}
The constant $\textsc{blob}$ $=$ 100644 is actually a Unix file permission flag.\\
Show that this flag corresponds to owner: read/write, group+other read.

100644 = 0b11000100100100100\\
So splitting gives: 011|000|100|100|100|100. \\
\end{example}
